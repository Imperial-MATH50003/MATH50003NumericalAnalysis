\documentclass[12pt,a4paper]{article}

\usepackage[a4paper,text={16.5cm,25.2cm},centering]{geometry}
\usepackage{lmodern}
\usepackage{amssymb,amsmath}
\usepackage{bm}
\usepackage{graphicx}
\usepackage{microtype}
\usepackage{hyperref}
\usepackage[usenames,dvipsnames]{xcolor}
\setlength{\parindent}{0pt}
\setlength{\parskip}{1.2ex}




\hypersetup
       {   pdfauthor = {  },
           pdftitle={  },
           colorlinks=TRUE,
           linkcolor=black,
           citecolor=blue,
           urlcolor=blue
       }




\usepackage{upquote}
\usepackage{listings}
\usepackage{xcolor}
\lstset{
    basicstyle=\ttfamily\footnotesize,
    upquote=true,
    breaklines=true,
    breakindent=0pt,
    keepspaces=true,
    showspaces=false,
    columns=fullflexible,
    showtabs=false,
    showstringspaces=false,
    escapeinside={(*@}{@*)},
    extendedchars=true,
}
\newcommand{\HLJLt}[1]{#1}
\newcommand{\HLJLw}[1]{#1}
\newcommand{\HLJLe}[1]{#1}
\newcommand{\HLJLeB}[1]{#1}
\newcommand{\HLJLo}[1]{#1}
\newcommand{\HLJLk}[1]{\textcolor[RGB]{148,91,176}{\textbf{#1}}}
\newcommand{\HLJLkc}[1]{\textcolor[RGB]{59,151,46}{\textit{#1}}}
\newcommand{\HLJLkd}[1]{\textcolor[RGB]{214,102,97}{\textit{#1}}}
\newcommand{\HLJLkn}[1]{\textcolor[RGB]{148,91,176}{\textbf{#1}}}
\newcommand{\HLJLkp}[1]{\textcolor[RGB]{148,91,176}{\textbf{#1}}}
\newcommand{\HLJLkr}[1]{\textcolor[RGB]{148,91,176}{\textbf{#1}}}
\newcommand{\HLJLkt}[1]{\textcolor[RGB]{148,91,176}{\textbf{#1}}}
\newcommand{\HLJLn}[1]{#1}
\newcommand{\HLJLna}[1]{#1}
\newcommand{\HLJLnb}[1]{#1}
\newcommand{\HLJLnbp}[1]{#1}
\newcommand{\HLJLnc}[1]{#1}
\newcommand{\HLJLncB}[1]{#1}
\newcommand{\HLJLnd}[1]{\textcolor[RGB]{214,102,97}{#1}}
\newcommand{\HLJLne}[1]{#1}
\newcommand{\HLJLneB}[1]{#1}
\newcommand{\HLJLnf}[1]{\textcolor[RGB]{66,102,213}{#1}}
\newcommand{\HLJLnfm}[1]{\textcolor[RGB]{66,102,213}{#1}}
\newcommand{\HLJLnp}[1]{#1}
\newcommand{\HLJLnl}[1]{#1}
\newcommand{\HLJLnn}[1]{#1}
\newcommand{\HLJLno}[1]{#1}
\newcommand{\HLJLnt}[1]{#1}
\newcommand{\HLJLnv}[1]{#1}
\newcommand{\HLJLnvc}[1]{#1}
\newcommand{\HLJLnvg}[1]{#1}
\newcommand{\HLJLnvi}[1]{#1}
\newcommand{\HLJLnvm}[1]{#1}
\newcommand{\HLJLl}[1]{#1}
\newcommand{\HLJLld}[1]{\textcolor[RGB]{148,91,176}{\textit{#1}}}
\newcommand{\HLJLs}[1]{\textcolor[RGB]{201,61,57}{#1}}
\newcommand{\HLJLsa}[1]{\textcolor[RGB]{201,61,57}{#1}}
\newcommand{\HLJLsb}[1]{\textcolor[RGB]{201,61,57}{#1}}
\newcommand{\HLJLsc}[1]{\textcolor[RGB]{201,61,57}{#1}}
\newcommand{\HLJLsd}[1]{\textcolor[RGB]{201,61,57}{#1}}
\newcommand{\HLJLsdB}[1]{\textcolor[RGB]{201,61,57}{#1}}
\newcommand{\HLJLsdC}[1]{\textcolor[RGB]{201,61,57}{#1}}
\newcommand{\HLJLse}[1]{\textcolor[RGB]{59,151,46}{#1}}
\newcommand{\HLJLsh}[1]{\textcolor[RGB]{201,61,57}{#1}}
\newcommand{\HLJLsi}[1]{#1}
\newcommand{\HLJLso}[1]{\textcolor[RGB]{201,61,57}{#1}}
\newcommand{\HLJLsr}[1]{\textcolor[RGB]{201,61,57}{#1}}
\newcommand{\HLJLss}[1]{\textcolor[RGB]{201,61,57}{#1}}
\newcommand{\HLJLssB}[1]{\textcolor[RGB]{201,61,57}{#1}}
\newcommand{\HLJLnB}[1]{\textcolor[RGB]{59,151,46}{#1}}
\newcommand{\HLJLnbB}[1]{\textcolor[RGB]{59,151,46}{#1}}
\newcommand{\HLJLnfB}[1]{\textcolor[RGB]{59,151,46}{#1}}
\newcommand{\HLJLnh}[1]{\textcolor[RGB]{59,151,46}{#1}}
\newcommand{\HLJLni}[1]{\textcolor[RGB]{59,151,46}{#1}}
\newcommand{\HLJLnil}[1]{\textcolor[RGB]{59,151,46}{#1}}
\newcommand{\HLJLnoB}[1]{\textcolor[RGB]{59,151,46}{#1}}
\newcommand{\HLJLoB}[1]{\textcolor[RGB]{102,102,102}{\textbf{#1}}}
\newcommand{\HLJLow}[1]{\textcolor[RGB]{102,102,102}{\textbf{#1}}}
\newcommand{\HLJLp}[1]{#1}
\newcommand{\HLJLc}[1]{\textcolor[RGB]{153,153,119}{\textit{#1}}}
\newcommand{\HLJLch}[1]{\textcolor[RGB]{153,153,119}{\textit{#1}}}
\newcommand{\HLJLcm}[1]{\textcolor[RGB]{153,153,119}{\textit{#1}}}
\newcommand{\HLJLcp}[1]{\textcolor[RGB]{153,153,119}{\textit{#1}}}
\newcommand{\HLJLcpB}[1]{\textcolor[RGB]{153,153,119}{\textit{#1}}}
\newcommand{\HLJLcs}[1]{\textcolor[RGB]{153,153,119}{\textit{#1}}}
\newcommand{\HLJLcsB}[1]{\textcolor[RGB]{153,153,119}{\textit{#1}}}
\newcommand{\HLJLg}[1]{#1}
\newcommand{\HLJLgd}[1]{#1}
\newcommand{\HLJLge}[1]{#1}
\newcommand{\HLJLgeB}[1]{#1}
\newcommand{\HLJLgh}[1]{#1}
\newcommand{\HLJLgi}[1]{#1}
\newcommand{\HLJLgo}[1]{#1}
\newcommand{\HLJLgp}[1]{#1}
\newcommand{\HLJLgs}[1]{#1}
\newcommand{\HLJLgsB}[1]{#1}
\newcommand{\HLJLgt}[1]{#1}


\def\endash{–}
\def\bbD{ {\mathbb D} }
\def\bbZ{ {\mathbb Z} }
\def\bbR{ {\mathbb R} }
\def\bbC{ {\mathbb C} }

\def\x{ {\vc x} }
\def\a{ {\vc a} }
\def\b{ {\vc b} }
\def\e{ {\vc e} }
\def\f{ {\vc f} }
\def\u{ {\vc u} }
\def\v{ {\vc v} }
\def\y{ {\vc y} }
\def\z{ {\vc z} }
\def\w{ {\vc w} }

\def\bt{ {\tilde b} }
\def\ct{ {\tilde c} }
\def\Ut{ {\tilde U} }
\def\Qt{ {\tilde Q} }
\def\Rt{ {\tilde R} }
\def\Xt{ {\tilde X} }
\def\acos{ {\rm acos}\, }

\def\red#1{ {\color{red} #1} }
\def\blue#1{ {\color{blue} #1} }
\def\green#1{ {\color{ForestGreen} #1} }
\def\magenta#1{ {\color{magenta} #1} }


\def\addtab#1={#1\;&=}

\def\meeq#1{\def\ccr{\\\addtab}
%\tabskip=\@centering
 \begin{align*}
 \addtab#1
 \end{align*}
  }  
  
  \def\leqaddtab#1\leq{#1\;&\leq}
  \def\mleeq#1{\def\ccr{\\\addtab}
%\tabskip=\@centering
 \begin{align*}
 \leqaddtab#1
 \end{align*}
  }  


\def\vc#1{\mbox{\boldmath$#1$\unboldmath}}

\def\vcsmall#1{\mbox{\boldmath$\scriptstyle #1$\unboldmath}}

\def\vczero{{\mathbf 0}}


%\def\beginlist{\begin{itemize}}
%
%\def\endlist{\end{itemize}}


\def\pr(#1){\left({#1}\right)}
\def\br[#1]{\left[{#1}\right]}
\def\fbr[#1]{\!\left[{#1}\right]}
\def\set#1{\left\{{#1}\right\}}
\def\ip<#1>{\left\langle{#1}\right\rangle}
\def\iip<#1>{\left\langle\!\langle{#1}\right\rangle\!\rangle}

\def\norm#1{\left\| #1 \right\|}

\def\abs#1{\left|{#1}\right|}
\def\fpr(#1){\!\pr({#1})}

\def\Re{{\rm Re}\,}
\def\Im{{\rm Im}\,}

\def\floor#1{\left\lfloor#1\right\rfloor}
\def\ceil#1{\left\lceil#1\right\rceil}


\def\mapengine#1,#2.{\mapfunction{#1}\ifx\void#2\else\mapengine #2.\fi }

\def\map[#1]{\mapengine #1,\void.}

\def\mapenginesep_#1#2,#3.{\mapfunction{#2}\ifx\void#3\else#1\mapengine #3.\fi }

\def\mapsep_#1[#2]{\mapenginesep_{#1}#2,\void.}


\def\vcbr{\br}


\def\bvect[#1,#2]{
{
\def\dots{\cdots}
\def\mapfunction##1{\ | \  ##1}
\begin{pmatrix}
		 \,#1\map[#2]\,
\end{pmatrix}
}
}

\def\vect[#1]{
{\def\dots{\ldots}
	\vcbr[{#1}]
}}

\def\vectt[#1]{
{\def\dots{\ldots}
	\vect[{#1}]^{\top}
}}

\def\Vectt[#1]{
{
\def\mapfunction##1{##1 \cr} 
\def\dots{\vdots}
	\begin{pmatrix}
		\map[#1]
	\end{pmatrix}
}}



\def\thetaB{\mbox{\boldmath$\theta$}}
\def\zetaB{\mbox{\boldmath$\zeta$}}


\def\newterm#1{{\it #1}\index{#1}}


\def\TT{{\mathbb T}}
\def\C{{\mathbb C}}
\def\R{{\mathbb R}}
\def\II{{\mathbb I}}
\def\F{{\mathcal F}}
\def\E{{\rm e}}
\def\I{{\rm i}}
\def\D{{\rm d}}
\def\dx{\D x}
\def\CC{{\cal C}}
\def\DD{{\cal D}}
\def\U{{\mathbb U}}
\def\A{{\cal A}}
\def\K{{\cal K}}
\def\DTU{{\cal D}_{{\rm T} \rightarrow {\rm U}}}
\def\LL{{\cal L}}
\def\B{{\cal B}}
\def\T{{\cal T}}
\def\W{{\cal W}}


\def\tF_#1{{\tt F}_{#1}}
\def\Fm{\tF_m}
\def\Fab{\tF_{\alpha,\beta}}
\def\FC{\T}
\def\FCpmz{\FC^{\pm {\rm z}}}
\def\FCz{\FC^{\rm z}}

\def\tFC_#1{{\tt T}_{#1}}
\def\FCn{\tFC_n}

\def\rmz{{\rm z}}

\def\chapref#1{Chapter~\ref{Chapter:#1}}
\def\secref#1{Section~\ref{Section:#1}}
\def\exref#1{Exercise~\ref{Exercise:#1}}
\def\lmref#1{Lemma~\ref{Lemma:#1}}
\def\propref#1{Proposition~\ref{Proposition:#1}}
\def\warnref#1{Warning~\ref{Warning:#1}}
\def\thref#1{Theorem~\ref{Theorem:#1}}
\def\defref#1{Definition~\ref{Definition:#1}}
\def\probref#1{Problem~\ref{Problem:#1}}
\def\corref#1{Corollary~\ref{Corollary:#1}}

\def\sgn{{\rm sgn}\,}
\def\Ai{{\rm Ai}\,}
\def\Bi{{\rm Bi}\,}
\def\wind{{\rm wind}\,}
\def\erf{{\rm erf}\,}
\def\erfc{{\rm erfc}\,}
\def\qqquad{\qquad\quad}
\def\qqqquad{\qquad\qquad}


\def\spand{\hbox{ and }}
\def\spodd{\hbox{ odd}}
\def\speven{\hbox{ even}}
\def\qand{\quad\hbox{and}\quad}
\def\qqand{\qquad\hbox{and}\qquad}
\def\qfor{\quad\hbox{for}\quad}
\def\qqfor{\qquad\hbox{for}\qquad}
\def\qas{\quad\hbox{as}\quad}
\def\qqas{\qquad\hbox{as}\qquad}
\def\qor{\quad\hbox{or}\quad}
\def\qqor{\qquad\hbox{or}\qquad}
\def\qqwhere{\qquad\hbox{where}\qquad}



%%% Words

\def\naive{na\"\i ve\xspace}
\def\Jmap{Joukowsky map\xspace}
\def\Mobius{M\"obius\xspace}
\def\Holder{H\"older\xspace}
\def\Mathematica{{\sc Mathematica}\xspace}
\def\apriori{apriori\xspace}
\def\WHf{Weiner--Hopf factorization\xspace}
\def\WHfs{Weiner--Hopf factorizations\xspace}

\def\Jup{J_\uparrow^{-1}}
\def\Jdown{J_\downarrow^{-1}}
\def\Jin{J_+^{-1}}
\def\Jout{J_-^{-1}}



\def\bD{\D\!\!\!^-}

\def\Abstract#1\par{\begin{abstract}#1\end{abstract}}
\def\Keywords#1\par{\begin{keywords}{#1}\end{keywords}}
\def\Section#1#2.{\section{#2}\label{Section:#1} }
\def\Appendix#1#2.{\appendix \section{#2}\label{Section:#1} }

\def\Subsectionl#1#2.{\subsection{#2}\label{subsec:#1}}
\def\Subsection#1.{\subsection{#1}}

\def\Subsubsection#1.{\subsubsection{#1}}


\def\Problem#1#2\par{\begin{problem}\label{Problem:#1} #2\end{problem}}
\def\Theorem#1#2\par{\begin{theorem}\label{Theorem:#1} #2\end{theorem}}
\def\Conjecture#1#2\par{\begin{conjecture}\label{Conjecture:#1} #2\end{conjecture}}
\def\Proposition#1#2\par{\begin{proposition}\label{Proposition:#1} #2\end{proposition}}
\def\Definition#1#2\par{\begin{definition}\label{Definition:#1} #2\end{definition}}
\def\Corollary#1#2\par{\begin{corollary}\label{Corollary:#1} #2\end{corollary}}
\def\Lemma#1#2\par{\begin{lemma}\label{Lemma:#1} #2\end{lemma}}
\def\Example#1#2\par{\begin{example}\label{Example:#1} #2\end{example}}
\def\Remark #1\par{\begin{remark*}#1\end{remark*}}

\def\figref#1{Figure~\ref{fig:#1}}

\def\Figurew[#1]#2#3\par{
\begin{figure}[tb]
\begin{center}{
\includegraphics[width=#2]{Figures/#1}}
\end{center}
\caption{#3}\label{fig:#1} 
\end{figure}
}

\def\Figure[#1]#2\par{
\begin{figure}[tb]
\begin{center}{
\includegraphics{Figures/#1}}
\end{center}
\caption{#2}\label{fig:#1} 
\end{figure}
}

\def\Figurefixed[#1]#2\par{
\Figurew[#1]{0.48 \hsize}{#2}\par
}

\def\Figuretwow#1#2#3#4\par{
\begin{figure}[tb]
\begin{center}{
\includegraphics[width=#3]{Figures/#1}\includegraphics[width=#3]{Figures/#2}}
\end{center}
\caption{#4}\label{fig:#1} 
\end{figure}
}

\def\Figuretwowframed#1#2#3#4\par{
\begin{figure}[tb]
\begin{center}{
\fbox{\includegraphics[width=#3]{Figures/#1}}\fbox{\includegraphics[width=#3]{Figures/#2}}}
\end{center}
\caption{#4}\label{fig:#1} 
\end{figure}
}

\def\Figuretwo[#1,#2]#3\par{
	\Figuretwow{#1}{#2}{0.48 \hsize}
		#3\par	
}

\def\Figuretwoframed[#1,#2]#3\par{
	\Figuretwowframed{#1}{#2}{0.48 \hsize}
		#3\par	
}

\def\Figurethreew#1#2#3#4#5\par{
\begin{figure}[tb]
\begin{center}{
\includegraphics[width=#4]{Figures/#1} \includegraphics[width=#4]{Figures/#2} \includegraphics[width=#4]{Figures/#3}}
\end{center}
\caption{#5}\label{fig:#1} %\prooflabel{#1}
\end{figure}
}

\def\Figurethree#1#2#3#4\par{
	\Figurethreew{#1}{#2}{#3}{0.3 \hsize}
		{#4}\par	
}

\def\Figurematrixfour#1#2#3#4#5\par{
\begin{figure}[tb]
\begin{center}{
\vbox{\hbox{\includegraphics[width= 0.48 \hsize]{Figures/#1} \includegraphics[width= 0.48 \hsize]{Figures/#2}}\hbox{\includegraphics[width= 0.48 \hsize]{Figures/#3}\includegraphics[width= 0.48 \hsize]{Figures/#4}}}}
\end{center}
\caption{#5}\label{fig:#1} %\prooflabel{#1}
\end{figure}
}


\def\questionequals{= \!\!\!\!\!\!{\scriptstyle ? \atop }\,\,\,}

\def\elll#1{\ell^{\lambda,#1}}
\def\elllp{\ell^{\lambda,p}}
\def\elllRp{\ell^{(\lambda,R),p}}


\def\elllRpz_#1{\ell_{#1{\rm z}}^{(\lambda,R),p}}


\def\sopmatrix#1{\begin{pmatrix}#1\end{pmatrix}}

\def\Proof{\begin{proof}}
\def\mqed{\end{proof}}

\gdef\reffilename{\jobname}
\def\References{\bibliography{\reffilename}}

\outer\def\ends{ 
\end{document}
}


\begin{document}



\textbf{Numerical Analysis MATH50003 (2024\ensuremath{\endash}25) Problem Sheet 3}

\textbf{Problem 1} What is $\ensuremath{\pi}$ to 5 binary places? Hint: recall that $\ensuremath{\pi} \ensuremath{\approx} 3.14$.

\textbf{SOLUTION} We subtract off powers of two until we get 5 places. Eg we have
\[
\ensuremath{\pi} = 3.14\ensuremath{\ldots} = 2 + 1.14\ensuremath{\ldots} = 2 + 1 + 0.14\ensuremath{\ldots} = 2 + 1 + 1/8 + 0.016\ensuremath{\ldots} = 2 + 1 + 1/8  + 1/64  + 0.000\ensuremath{\ldots}
\]
Thus we have $\ensuremath{\pi} = (11.001001\ensuremath{\ldots})_2$. The question is slightly ambiguous whether we want to round to 5 digits so either \texttt{11.00100} or \texttt{11.00101} would be acceptable. \textbf{END}

\textbf{Problem 2} What are the single precision $F_{32} = F_{127,8,23}$ floating point representations for the following: 
\[
2, \quad 31, \quad 32, \quad 23/4, \quad (23/4)\times 2^{100}
\]
\textbf{SOLUTION} Recall that we have \texttt{\ensuremath{\sigma},Q,S = 127,8,23}. Thus we write
\[
2 = 2^{128-127} * (1.00000000000000000000000)_2
\]
The exponent bits are those of
\[
128 = 2^7 = (10000000)_2
\]
Hence we get the bits 

\begin{verbatim}
0 10000000 00000000000000000000000
\end{verbatim}
We write
\[
31 = (11111)_2 = 2^{131-127} * (1.1111)_2
\]
And note that $131 = (10000011)_2$ Hence we have the bits

\begin{verbatim}
0 10000011 11110000000000000000000
\end{verbatim}
On the other hand,
\[
32 = (100000)_2 = 2^{132-127}
\]
and $132 = (10000100)_2$ hence we have the bits 

\begin{verbatim}
0 10000100 00000000000000000000000
\end{verbatim}
Note that
\[
23/4 = 2^{-2} * (10111)_2 = 2^{129-127} * (1.0111)_2
\]
and $129 = (10000001)_2$ hence we get:

\begin{verbatim}
0 10000001 01110000000000000000000
\end{verbatim}
Finally,
\[
23/4 * 2^{100} = 2^{229-127} * (1.0111)_2
\]
and $229 = (11100101)_2$ giving us:

\begin{verbatim}
0 11100101 01110000000000000000000
\end{verbatim}
\textbf{END}

\textbf{Problem 3} Let $m(y) = \min\{x \in F_{32} : x > y \}$ be the smallest single precision number greater than $y$. What is $m(2) - 2$ and $m(1024) - 1024$? 

\textbf{SOLUTION} The next float after $2$ is $2 * (1 + 2^{-23})$ hence we get $m(2) - 2 = 2^{-22}$:


\begin{lstlisting}
(*@\HLJLnf{nextfloat}@*)(*@\HLJLp{(}@*)(*@\HLJLnfB{2f0}@*)(*@\HLJLp{)}@*) (*@\HLJLoB{-}@*) (*@\HLJLni{2}@*)(*@\HLJLp{,}@*) (*@\HLJLni{2}@*)(*@\HLJLoB{{\textasciicircum}}@*)(*@\HLJLp{(}@*)(*@\HLJLoB{-}@*)(*@\HLJLni{22}@*)(*@\HLJLp{)}@*)
\end{lstlisting}

\begin{lstlisting}
(2.3841858f-7, 2.384185791015625e-7)
\end{lstlisting}


similarly, for $1024 = 2^{10}$ we find that the difference $m(1024) - 1024$ is $2^{10-23} = 2^{-13}$:


\begin{lstlisting}
(*@\HLJLnf{nextfloat}@*)(*@\HLJLp{(}@*)(*@\HLJLnfB{1024f0}@*)(*@\HLJLp{)}@*) (*@\HLJLoB{-}@*) (*@\HLJLni{1024}@*)(*@\HLJLp{,}@*) (*@\HLJLni{2}@*)(*@\HLJLoB{{\textasciicircum}}@*)(*@\HLJLp{(}@*)(*@\HLJLoB{-}@*)(*@\HLJLni{13}@*)(*@\HLJLp{)}@*)
\end{lstlisting}

\begin{lstlisting}
(0.00012207031f0, 0.0001220703125)
\end{lstlisting}


\textbf{END}

\textbf{Problem 4} Suppose $x = 1.25$ and consider 16-bit floating point arithmetic ($F_{16}$). What is the error in approximating $x$ by the nearest float point number ${\rm fl}(x)$? What is the error in approximating $2x$, $x/2$, $x + 2$ and $x - 2$ by $2 \otimes x$, $x \oslash 2$, $x \ensuremath{\oplus} 2$ and $x \ominus 2$?

\textbf{SOLUTION} None of these computations have errors since they are all exactly representable as floating point numbers. \textbf{END}

\textbf{Problem 5} Show that $1/5 = 2^{-3} (1.1001100110011\ensuremath{\ldots})_2$. What are the exact bits for $1 \ensuremath{\oslash} 5$, $1 \ensuremath{\oslash} 5 \ensuremath{\oplus} 1$ computed using  half-precision arithmetic ($F_{16} := F_{15,5,10}$) (using default rounding)?

\textbf{SOLUTION}

For the first part we use Geometric series:
\meeq{
 2^{-3} (1.1001100110\magenta{011\ensuremath{\ldots}})_2 =  2^{-3} \left(\ensuremath{\sum}_{k=0}^\ensuremath{\infty} {1 \over 2^{4k}} +
{1 \over 2} \ensuremath{\sum}_{k=0}^\ensuremath{\infty} {1 \over 2^{4k}}\right) \ccr
= {3 \over 2^4} {1 \over 1-1/2^4} = {3 \over 2^4-1} = {1 \over 5}
}
Write $-3 = 12 - 15$ hence we have $q = 12 = (01100)_2$. Since $1/5$ is below the midpoint (the midpoint would have been the first magenta bit was 1 and all other bits are 0) we round down and hence have the bits:
\[
\red{0}\ \green{01100}\ \blue{1001100110}
\]
Adding $1$ we get:
\[
1 + 2^{-3} * (1.1001100110)_2 = (1.0011001100\magenta{11})_2 \ensuremath{\approx} (1.0011001101)_2
\]
Here we write the exponent as $0 = 15 - 15$ where $q = 15 = (01111)_2$. Thus we have the bits:
\[
\red{0}\ \green{01111}\ \blue{0011001101}
\]
\textbf{END}

\rule{\textwidth}{1pt}
\textbf{Problem 6} Prove the following bounds on the \emph{absolute error} of a floating point calculation in idealised floating-point arithmetic $F_{\ensuremath{\infty},S}$ (i.e., you may assume all operations involve normal floating point numbers):
\begin{align*}
({\rm fl}(1.1) \ensuremath{\otimes} {\rm fl}(1.2)) &\ensuremath{\oplus} {\rm fl}(1.3) = 2.62 + \ensuremath{\varepsilon}_1 \\
({\rm fl}(1.1) \ensuremath{\ominus} 1) & \ensuremath{\oslash} {\rm fl}(0.1) = 1 + \ensuremath{\varepsilon}_2
\end{align*}
such that $|\ensuremath{\varepsilon}_1| \ensuremath{\leq} 11 \ensuremath{\epsilon}_{\rm m}$ and $|\ensuremath{\varepsilon}_2| \ensuremath{\leq} 40 \ensuremath{\epsilon}_{\rm m}$, where $\ensuremath{\epsilon}_{\rm m}$ is machine epsilon.

\textbf{SOLUTION}

The first problem is very similar to what we saw in lecture. Write
\[
({\rm fl}(1.1)\ensuremath{\otimes} {\rm fl}(1.2)) \ensuremath{\oplus} {\rm fl}(1.3) = ( 1.1(1 + \ensuremath{\delta}_1)1.2(1+\ensuremath{\delta}_2)(1+\ensuremath{\delta}_3) + 1.3(1+\ensuremath{\delta}_4))(1+\ensuremath{\delta}_5)
\]
where we have $|\ensuremath{\delta}_1|,\ensuremath{\ldots},|\ensuremath{\delta}_5| \ensuremath{\leq} \ensuremath{\epsilon}_{\rm m}/2$. We first write
\[
1.1(1 + \ensuremath{\delta}_1)1.2(1+\ensuremath{\delta}_2)(1+\ensuremath{\delta}_3) = 1.32( 1+ \ensuremath{\varepsilon}_1)
\]
where, using the bounds:
\[
|\ensuremath{\delta}_1\ensuremath{\delta}_2|,|\ensuremath{\delta}_1\ensuremath{\delta}_3|,|\ensuremath{\delta}_2\ensuremath{\delta}_3| \ensuremath{\leq} \ensuremath{\epsilon}_{\rm m}/4, |\ensuremath{\delta}_1\ensuremath{\delta}_2\ensuremath{\delta}_3| \ensuremath{\leq} \ensuremath{\epsilon}_{\rm m}/8
\]
we find that
\[
|\ensuremath{\varepsilon}_1| \ensuremath{\leq} |\ensuremath{\delta}_1| + |\ensuremath{\delta}_2| + |\ensuremath{\delta}_3| + |\ensuremath{\delta}_1\ensuremath{\delta}_2| + |\ensuremath{\delta}_1\ensuremath{\delta}_3| + |\ensuremath{\delta}_2\ensuremath{\delta}_3|+ |\ensuremath{\delta}_1\ensuremath{\delta}_2\ensuremath{\delta}_3|
     \ensuremath{\leq} (3/2 + 3/4 + 1/8) \ensuremath{\leq} 5/2 \ensuremath{\epsilon}_{\rm m}
\]
Then we have
\[
1.32 (1 + \ensuremath{\varepsilon}_1) + 1.3 (1 + \ensuremath{\delta}_4) = 2.62 + \underbrace{1.32 \ensuremath{\varepsilon}_1 + 1.3\ensuremath{\delta}_4}_{\ensuremath{\varepsilon}_2}
\]
where
\[
|\ensuremath{\varepsilon}_2| \ensuremath{\leq} (15/4 + 3/4) \ensuremath{\epsilon}_{\rm m} \ensuremath{\leq} 5\ensuremath{\epsilon}_{\rm m}.
\]
Finally,
\[
(2.62 + \ensuremath{\varepsilon}_2)(1+\ensuremath{\delta}_5) = 2.62 + \underbrace{\ensuremath{\varepsilon}_2 + 2.62\ensuremath{\delta}_5 + \ensuremath{\varepsilon}_2 \ensuremath{\delta}_5}_{\ensuremath{\varepsilon}_3}
\]
where, using $|\ensuremath{\varepsilon}_2 \ensuremath{\delta}_5| \ensuremath{\leq} 3\ensuremath{\epsilon}_{\rm m}$ we get,
\[
|\ensuremath{\varepsilon}_3| \ensuremath{\leq} (5 + 3/2 + 3) \ensuremath{\epsilon}_{\rm m}  \ensuremath{\leq} 10\ensuremath{\epsilon}_{\rm m}.
\]
For the second part, we do:
\[
({\rm fl}(1.1) \ensuremath{\ominus} 1) \ensuremath{\oslash} {\rm fl}(0.1) = {(1.1 (1 + \ensuremath{\delta}_1) - 1)(1 + \ensuremath{\delta}_2) \over 0.1 (1 + \ensuremath{\delta}_3)} (1 + \ensuremath{\delta}_4)
\]
where we have $|\ensuremath{\delta}_1|,\ensuremath{\ldots},|\ensuremath{\delta}_4| \ensuremath{\leq} \ensuremath{\epsilon}_{\rm m}/2$. Write
\[
{1 \over 1 + \ensuremath{\delta}_3} = 1 + \ensuremath{\varepsilon}_1
\]
where, using that $|\ensuremath{\delta}_3| \ensuremath{\leq} \ensuremath{\epsilon}_{\rm m}/2 \ensuremath{\leq} 1/2$, we have
\[
|\ensuremath{\varepsilon}_1| \ensuremath{\leq} \left| {\ensuremath{\delta}_3 \over 1 + \ensuremath{\delta}_3} \right| \ensuremath{\leq}  {\ensuremath{\epsilon}_{\rm m} \over 2} {1 \over 1 - 1/2} \ensuremath{\leq} \ensuremath{\epsilon}_{\rm m}.
\]
Further write
\[
(1 + \ensuremath{\varepsilon}_1)(1 + \ensuremath{\delta}_4) = 1 + \ensuremath{\varepsilon}_2
\]
where
\[
|\ensuremath{\varepsilon}_2| \ensuremath{\leq} |\ensuremath{\varepsilon}_1| + |\ensuremath{\delta}_4| + |\ensuremath{\varepsilon}_1| |\ensuremath{\delta}_4| \ensuremath{\leq} (1 + 1/2 + 1/2) \ensuremath{\epsilon}_{\rm m} =   2\ensuremath{\epsilon}_{\rm m}.
\]
We also write
\[
{(1.1 (1 + \ensuremath{\delta}_1) - 1)(1 + \ensuremath{\delta}_2) \over 0.1} = 1 + \underbrace{11\ensuremath{\delta}_1 + \ensuremath{\delta}_2 + 11\ensuremath{\delta}_1\ensuremath{\delta}_2}_{\ensuremath{\varepsilon}_3}
\]
where
\[
|\ensuremath{\varepsilon}_3| \ensuremath{\leq} (11/2 + 1/2  + 11/4) \ensuremath{\leq} 9 \ensuremath{\epsilon}_{\rm m}
\]
Then we get
\[
({\rm fl}(1.1) \ensuremath{\ominus} 1) \ensuremath{\oslash} {\rm fl}(0.1) = (1 + \ensuremath{\varepsilon}_3) (1 + \ensuremath{\varepsilon}_2) =  1 + \underbrace{\ensuremath{\varepsilon}_3 + \ensuremath{\varepsilon}_2 + \ensuremath{\varepsilon}_2 \ensuremath{\varepsilon}_3}_{\ensuremath{\varepsilon}_4}
\]
and the error is bounded by:
\[
|\ensuremath{\varepsilon}_4| \ensuremath{\leq} (9 + 2 + 18) \ensuremath{\epsilon}_{\rm m} \ensuremath{\leq} 29 \ensuremath{\epsilon}_{\rm m}.
\]
\textbf{END}

\textbf{Problem 7} Assume that $f^{\rm FP} : F_{\ensuremath{\infty},S} \ensuremath{\rightarrow} F_{\ensuremath{\infty},S}$ satisfies $f^{\rm FP}(x) = f(x) + \ensuremath{\delta}_x$ where $|\ensuremath{\delta}_x| \ensuremath{\leq} c \ensuremath{\epsilon}_{\rm m}$ for all $x \ensuremath{\in} F_{\ensuremath{\infty},S}$. Show that
\[
{f^{\rm FP}(x+h) \ensuremath{\ominus} f^{\rm FP}(x-h) \over  2h} = f'(x) + \ensuremath{\varepsilon}
\]
where the (absolute) error is bounded by
\[
|\ensuremath{\varepsilon}| \ensuremath{\leq} {|f'(x)| \over 2} \ensuremath{\epsilon}_{\rm m} + {M \over 3} h^2 + {2 c \ensuremath{\epsilon}_{\rm m} \over h}.
\]
\textbf{SOLUTION}

In floating point we have
\begin{align*}
{f^{\rm FP}(x + h) \ensuremath{\ominus} f^{\rm FP}(x-h) \over 2h} &= {f(x + h) +  \ensuremath{\delta}_{x+h} - f(x-h) - \ensuremath{\delta}_{x-h} \over 2h} (1 + \ensuremath{\delta}_1) \\
&= {f(x+h) - f(x-h) \over 2h} (1 + \ensuremath{\delta}_1) + {\ensuremath{\delta}_{x+h}- \ensuremath{\delta}_{x-h} \over 2h} (1 + \ensuremath{\delta}_1)
\end{align*}
From PS1 Q4 we get the error term
\[
f'(x) = {f(x + h) - f(x-h) \over 2h} + \ensuremath{\delta}^{\rm T}
\]
where
\[
|\ensuremath{\delta}^{\rm T}| \ensuremath{\leq} Mh^2/6.
\]
Thus
\[
(f^{\rm FP}(x + h) \ensuremath{\ominus} f^{\rm FP}(x-h)) / (2h) =
f'(x) + \underbrace{f'(x) \ensuremath{\delta}_1 - \ensuremath{\delta}^{\rm T} (1 + \ensuremath{\delta}_1) + {\ensuremath{\delta}_{x+h}- \ensuremath{\delta}_{x-h} \over 2h} (1 + \ensuremath{\delta}_1)}_{\ensuremath{\varepsilon}}
\]
where
\[
|\ensuremath{\varepsilon}| \ensuremath{\leq} {|f'(x)| \over 2} \ensuremath{\epsilon}_{\rm m} + {M \over 3} h^2 + {2 c \ensuremath{\epsilon}_{\rm m} \over h}.
\]
\textbf{END}

\rule{\textwidth}{1pt}
\textbf{Problem 8(a)} Suppose $|\ensuremath{\epsilon}_k| \ensuremath{\leq} \ensuremath{\epsilon}$ and $n \ensuremath{\epsilon} < 1$. Show that $\ensuremath{\prod}_{k=1}^n (1+\ensuremath{\epsilon}_k) = 1+\ensuremath{\theta}_n$ for some constant $\ensuremath{\theta}_n$ satisfying
\[
|\ensuremath{\theta}_n| \ensuremath{\leq} \underbrace{n \ensuremath{\epsilon} \over 1-n\ensuremath{\epsilon}}_{E_{n,\ensuremath{\epsilon}}}.
\]
\textbf{SOLUTION}
\[
\ensuremath{\prod}_{k=1}^{n+1} (1+\ensuremath{\epsilon}_k) = \ensuremath{\prod}_{k=1}^n (1+\ensuremath{\epsilon}_k) (1+\ensuremath{\epsilon}_{n+1}) = (1+\ensuremath{\theta}_n)(1+\ensuremath{\epsilon}_{n+1}) = 1 + \underbrace{\ensuremath{\theta}_n + \ensuremath{\epsilon}_{n+1} + \ensuremath{\theta}_n\ensuremath{\epsilon}_{n+1}}_{\ensuremath{\theta}_{n+1}}
\]
where
\begin{align*}
|\ensuremath{\theta}_{n+1}| &\ensuremath{\leq} {n \ensuremath{\epsilon} \over 1-n\ensuremath{\epsilon}}(1+\ensuremath{\epsilon}) + \ensuremath{\epsilon} \\
&={n \ensuremath{\epsilon} + n \ensuremath{\epsilon}^2 \over 1-(n+1)\ensuremath{\epsilon}} \underbrace{{1-(n+1)\ensuremath{\epsilon} \over 1-n\ensuremath{\epsilon}}}_{\ensuremath{\leq} 1}  + {\ensuremath{\epsilon}-(n+1)\ensuremath{\epsilon}^2 \over 1-(n+1)\ensuremath{\epsilon}} \\
&\ensuremath{\leq} {(n+1)  -  \ensuremath{\epsilon} \over 1-(n+1)\ensuremath{\epsilon}} \ensuremath{\epsilon} \ensuremath{\leq} {(n+1) \ensuremath{\epsilon} \over 1-(n+1)\ensuremath{\epsilon}} = E_{n+1,\ensuremath{\epsilon}}.
\end{align*}
\textbf{END}

\textbf{Problem 8(b)} Show if $x_1,\ensuremath{\ldots},x_n \ensuremath{\in} F_{\ensuremath{\infty},S}$ then
\[
x_1 \ensuremath{\otimes} \ensuremath{\cdots} \ensuremath{\otimes} x_n = x_1 \ensuremath{\cdots} x_n (1 + \ensuremath{\theta}_{n-1})
\]
where $|\ensuremath{\theta}_n| \ensuremath{\leq} E_{n,\ensuremath{\epsilon}_{\rm m}/2}$, assuming $n \ensuremath{\epsilon}_{\rm m} < 2$.

\textbf{SOLUTION}

We can expand out:
\[
x_1 \ensuremath{\otimes} \ensuremath{\cdots} \ensuremath{\otimes} x_n = (\ensuremath{\cdots}((x_1 x_2)(1+\ensuremath{\delta}_1) x_3(1+\ensuremath{\delta}_2)\ensuremath{\cdots} x_n(1+\ensuremath{\delta}_{n-1})) = x_1 \ensuremath{\cdots} x_n (1+\ensuremath{\delta}_1) \ensuremath{\cdots} (1+\ensuremath{\delta}_{n-1})
\]
where $|\ensuremath{\delta}_k| \ensuremath{\leq} \ensuremath{\epsilon}_{\rm m}/2$. The result then follows from the previous result.

\textbf{END}

\textbf{Problem 8(c)} Show if $x_1,\ensuremath{\ldots},x_n \ensuremath{\in} F_{\ensuremath{\infty},S}$ then
\[
x_1 \ensuremath{\oplus} \ensuremath{\cdots} \ensuremath{\oplus} x_n = x_1 +  \ensuremath{\cdots} + x_n + \ensuremath{\sigma}_n
\]
where, for $M = \ensuremath{\Sigma}_{k=1}^n |x_k|$, $|\ensuremath{\sigma}_n| \ensuremath{\leq} M E_{n-1,\ensuremath{\epsilon}_{\rm m}/2},$ assuming $n \ensuremath{\epsilon}_{\rm m} < 2$.

\textbf{SOLUTION}

Using Problem 8(a) we write:
\meeq{
(\ensuremath{\cdots}((x_1 + x_2)(1+\ensuremath{\delta}_1) + x_3)(1+\ensuremath{\delta}_2)\ensuremath{\cdots}+x_n)(1+\ensuremath{\delta}_{n-1})
= x_1 \ensuremath{\prod}_{k=1}^{n-1} (1+\ensuremath{\delta}_k) +  \ensuremath{\sum}_{j=2}^n x_j \ensuremath{\prod}_{k=j-1}^{n-1} (1+\ensuremath{\delta}_j) \ccr
= x_1(1+\ensuremath{\theta}_{n-1}) + \ensuremath{\sum}_{j=2}^n x_j (1 + \ensuremath{\theta}_{n-j+1})
}
where we have for $j = 2,\ensuremath{\ldots},n$
\[
|\ensuremath{\theta}_{n-j+1}| \ensuremath{\leq} E_{n-j+1,\ensuremath{\epsilon}_{\rm m}/2} \ensuremath{\leq} E_{n-1,\ensuremath{\epsilon}_{\rm m}/2}.
\]
Thus we have 
\[
\ensuremath{\sum}_{j=1}^n x_j (1 + \ensuremath{\theta}_{n-j+1})= \ensuremath{\sum}_{j=1}^n x_j + \underbrace{\ensuremath{\sum}_{j=1}^n x_j \ensuremath{\theta}_{n-j+1}}_{\ensuremath{\sigma}_n}
\]
where
\[
|\ensuremath{\sigma}_n| \ensuremath{\leq}  \ensuremath{\sum}_{j=1}^n |x_j \ensuremath{\theta}_{n-j+1}| \ensuremath{\leq} \sup_j |\ensuremath{\theta}_{n-j+1}| \ensuremath{\sum}_{j=1}^n |x_j|  \ensuremath{\leq} \| \ensuremath{\bm{\x}}\|_1 E_{n-1,\ensuremath{\epsilon}_{\rm m}/2}.
\]
\textbf{END}



\end{document}