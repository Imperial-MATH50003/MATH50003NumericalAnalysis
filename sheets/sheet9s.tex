\documentclass[12pt,a4paper]{article}

\usepackage[a4paper,text={16.5cm,25.2cm},centering]{geometry}
\usepackage{lmodern}
\usepackage{amssymb,amsmath}
\usepackage{bm}
\usepackage{graphicx}
\usepackage{microtype}
\usepackage{hyperref}
\usepackage[usenames,dvipsnames]{xcolor}
\setlength{\parindent}{0pt}
\setlength{\parskip}{1.2ex}




\hypersetup
       {   pdfauthor = {  },
           pdftitle={  },
           colorlinks=TRUE,
           linkcolor=black,
           citecolor=blue,
           urlcolor=blue
       }




\usepackage{upquote}
\usepackage{listings}
\usepackage{xcolor}
\lstset{
    basicstyle=\ttfamily\footnotesize,
    upquote=true,
    breaklines=true,
    breakindent=0pt,
    keepspaces=true,
    showspaces=false,
    columns=fullflexible,
    showtabs=false,
    showstringspaces=false,
    escapeinside={(*@}{@*)},
    extendedchars=true,
}
\newcommand{\HLJLt}[1]{#1}
\newcommand{\HLJLw}[1]{#1}
\newcommand{\HLJLe}[1]{#1}
\newcommand{\HLJLeB}[1]{#1}
\newcommand{\HLJLo}[1]{#1}
\newcommand{\HLJLk}[1]{\textcolor[RGB]{148,91,176}{\textbf{#1}}}
\newcommand{\HLJLkc}[1]{\textcolor[RGB]{59,151,46}{\textit{#1}}}
\newcommand{\HLJLkd}[1]{\textcolor[RGB]{214,102,97}{\textit{#1}}}
\newcommand{\HLJLkn}[1]{\textcolor[RGB]{148,91,176}{\textbf{#1}}}
\newcommand{\HLJLkp}[1]{\textcolor[RGB]{148,91,176}{\textbf{#1}}}
\newcommand{\HLJLkr}[1]{\textcolor[RGB]{148,91,176}{\textbf{#1}}}
\newcommand{\HLJLkt}[1]{\textcolor[RGB]{148,91,176}{\textbf{#1}}}
\newcommand{\HLJLn}[1]{#1}
\newcommand{\HLJLna}[1]{#1}
\newcommand{\HLJLnb}[1]{#1}
\newcommand{\HLJLnbp}[1]{#1}
\newcommand{\HLJLnc}[1]{#1}
\newcommand{\HLJLncB}[1]{#1}
\newcommand{\HLJLnd}[1]{\textcolor[RGB]{214,102,97}{#1}}
\newcommand{\HLJLne}[1]{#1}
\newcommand{\HLJLneB}[1]{#1}
\newcommand{\HLJLnf}[1]{\textcolor[RGB]{66,102,213}{#1}}
\newcommand{\HLJLnfm}[1]{\textcolor[RGB]{66,102,213}{#1}}
\newcommand{\HLJLnp}[1]{#1}
\newcommand{\HLJLnl}[1]{#1}
\newcommand{\HLJLnn}[1]{#1}
\newcommand{\HLJLno}[1]{#1}
\newcommand{\HLJLnt}[1]{#1}
\newcommand{\HLJLnv}[1]{#1}
\newcommand{\HLJLnvc}[1]{#1}
\newcommand{\HLJLnvg}[1]{#1}
\newcommand{\HLJLnvi}[1]{#1}
\newcommand{\HLJLnvm}[1]{#1}
\newcommand{\HLJLl}[1]{#1}
\newcommand{\HLJLld}[1]{\textcolor[RGB]{148,91,176}{\textit{#1}}}
\newcommand{\HLJLs}[1]{\textcolor[RGB]{201,61,57}{#1}}
\newcommand{\HLJLsa}[1]{\textcolor[RGB]{201,61,57}{#1}}
\newcommand{\HLJLsb}[1]{\textcolor[RGB]{201,61,57}{#1}}
\newcommand{\HLJLsc}[1]{\textcolor[RGB]{201,61,57}{#1}}
\newcommand{\HLJLsd}[1]{\textcolor[RGB]{201,61,57}{#1}}
\newcommand{\HLJLsdB}[1]{\textcolor[RGB]{201,61,57}{#1}}
\newcommand{\HLJLsdC}[1]{\textcolor[RGB]{201,61,57}{#1}}
\newcommand{\HLJLse}[1]{\textcolor[RGB]{59,151,46}{#1}}
\newcommand{\HLJLsh}[1]{\textcolor[RGB]{201,61,57}{#1}}
\newcommand{\HLJLsi}[1]{#1}
\newcommand{\HLJLso}[1]{\textcolor[RGB]{201,61,57}{#1}}
\newcommand{\HLJLsr}[1]{\textcolor[RGB]{201,61,57}{#1}}
\newcommand{\HLJLss}[1]{\textcolor[RGB]{201,61,57}{#1}}
\newcommand{\HLJLssB}[1]{\textcolor[RGB]{201,61,57}{#1}}
\newcommand{\HLJLnB}[1]{\textcolor[RGB]{59,151,46}{#1}}
\newcommand{\HLJLnbB}[1]{\textcolor[RGB]{59,151,46}{#1}}
\newcommand{\HLJLnfB}[1]{\textcolor[RGB]{59,151,46}{#1}}
\newcommand{\HLJLnh}[1]{\textcolor[RGB]{59,151,46}{#1}}
\newcommand{\HLJLni}[1]{\textcolor[RGB]{59,151,46}{#1}}
\newcommand{\HLJLnil}[1]{\textcolor[RGB]{59,151,46}{#1}}
\newcommand{\HLJLnoB}[1]{\textcolor[RGB]{59,151,46}{#1}}
\newcommand{\HLJLoB}[1]{\textcolor[RGB]{102,102,102}{\textbf{#1}}}
\newcommand{\HLJLow}[1]{\textcolor[RGB]{102,102,102}{\textbf{#1}}}
\newcommand{\HLJLp}[1]{#1}
\newcommand{\HLJLc}[1]{\textcolor[RGB]{153,153,119}{\textit{#1}}}
\newcommand{\HLJLch}[1]{\textcolor[RGB]{153,153,119}{\textit{#1}}}
\newcommand{\HLJLcm}[1]{\textcolor[RGB]{153,153,119}{\textit{#1}}}
\newcommand{\HLJLcp}[1]{\textcolor[RGB]{153,153,119}{\textit{#1}}}
\newcommand{\HLJLcpB}[1]{\textcolor[RGB]{153,153,119}{\textit{#1}}}
\newcommand{\HLJLcs}[1]{\textcolor[RGB]{153,153,119}{\textit{#1}}}
\newcommand{\HLJLcsB}[1]{\textcolor[RGB]{153,153,119}{\textit{#1}}}
\newcommand{\HLJLg}[1]{#1}
\newcommand{\HLJLgd}[1]{#1}
\newcommand{\HLJLge}[1]{#1}
\newcommand{\HLJLgeB}[1]{#1}
\newcommand{\HLJLgh}[1]{#1}
\newcommand{\HLJLgi}[1]{#1}
\newcommand{\HLJLgo}[1]{#1}
\newcommand{\HLJLgp}[1]{#1}
\newcommand{\HLJLgs}[1]{#1}
\newcommand{\HLJLgsB}[1]{#1}
\newcommand{\HLJLgt}[1]{#1}


\def\endash{–}
\def\bbD{ {\mathbb D} }
\def\bbZ{ {\mathbb Z} }
\def\bbR{ {\mathbb R} }
\def\bbC{ {\mathbb C} }

\def\x{ {\vc x} }
\def\a{ {\vc a} }
\def\b{ {\vc b} }
\def\e{ {\vc e} }
\def\f{ {\vc f} }
\def\u{ {\vc u} }
\def\v{ {\vc v} }
\def\y{ {\vc y} }
\def\z{ {\vc z} }
\def\w{ {\vc w} }

\def\bt{ {\tilde b} }
\def\ct{ {\tilde c} }
\def\Ut{ {\tilde U} }
\def\Qt{ {\tilde Q} }
\def\Rt{ {\tilde R} }
\def\Xt{ {\tilde X} }
\def\acos{ {\rm acos}\, }

\def\red#1{ {\color{red} #1} }
\def\blue#1{ {\color{blue} #1} }
\def\green#1{ {\color{ForestGreen} #1} }
\def\magenta#1{ {\color{magenta} #1} }


\def\addtab#1={#1\;&=}

\def\meeq#1{\def\ccr{\\\addtab}
%\tabskip=\@centering
 \begin{align*}
 \addtab#1
 \end{align*}
  }  
  
  \def\leqaddtab#1\leq{#1\;&\leq}
  \def\mleeq#1{\def\ccr{\\\addtab}
%\tabskip=\@centering
 \begin{align*}
 \leqaddtab#1
 \end{align*}
  }  


\def\vc#1{\mbox{\boldmath$#1$\unboldmath}}

\def\vcsmall#1{\mbox{\boldmath$\scriptstyle #1$\unboldmath}}

\def\vczero{{\mathbf 0}}


%\def\beginlist{\begin{itemize}}
%
%\def\endlist{\end{itemize}}


\def\pr(#1){\left({#1}\right)}
\def\br[#1]{\left[{#1}\right]}
\def\fbr[#1]{\!\left[{#1}\right]}
\def\set#1{\left\{{#1}\right\}}
\def\ip<#1>{\left\langle{#1}\right\rangle}
\def\iip<#1>{\left\langle\!\langle{#1}\right\rangle\!\rangle}

\def\norm#1{\left\| #1 \right\|}

\def\abs#1{\left|{#1}\right|}
\def\fpr(#1){\!\pr({#1})}

\def\Re{{\rm Re}\,}
\def\Im{{\rm Im}\,}

\def\floor#1{\left\lfloor#1\right\rfloor}
\def\ceil#1{\left\lceil#1\right\rceil}


\def\mapengine#1,#2.{\mapfunction{#1}\ifx\void#2\else\mapengine #2.\fi }

\def\map[#1]{\mapengine #1,\void.}

\def\mapenginesep_#1#2,#3.{\mapfunction{#2}\ifx\void#3\else#1\mapengine #3.\fi }

\def\mapsep_#1[#2]{\mapenginesep_{#1}#2,\void.}


\def\vcbr{\br}


\def\bvect[#1,#2]{
{
\def\dots{\cdots}
\def\mapfunction##1{\ | \  ##1}
\begin{pmatrix}
		 \,#1\map[#2]\,
\end{pmatrix}
}
}

\def\vect[#1]{
{\def\dots{\ldots}
	\vcbr[{#1}]
}}

\def\vectt[#1]{
{\def\dots{\ldots}
	\vect[{#1}]^{\top}
}}

\def\Vectt[#1]{
{
\def\mapfunction##1{##1 \cr} 
\def\dots{\vdots}
	\begin{pmatrix}
		\map[#1]
	\end{pmatrix}
}}



\def\thetaB{\mbox{\boldmath$\theta$}}
\def\zetaB{\mbox{\boldmath$\zeta$}}


\def\newterm#1{{\it #1}\index{#1}}


\def\TT{{\mathbb T}}
\def\C{{\mathbb C}}
\def\R{{\mathbb R}}
\def\II{{\mathbb I}}
\def\F{{\mathcal F}}
\def\E{{\rm e}}
\def\I{{\rm i}}
\def\D{{\rm d}}
\def\dx{\D x}
\def\CC{{\cal C}}
\def\DD{{\cal D}}
\def\U{{\mathbb U}}
\def\A{{\cal A}}
\def\K{{\cal K}}
\def\DTU{{\cal D}_{{\rm T} \rightarrow {\rm U}}}
\def\LL{{\cal L}}
\def\B{{\cal B}}
\def\T{{\cal T}}
\def\W{{\cal W}}


\def\tF_#1{{\tt F}_{#1}}
\def\Fm{\tF_m}
\def\Fab{\tF_{\alpha,\beta}}
\def\FC{\T}
\def\FCpmz{\FC^{\pm {\rm z}}}
\def\FCz{\FC^{\rm z}}

\def\tFC_#1{{\tt T}_{#1}}
\def\FCn{\tFC_n}

\def\rmz{{\rm z}}

\def\chapref#1{Chapter~\ref{Chapter:#1}}
\def\secref#1{Section~\ref{Section:#1}}
\def\exref#1{Exercise~\ref{Exercise:#1}}
\def\lmref#1{Lemma~\ref{Lemma:#1}}
\def\propref#1{Proposition~\ref{Proposition:#1}}
\def\warnref#1{Warning~\ref{Warning:#1}}
\def\thref#1{Theorem~\ref{Theorem:#1}}
\def\defref#1{Definition~\ref{Definition:#1}}
\def\probref#1{Problem~\ref{Problem:#1}}
\def\corref#1{Corollary~\ref{Corollary:#1}}

\def\sgn{{\rm sgn}\,}
\def\Ai{{\rm Ai}\,}
\def\Bi{{\rm Bi}\,}
\def\wind{{\rm wind}\,}
\def\erf{{\rm erf}\,}
\def\erfc{{\rm erfc}\,}
\def\qqquad{\qquad\quad}
\def\qqqquad{\qquad\qquad}


\def\spand{\hbox{ and }}
\def\spodd{\hbox{ odd}}
\def\speven{\hbox{ even}}
\def\qand{\quad\hbox{and}\quad}
\def\qqand{\qquad\hbox{and}\qquad}
\def\qfor{\quad\hbox{for}\quad}
\def\qqfor{\qquad\hbox{for}\qquad}
\def\qas{\quad\hbox{as}\quad}
\def\qqas{\qquad\hbox{as}\qquad}
\def\qor{\quad\hbox{or}\quad}
\def\qqor{\qquad\hbox{or}\qquad}
\def\qqwhere{\qquad\hbox{where}\qquad}



%%% Words

\def\naive{na\"\i ve\xspace}
\def\Jmap{Joukowsky map\xspace}
\def\Mobius{M\"obius\xspace}
\def\Holder{H\"older\xspace}
\def\Mathematica{{\sc Mathematica}\xspace}
\def\apriori{apriori\xspace}
\def\WHf{Weiner--Hopf factorization\xspace}
\def\WHfs{Weiner--Hopf factorizations\xspace}

\def\Jup{J_\uparrow^{-1}}
\def\Jdown{J_\downarrow^{-1}}
\def\Jin{J_+^{-1}}
\def\Jout{J_-^{-1}}



\def\bD{\D\!\!\!^-}

\def\Abstract#1\par{\begin{abstract}#1\end{abstract}}
\def\Keywords#1\par{\begin{keywords}{#1}\end{keywords}}
\def\Section#1#2.{\section{#2}\label{Section:#1} }
\def\Appendix#1#2.{\appendix \section{#2}\label{Section:#1} }

\def\Subsectionl#1#2.{\subsection{#2}\label{subsec:#1}}
\def\Subsection#1.{\subsection{#1}}

\def\Subsubsection#1.{\subsubsection{#1}}


\def\Problem#1#2\par{\begin{problem}\label{Problem:#1} #2\end{problem}}
\def\Theorem#1#2\par{\begin{theorem}\label{Theorem:#1} #2\end{theorem}}
\def\Conjecture#1#2\par{\begin{conjecture}\label{Conjecture:#1} #2\end{conjecture}}
\def\Proposition#1#2\par{\begin{proposition}\label{Proposition:#1} #2\end{proposition}}
\def\Definition#1#2\par{\begin{definition}\label{Definition:#1} #2\end{definition}}
\def\Corollary#1#2\par{\begin{corollary}\label{Corollary:#1} #2\end{corollary}}
\def\Lemma#1#2\par{\begin{lemma}\label{Lemma:#1} #2\end{lemma}}
\def\Example#1#2\par{\begin{example}\label{Example:#1} #2\end{example}}
\def\Remark #1\par{\begin{remark*}#1\end{remark*}}

\def\figref#1{Figure~\ref{fig:#1}}

\def\Figurew[#1]#2#3\par{
\begin{figure}[tb]
\begin{center}{
\includegraphics[width=#2]{Figures/#1}}
\end{center}
\caption{#3}\label{fig:#1} 
\end{figure}
}

\def\Figure[#1]#2\par{
\begin{figure}[tb]
\begin{center}{
\includegraphics{Figures/#1}}
\end{center}
\caption{#2}\label{fig:#1} 
\end{figure}
}

\def\Figurefixed[#1]#2\par{
\Figurew[#1]{0.48 \hsize}{#2}\par
}

\def\Figuretwow#1#2#3#4\par{
\begin{figure}[tb]
\begin{center}{
\includegraphics[width=#3]{Figures/#1}\includegraphics[width=#3]{Figures/#2}}
\end{center}
\caption{#4}\label{fig:#1} 
\end{figure}
}

\def\Figuretwowframed#1#2#3#4\par{
\begin{figure}[tb]
\begin{center}{
\fbox{\includegraphics[width=#3]{Figures/#1}}\fbox{\includegraphics[width=#3]{Figures/#2}}}
\end{center}
\caption{#4}\label{fig:#1} 
\end{figure}
}

\def\Figuretwo[#1,#2]#3\par{
	\Figuretwow{#1}{#2}{0.48 \hsize}
		#3\par	
}

\def\Figuretwoframed[#1,#2]#3\par{
	\Figuretwowframed{#1}{#2}{0.48 \hsize}
		#3\par	
}

\def\Figurethreew#1#2#3#4#5\par{
\begin{figure}[tb]
\begin{center}{
\includegraphics[width=#4]{Figures/#1} \includegraphics[width=#4]{Figures/#2} \includegraphics[width=#4]{Figures/#3}}
\end{center}
\caption{#5}\label{fig:#1} %\prooflabel{#1}
\end{figure}
}

\def\Figurethree#1#2#3#4\par{
	\Figurethreew{#1}{#2}{#3}{0.3 \hsize}
		{#4}\par	
}

\def\Figurematrixfour#1#2#3#4#5\par{
\begin{figure}[tb]
\begin{center}{
\vbox{\hbox{\includegraphics[width= 0.48 \hsize]{Figures/#1} \includegraphics[width= 0.48 \hsize]{Figures/#2}}\hbox{\includegraphics[width= 0.48 \hsize]{Figures/#3}\includegraphics[width= 0.48 \hsize]{Figures/#4}}}}
\end{center}
\caption{#5}\label{fig:#1} %\prooflabel{#1}
\end{figure}
}


\def\questionequals{= \!\!\!\!\!\!{\scriptstyle ? \atop }\,\,\,}

\def\elll#1{\ell^{\lambda,#1}}
\def\elllp{\ell^{\lambda,p}}
\def\elllRp{\ell^{(\lambda,R),p}}


\def\elllRpz_#1{\ell_{#1{\rm z}}^{(\lambda,R),p}}


\def\sopmatrix#1{\begin{pmatrix}#1\end{pmatrix}}

\def\Proof{\begin{proof}}
\def\mqed{\end{proof}}

\gdef\reffilename{\jobname}
\def\References{\bibliography{\reffilename}}

\outer\def\ends{ 
\end{document}
}


\begin{document}



\textbf{Numerical Analysis MATH50003 (2023\ensuremath{\endash}24) Problem Sheet 9}

\textbf{Problem 1} Construct the monic and orthonormal polynomials up to degree $3$ for the weights $\sqrt{1-x^2}$ and $1-x$ on $[-1,1]$. What are the top $3 \ensuremath{\times} 3$ entries of the corresponding Jacobi matrices? Hint: for the first weight, find a recursive formula for $\int_{-1}^1 x^k \sqrt{1-x^2} {\rm d} x$ using a change-of-variables.

\textbf{SOLUTION}

\textbf{Weight 1} ($\sqrt{1-x^2}$)

Following the hint, we first calculate $\int_{-1}^1 x^k\sqrt{1-x^2}{\rm d}x$. By symmetry, it's zero when $k$ is odd and double the integral on $[0,1]$ when $k$ is even.
\[
\underbrace{\int_0^1 x^k\sqrt{1-x^2}{\rm d}x}_{I_k} = _{x=\sin t}\underbrace{\int_0^{\ensuremath{\pi}/2}\sin^k(t)\cos^2(t){\rm d}t}_{I_k}=\underbrace{\int_0^{\ensuremath{\pi}/2}\sin^k t{\rm d}t}_{J_k}-\underbrace{\int_0^{\ensuremath{\pi}/2}\sin^{k+2}t{\rm d}t}_{J_{k+2}}.
\]
Meanwhile,
\[
J_k=-\int_0^{\ensuremath{\pi}/2}\sin^{k-1}t{\rm d}(\cos t)=_{\text{integral by part}}(k-1)I_{k-2}.
\]
Putting the above 2 equations together, we have $I_k=(k-1)I_{k-2}-(k+1)I_k$, so $I_k=\frac{k-1}{k+2}I_{k-2}$. Since $I_0=\ensuremath{\pi}/4$ we have $I_2 = {\ensuremath{\pi} / 16}$ and $I_4 = \ensuremath{\pi}/32$ hence
\meeq{
\int_{-1}^1 \sqrt{1-x^2} {\rm d}x = {\ensuremath{\pi} \over 2}, \int_{-1}^1 x  \sqrt{1-x^2} {\rm d}x = 0, \int_{-1}^1 x^2 \sqrt{1-x^2} {\rm d}x = {\ensuremath{\pi} \over 8}, \ccr
\int_{-1}^1 x^3 \sqrt{1-x^2} {\rm d}x = 0, \int_{-1}^1 x^4 \sqrt{1-x^2} {\rm d}x = {\ensuremath{\pi} \over 16}.
}
Let $p_0(x)=1$, then $\|p_0\|^2=2I_0=\ensuremath{\pi}/2$. We know from the 3-term recurrence that
\[
xp_0(x)=a_0p_0(x)+p_1(x)
\]
where
\[
a_0=\frac{\ensuremath{\langle} p_0,xp_0\ensuremath{\rangle}}{\|p_0\|^2}=0.
\]
Thus $p_1(x)=x$ and $\|p_1\|^2=2I_2=\ensuremath{\pi}/8$. From
\[
xp_1(x)=c_0p_0(x)+a_1p_1(x)+p_2(x)
\]
we have
\meeq{
c_0=\frac{\ensuremath{\langle} p_0,xp_1\ensuremath{\rangle}}{\|p_0\|^2}=2I_2/2I_0=1/4 \ccr
a_1=\frac{\ensuremath{\langle} p_1,xp_1\ensuremath{\rangle}}{\|p_1\|^2}=0 \ccr
p_2(x)=xp_1(x)-c_0-a_1p_1(x)=x^2-1/4 \ccr
\|p_2\|^2=2I_4-I_2+1/8I_0=\ensuremath{\pi}/32
}
Finally, from
\[
xp_2(x)=c_1p_1(x)+a_2p_2(x)+p_3(x)
\]
we have
\meeq{
c_1=\frac{\ensuremath{\langle} p_1,xp_2\ensuremath{\rangle}}{\|p_1\|^2}=(2I_4-1/2I_2)/(\ensuremath{\pi}/8)=1/4\ccr
a_2=\frac{\ensuremath{\langle} p_2,xp_2\ensuremath{\rangle}}{\|p_2\|^2}=0\ccr
p_3(x)=xp_2(x)-c_1p_1(x)-a_2p_2(x)=x^3-x/2
}
We need one more constant:  from
\[
x p_3(x) p_2(x) = (x^4-x^2/2)(x^2-1/4) = x^6 - 3x^4/4 - x^2/8
\]
we find (since $I_6 = 5I_4/8 =5\ensuremath{\pi}/256$)
\[
c_2 = \frac{\ensuremath{\langle} p_2,xp_3\ensuremath{\rangle}}{\|p_2\|^2}=2{I_6-3I_4/4 + I_2/8 \over \ensuremath{\pi}/32} =1/4
\]
We see from this that
\[
x [p_0(x),p_1(x),\ensuremath{\ldots}] = [p_0(x),p_1(x),\ensuremath{\ldots}] \begin{bmatrix} 0 & 1/4 \\
                                                        1 & 0 & 1/4 \\
                                                          & 1 & 0  & 1/4 \\
                                                          && 1 & 0  & \ensuremath{\ddots} \\
                                                          &&& \ensuremath{\ddots} & \ensuremath{\ddots}
                                                          \end{bmatrix}
\]
To make this symmetric we choose
\meeq{
k_0 = \|p_0\|^{-1} = \sqrt{2/\ensuremath{\pi}} \ccr
k_1 = k_0 \sqrt{4} = 2\sqrt{2/\ensuremath{\pi}} \ccr
k_2 = k_1 \sqrt{4} = 4\sqrt{2/\ensuremath{\pi}} \ccr
k_3 = k_2 \sqrt{4} = 8\sqrt{2/\ensuremath{\pi}}
}
Giving us (also computable from norms of $p_n(x)$):
\meeq{
q_0(x) = \sqrt{2/\ensuremath{\pi}} \ccr
q_1(x) = 2\sqrt{2/\ensuremath{\pi}} x \ccr
q_2(x) = 4 \sqrt{2/\ensuremath{\pi}} (x^2-1/4) \ccr
q_3(x) = 8\sqrt{2/\ensuremath{\pi}}(x^3-x/2)
}
with Jacobi matrix
\[
J = \begin{bmatrix} 0 & 1/2 \\
                    1/2 & 0 & 1/2 \\
                        & 1/2 & 0  & 1/2 \\
                        && \ensuremath{\ddots} & \ensuremath{\ddots}  & \ensuremath{\ddots}
                        \end{bmatrix}
\]
\textbf{Weight 2} ($1-x$)

Here we have polynomials so computing the moments are easier. We have $p_0(x) = 1$ hence
\[
\| p_0\|^2 = \ensuremath{\int}_{-1}^1 (1-x) {\rm d x} = 2
\]
Hence
\meeq{
a_0 = {\ensuremath{\langle}xp_0, p_0 \ensuremath{\rangle} \over \|p_0\|^2} = {\ensuremath{\int}_{-1}^1 x(1-x) {\rm d x}  \over 2} = -{1 \over 3} \ensuremath{\Rightarrow} \ccr
p_1(x) = x p_0(x) - a_0 p_0(x) = x + 1/3 \ensuremath{\Rightarrow} \ccr
\|p_1\|^2 = \ensuremath{\int}_{-1}^1 (-x^3 + x^2/3 + 5x/9 + 1/9) {\rm d x} = {4 \over 9} \ensuremath{\Rightarrow}\ccr
c_0 =  {\ensuremath{\langle}xp_1, p_0 \ensuremath{\rangle} \over \|p_0\|^2} = {2 \over 9}, \ccr
a_1 =  {\ensuremath{\langle}xp_1, p_1 \ensuremath{\rangle} \over \|p_1\|^2} = -{1 \over 15} \ensuremath{\Rightarrow} \ccr
p_2(x) = x p_1(x) - a_1 p_1(x) - c_0 p_0(x) = x^2 + {2x \over 5} - {1 \over 5} \ensuremath{\Rightarrow} \ccr
\|p_2\|^2 = {8 \over 75} \ccr
c_1 =  {\ensuremath{\langle}xp_2, p_1 \ensuremath{\rangle} \over \|p_1\|^2} = {6 \over 25}, \ccr
a_2 =  {\ensuremath{\langle}xp_2, p_2 \ensuremath{\rangle} \over \|p_2\|^2} = -{1 \over 35} \ensuremath{\Rightarrow} \ccr
p_3(x) = x p_2(x) - a_2 p_2(x) - c_1 p_1(x) = x^3 + {3x^2 \over 7} - {3 x \over 7} - {3 \over 35} \ccr
c_2 =  {\ensuremath{\langle}xp_3, p_2 \ensuremath{\rangle} \over \|p_2\|^2} = {12 \over 49}.
}
Thus the multiplication matrix is
\[
x [p_0(x),p_1(x),\ensuremath{\ldots}] = [p_0(x),p_1(x),\ensuremath{\ldots}] \begin{bmatrix} -1/3 & 2/9\\
                                                        1 & -1/15 & 6/25 \\
                                                          & 1 & -1/35  & 12/49 \\
                                                          && 1 & \ensuremath{\ddots}  & \ensuremath{\ddots} \\
                                                          &&& \ensuremath{\ddots} & \ensuremath{\ddots}
                                                          \end{bmatrix}
\]
To make this symmetric we choose
\meeq{
k_0 = \|p_0\|^{-1} = 1/\sqrt{2} \ccr
k_1 = k_0 \sqrt{9/2} = 3/2 \ccr
k_2 = k_1 \sqrt{25/6} = 15/(2\sqrt 6) \ccr
k_3 = k_2 \sqrt{49/12} = 35/(4\sqrt 2)
}
That is
\meeq{
q_0(x) = 1/\sqrt{2} \ccr
q_1(x) = {3 \over 2} (x + 1/3) \ccr
q_2(x) = {15 \over 2\sqrt 6} ( x^2 + {2x \over 5} - {1 \over 5}) \ccr
q_3(x) = {35 \over 4\sqrt 2}( x^3 + {3x^2 \over 7} - {3 x \over 7} - {3 \over 35}).
}
with Jacobi matrix
\[
J = \begin{bmatrix} -1/3 & \sqrt 2 / 3 \\
                    \sqrt 2 / 3  & -1/15 & \sqrt{6}/5 \\
                        & \sqrt{6}/5 &  -1/35   & \ensuremath{\ddots} \\
                        && \ensuremath{\ddots} & \ensuremath{\ddots}
                        \end{bmatrix}
\]
\textbf{END}

\textbf{Problem 2} Prove Theorem 13: a precisely degree $n$ polynomial
\[
p(x) = k_n x^n + O(x^{n-1})
\]
satisfies
\[
\ensuremath{\langle}p,f_m\ensuremath{\rangle} = 0
\]
for all  polynomials $f_m$ of degree $m < n$ of degree less than $n$ if and only if $p(x) = c \ensuremath{\pi}_n$ for some constant $c$, where $\ensuremath{\pi}_n$ are monic orthogonal polynomials.

\textbf{SOLUTION} As $\{\ensuremath{\pi}_0,\ensuremath{\ldots},\ensuremath{\pi}_n\}$ are a basis of all polynomials of degree $n$, we can write
\[
r(x) = \ensuremath{\sum}_{k=0}^m a_k \ensuremath{\pi}_k(x)
\]
Thus if $p(x) = c \ensuremath{\pi}_n(x)$, by linearity of inner products we have
\[
\ensuremath{\langle}p,r\ensuremath{\rangle} = \ensuremath{\langle}c\ensuremath{\pi}_n,\ensuremath{\sum}_{k=0}^m a_k \ensuremath{\pi}_k\ensuremath{\rangle} = \ensuremath{\sum}_{k=0}^m ca_k \ensuremath{\langle}\ensuremath{\pi}_n, \ensuremath{\pi}_k\ensuremath{\rangle} = 0.
\]
Now suppose
\[
p(x) = c x^n + O(x^{n-1})
\]
and consider $p(x) - c \ensuremath{\pi}_n(x)$ which is of degree $n-1$. It satisfies for $k \ensuremath{\leq} n-1$
\[
\ensuremath{\langle}\ensuremath{\pi}_k, p - c \ensuremath{\pi}_n\ensuremath{\rangle} = \ensuremath{\langle}\ensuremath{\pi}_k, p\ensuremath{\rangle} - c \ensuremath{\langle}\ensuremath{\pi}_k, \ensuremath{\pi}_n\ensuremath{\rangle} = 0.
\]
Thus $p - c \ensuremath{\pi}_n$ is zero, i.e., $p(x) = c \ensuremath{\pi}_n(x)$.

\textbf{END}

\textbf{Problem 3} If $w(-x) = w(x)$ for a weight supported on $[-b,b]$ show that $a_n = 0$. Hint: first show that the (monic) polynomials $p_{2n}(x)$ are even and $p_{2n+1}(x)$ are odd.

\textbf{SOLUTION}

An integral is zero if its integrand is odd. Moreover an even function times an odd function is odd and an odd function times an odd fuction is even. Note that $p_0(x)$ and $w(x)$ are even and $x$ is odd.

We see that $a_0$ is zero:
\[
\ensuremath{\langle}p_0,x p_0(x)\ensuremath{\rangle} = \int_{-b}^b x w(x) {\rm d} x = 0
\]
since $xw(x)$ is odd, which shows that
\[
p_1(x) = xp_0(x)
\]
is odd. We now proceed by induction. Assume that $p_{2n}$ is even and $p_{2n-1}$ is odd. We have:
\[
\ensuremath{\langle}p_{2n},x p_{2n}(x)\ensuremath{\rangle} = \int_{-b}^b x w(x) p_{2n}(x)^2 {\rm d} x = 0
\]
since $x w(x) p_{2n}(x)^2$ is odd, therefore $a_{2n} = 0$. Thus from
\[
p_{2n+1}(x) = (x p_{2n}(x)- c_{2n-1} p_{2n-1}(x)) / b_{2n}
\]
we see that $p_{2n+1}$ is odd. Then
\[
\ensuremath{\langle}p_{2n+1},x p_{2n+1}(x)\ensuremath{\rangle} = \int_{-b}^b x w(x) p_{2n+1}(x)^2 {\rm d} x = 0
\]
since $x w(x) p_{2n+1}(x)^2$ is odd, therefore $a_{2n+1} = 0$. and hence
\[
p_{2n+2}(x) = (x p_{2n+1}(x)- c_{2n} p_{2n}(x)) / b_{2n+1}
\]
is even.

\textbf{END}

\textbf{Problem 4(a)} Prove that
\[
U_n(\cos \ensuremath{\theta}) = {\sin(n+1) \ensuremath{\theta} \over \sin \ensuremath{\theta}}.
\]
\textbf{SOLUTION}

We need to verify
\[
p_n(x) := {\sin(n+1) \ensuremath{\theta} \over \sin \ensuremath{\theta}}
\]
are

\begin{itemize}
\item[1. ] graded polynomials


\item[2. ] orthogonal w.r.t. $\sqrt{1-x^2}$ on $[-1,1]$, and


\item[3. ] have the leading coefficient $2^n$.

\end{itemize}
Then uniqueness will guarantee that $p_n(x) = U_n(x)$.

(2) follows under a change of variables
\begin{align*}
\int_{-1}^1 p_n(x)p_m(x)\sqrt{1-x^2}{\rm d}x &= \int_0^{\ensuremath{\pi}}p_n(\cos \ensuremath{\theta})p_m(\cos \ensuremath{\theta})\sin^2 \ensuremath{\theta}{\rm d} \ensuremath{\theta} \\
&= \int_0^{\ensuremath{\pi}}\sin(n+1) \ensuremath{\theta}\sin(m+1) \ensuremath{\theta}{\rm d} \ensuremath{\theta}=\frac{\ensuremath{\pi}}{2}\ensuremath{\delta}_{mn}
\end{align*}
where the last step can be shown by substituting $\sin k\ensuremath{\theta} = (\exp(ik\ensuremath{\theta})-\exp(-ik\ensuremath{\theta})/(2i)$.

To see that they are graded, first note that
\[
p_0(x) = {\sin \ensuremath{\theta} /\sin \ensuremath{\theta}} = 1, p_1(x) =  {\sin 2 \ensuremath{\theta} \over \sin \ensuremath{\theta}} = {2 \sin \ensuremath{\theta} \cos \ensuremath{\theta} \over \sin \ensuremath{\theta}} = 2x.
\]
Now for $n = 1,2,\ensuremath{\ldots}$  use the fact that
\[
xp_n(x)=\frac{\cos \ensuremath{\theta}\sin(n+1) \ensuremath{\theta}}{\sin \ensuremath{\theta}}=\frac{\sin (n+2) \ensuremath{\theta}+\sin n \ensuremath{\theta}}{2\sin \ensuremath{\theta}}
\]
In other words $2xp_n(x)=p_{n+1}(x)+p_{n-1}(x)$, i.e. $p_{n+1}(x) = 2x p_n(x) + p_{n-1}(x).$ By induction it follows that
\[
p_n(x) = 2^n x^n + O(x^{n-1})
\]
which also proves (3).

\textbf{END}

\textbf{Problem 4(b)} Show that
\begin{align*}
x U_0(x) &= U_1(x)/2 \\
x U_n(x) &= {U_{n-1}(x) \over 2} + {U_{n+1}(x) \over 2}.
\end{align*}
\textbf{SOLUTION}

The first result is trivial. For the other parts, from the solution to the previous part we know $2xU_n(x)=U_{n+1}(x)+U_{n-1}(x)$ and this result is a reordering.

\textbf{END}

\textbf{Problem 5}  Use the fact that orthogonal polynomials are uniquely determined by their leading order coefficient and orthogonality to lower dimensional polynomials to show that:
\[
T_n'(x) = n U_{n-1}(x).
\]
\textbf{SOLUTION}

We need to verify that $T_n'(x)$

\begin{itemize}
\item[1. ] graded polynomials


\item[2. ] orthogonal w.r.t. $\sqrt{1-x^2}$ on $[-1,1]$, and


\item[3. ] have the leading coefficient $n 2^n$.

\end{itemize}
(1) and (3) are clear:
\[
T_n'(x) = n 2^{n-1} x^{n-1} + O(x^{n-2}).
\]
(2) For $f_m$ degree $m < n-1$ we have
\[
\ensuremath{\int}_{-1}^1 T_n'(x) f_m(x) \sqrt{1-x^2} {\rm d}x =
-\ensuremath{\int}_{-1}^1 T_n(x) \underbrace{(f_m'(x)(1-x^2) -x f_m)}_{\hbox{degree $m+1 < n$}}  (1-x^2)^{-1/2} {\rm d}x = 0.
\]
\textbf{END}

\textbf{Problem 6(a)} Consider Hermite polynomials orthogonal with respect to the weight $\exp(-x^2)$ on $\ensuremath{\bbR}$ with the normalisation
\[
H_n(x) = 2^n x^n + O(x^{n-1}).
\]
Prove the Rodrigues formula
\[
H_n(x) = (-1)^n \exp(x^2) {{\rm d}^n \over {\rm d}x^n} \exp(-x^2).
\]
Hint: use integration-by-parts.

\textbf{SOLUTION} Define
\[
p_n(x) := (-1)^n \exp(x^2) {{\rm d}^n \over {\rm d}x^n} \exp(-x^2)
\]
We need to verify that $p_n$

\begin{itemize}
\item[1. ] are graded polynomials


\item[2. ] are orthogonal to all lower degree polynomials on $\mathbb{R}$, and


\item[3. ] have the right leading coefficient $2^n$.

\end{itemize}
Comparing the Rodrigues formula for $n$ and $n-1$, we find that
\[
(-1)^n\exp(-x^2)p_n(x)={{\rm d} \over {\rm d}x}\left((-1)^{n-1}\exp(-x^2)p_{n-1}(x)\right)
\]
which reduces to $p_n(x)=2xp_{n-1}(x)-p_{n-1}'(x).$

(1) and (3) then follows from induction since $p_0(x)=1$.

(2) follows by integration by parts. If $r_m$ is any degree $m<n$ polynomial we have:
\begin{align*}
\int_{-\ensuremath{\infty}}^\ensuremath{\infty} p_n(x)r_m(x)\exp(-x^2){\rm d}x&=\int_{-\ensuremath{\infty}}^\ensuremath{\infty}{{\rm d}^n \over {\rm d}x^n} \exp(-x^2)r(x){\rm d}x = -\int_{-\ensuremath{\infty}}^\ensuremath{\infty}{{\rm d}^{n-1} \over {\rm d}x^{n-1}} \exp(-x^2)r'(x){\rm d}x\\
&=\ensuremath{\cdots}\text{integration by parts $n$ times}\ensuremath{\ldots}=(-1)^n\int_{-\ensuremath{\infty}}^\ensuremath{\infty} \exp(-x^2)r_m^{(n)}(x)=0
\end{align*}
Thus $p_n(x) = H_n(x)$ by uniqueness.

\textbf{END}

\textbf{Problem 6(b)} What are $k_n^{(1)}$ and $k_n^{(2)}$ such that
\[
H_n(x) = 2^n x^n + k_n^{(1)} x^{n-1} + k_n^{(2)} x^{n-2} + O(x^{n-3})
\]
\textbf{SOLUTION}

From the previous part we know:
\meeq{
H_{n+1}(x)=2xH_n(x)-H_n'(x) = 2x (2^n x^n + k_n^{(1)} x^{n-1} + k_n^{(2)} x^{n-2} + O(x^{n-3})) -
 (n 2^n x^{n-1} +  O(x^{n-2}) \ccr
  = 2^{n+1} x^{n+1} + 2 k_n^{(1)} x^n + (2k_n^{(2)} - n 2^n) x^{n-1} + O(x^{n-2})
}
hence
\meeq{
k_{n+1}^{(1)}=2k_n^{(1)}, \ccr
k_{n+1}^{(2)}=2k_n^{(2)}-n2^n
}
Since $k_0^{(1)}=0$, we have $k_n^{(1)}=0$ (which also follows by symmetry in the weight). For the second recurrence, lets see the pattern for the first few:
\meeq{
k_0^{(2)} = k_1^{(2)} = 0 \ccr
k_2^{(2)} = -2 \ccr
k_3^{(2)} = 2\ensuremath{\times}(-2) - 2\ensuremath{\times}2^2 = -3\ensuremath{\times}2^2 = -12 \ccr
k_4^{(2)} = 2\ensuremath{\times}(-3\ensuremath{\times}2^2) - 3\ensuremath{\times}2^3 = -6\ensuremath{\times}2^3 = -48 \ccr
k_5^{(2)} = 2\ensuremath{\times}(-6\ensuremath{\times}2^3) - 4\ensuremath{\times}2^4 = -10\ensuremath{\times}2^4 = -160
}
From this the pattern is clear:
\[
k_n^{(2)} = -(\ensuremath{\sum}_{k=1}^{n-1} k) 2^{n-1} = -n(n-1)2^{n-2}.
\]
This can be confirmed by induction:
\[
k_{n+1}^{(2)}=2k_n^{(2)}-n2^n = -n(n-1)2^{n-1} - n2^n = -n(n+1)2^{n-1}.
\]
\textbf{END}

\textbf{Problem 6(c)} Deduce the 3-term recurrence relationship for $H_n(x)$.

\textbf{SOLUTION}

Our goal is to find $a_n$, $b_n$ and $c_n$ such that
\[
xH_n(x)=c_{n-1}H_{n-1}(x)+a_nH_n(x)+b_nH_{n+1}(x).
\]
Matching terms we have $b_n = 1/2$ and $a_n = 0$ so that
\meeq{
c_{n-1} H_{n-1}(x) = xH_n(x) - H_{n+1}(x)/2 = 2^n x^{n+1} + k_n^{(2)}x^{n-1} 
- 2^n x^{n+1} - k_{n+1}^{(2)}/2 x^{n-1} + O(x^{n-2}) \ccr
= (k_n^{(2)} - k_{n+1}^{(2)}/2 ) x^{n-1} + O(x^{n-2}) \ccr 
= (-n(n-1)2^{n-2} +n(n+1)2^{n-2}) x^{n-1} + O(x^{n-2}) \ccr
= n2^{n-1} x^{n-1} + O(x^{n-2}).
}
Therefore we choose
\[
c_{n-1} = {n2^{n-1} \over 2^{n-1}} = n.
\]
\textbf{END}

\textbf{Problem 6(d)} Prove that $H_n'(x) = 2n H_{n-1}(x)$. Hint: show orthogonality of $H_n'$ to all lower degree polynomials, and that the normalisation constants match.

\textbf{SOLUTION}

We have for $f_m$ degree $m < n-1$, using integration by parts
\[
\ensuremath{\langle}H_n', f_m\ensuremath{\rangle} = \ensuremath{\int}_{-\ensuremath{\infty}}^\ensuremath{\infty} H_n'(x) f_m(x) {\rm e}^{-x^2} {\rm d} x =  \ensuremath{\int}_{-\ensuremath{\infty}}^\ensuremath{\infty} H_n(x) \underbrace{(f_m'(x) -2x f_m)}_{\hbox{degree $m+1 < n$}}  {\rm e}^{-x^2} {\rm d} x = 0.
\]
Further,
\[
H_n'(x) = n 2^n x^{n-1} + O(x^{n-1}) = 2n (2^{n-1} x^{n-1} + O(x^{n-1}))
\]
hence the normalisation constants match.

\textbf{END}



\end{document}