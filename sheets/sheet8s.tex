\documentclass[12pt,a4paper]{article}

\usepackage[a4paper,text={16.5cm,25.2cm},centering]{geometry}
\usepackage{lmodern}
\usepackage{amssymb,amsmath}
\usepackage{bm}
\usepackage{graphicx}
\usepackage{microtype}
\usepackage{hyperref}
\usepackage[usenames,dvipsnames]{xcolor}
\setlength{\parindent}{0pt}
\setlength{\parskip}{1.2ex}




\hypersetup
       {   pdfauthor = {  },
           pdftitle={  },
           colorlinks=TRUE,
           linkcolor=black,
           citecolor=blue,
           urlcolor=blue
       }




\usepackage{upquote}
\usepackage{listings}
\usepackage{xcolor}
\lstset{
    basicstyle=\ttfamily\footnotesize,
    upquote=true,
    breaklines=true,
    breakindent=0pt,
    keepspaces=true,
    showspaces=false,
    columns=fullflexible,
    showtabs=false,
    showstringspaces=false,
    escapeinside={(*@}{@*)},
    extendedchars=true,
}
\newcommand{\HLJLt}[1]{#1}
\newcommand{\HLJLw}[1]{#1}
\newcommand{\HLJLe}[1]{#1}
\newcommand{\HLJLeB}[1]{#1}
\newcommand{\HLJLo}[1]{#1}
\newcommand{\HLJLk}[1]{\textcolor[RGB]{148,91,176}{\textbf{#1}}}
\newcommand{\HLJLkc}[1]{\textcolor[RGB]{59,151,46}{\textit{#1}}}
\newcommand{\HLJLkd}[1]{\textcolor[RGB]{214,102,97}{\textit{#1}}}
\newcommand{\HLJLkn}[1]{\textcolor[RGB]{148,91,176}{\textbf{#1}}}
\newcommand{\HLJLkp}[1]{\textcolor[RGB]{148,91,176}{\textbf{#1}}}
\newcommand{\HLJLkr}[1]{\textcolor[RGB]{148,91,176}{\textbf{#1}}}
\newcommand{\HLJLkt}[1]{\textcolor[RGB]{148,91,176}{\textbf{#1}}}
\newcommand{\HLJLn}[1]{#1}
\newcommand{\HLJLna}[1]{#1}
\newcommand{\HLJLnb}[1]{#1}
\newcommand{\HLJLnbp}[1]{#1}
\newcommand{\HLJLnc}[1]{#1}
\newcommand{\HLJLncB}[1]{#1}
\newcommand{\HLJLnd}[1]{\textcolor[RGB]{214,102,97}{#1}}
\newcommand{\HLJLne}[1]{#1}
\newcommand{\HLJLneB}[1]{#1}
\newcommand{\HLJLnf}[1]{\textcolor[RGB]{66,102,213}{#1}}
\newcommand{\HLJLnfm}[1]{\textcolor[RGB]{66,102,213}{#1}}
\newcommand{\HLJLnp}[1]{#1}
\newcommand{\HLJLnl}[1]{#1}
\newcommand{\HLJLnn}[1]{#1}
\newcommand{\HLJLno}[1]{#1}
\newcommand{\HLJLnt}[1]{#1}
\newcommand{\HLJLnv}[1]{#1}
\newcommand{\HLJLnvc}[1]{#1}
\newcommand{\HLJLnvg}[1]{#1}
\newcommand{\HLJLnvi}[1]{#1}
\newcommand{\HLJLnvm}[1]{#1}
\newcommand{\HLJLl}[1]{#1}
\newcommand{\HLJLld}[1]{\textcolor[RGB]{148,91,176}{\textit{#1}}}
\newcommand{\HLJLs}[1]{\textcolor[RGB]{201,61,57}{#1}}
\newcommand{\HLJLsa}[1]{\textcolor[RGB]{201,61,57}{#1}}
\newcommand{\HLJLsb}[1]{\textcolor[RGB]{201,61,57}{#1}}
\newcommand{\HLJLsc}[1]{\textcolor[RGB]{201,61,57}{#1}}
\newcommand{\HLJLsd}[1]{\textcolor[RGB]{201,61,57}{#1}}
\newcommand{\HLJLsdB}[1]{\textcolor[RGB]{201,61,57}{#1}}
\newcommand{\HLJLsdC}[1]{\textcolor[RGB]{201,61,57}{#1}}
\newcommand{\HLJLse}[1]{\textcolor[RGB]{59,151,46}{#1}}
\newcommand{\HLJLsh}[1]{\textcolor[RGB]{201,61,57}{#1}}
\newcommand{\HLJLsi}[1]{#1}
\newcommand{\HLJLso}[1]{\textcolor[RGB]{201,61,57}{#1}}
\newcommand{\HLJLsr}[1]{\textcolor[RGB]{201,61,57}{#1}}
\newcommand{\HLJLss}[1]{\textcolor[RGB]{201,61,57}{#1}}
\newcommand{\HLJLssB}[1]{\textcolor[RGB]{201,61,57}{#1}}
\newcommand{\HLJLnB}[1]{\textcolor[RGB]{59,151,46}{#1}}
\newcommand{\HLJLnbB}[1]{\textcolor[RGB]{59,151,46}{#1}}
\newcommand{\HLJLnfB}[1]{\textcolor[RGB]{59,151,46}{#1}}
\newcommand{\HLJLnh}[1]{\textcolor[RGB]{59,151,46}{#1}}
\newcommand{\HLJLni}[1]{\textcolor[RGB]{59,151,46}{#1}}
\newcommand{\HLJLnil}[1]{\textcolor[RGB]{59,151,46}{#1}}
\newcommand{\HLJLnoB}[1]{\textcolor[RGB]{59,151,46}{#1}}
\newcommand{\HLJLoB}[1]{\textcolor[RGB]{102,102,102}{\textbf{#1}}}
\newcommand{\HLJLow}[1]{\textcolor[RGB]{102,102,102}{\textbf{#1}}}
\newcommand{\HLJLp}[1]{#1}
\newcommand{\HLJLc}[1]{\textcolor[RGB]{153,153,119}{\textit{#1}}}
\newcommand{\HLJLch}[1]{\textcolor[RGB]{153,153,119}{\textit{#1}}}
\newcommand{\HLJLcm}[1]{\textcolor[RGB]{153,153,119}{\textit{#1}}}
\newcommand{\HLJLcp}[1]{\textcolor[RGB]{153,153,119}{\textit{#1}}}
\newcommand{\HLJLcpB}[1]{\textcolor[RGB]{153,153,119}{\textit{#1}}}
\newcommand{\HLJLcs}[1]{\textcolor[RGB]{153,153,119}{\textit{#1}}}
\newcommand{\HLJLcsB}[1]{\textcolor[RGB]{153,153,119}{\textit{#1}}}
\newcommand{\HLJLg}[1]{#1}
\newcommand{\HLJLgd}[1]{#1}
\newcommand{\HLJLge}[1]{#1}
\newcommand{\HLJLgeB}[1]{#1}
\newcommand{\HLJLgh}[1]{#1}
\newcommand{\HLJLgi}[1]{#1}
\newcommand{\HLJLgo}[1]{#1}
\newcommand{\HLJLgp}[1]{#1}
\newcommand{\HLJLgs}[1]{#1}
\newcommand{\HLJLgsB}[1]{#1}
\newcommand{\HLJLgt}[1]{#1}


\def\endash{–}
\def\bbD{ {\mathbb D} }
\def\bbZ{ {\mathbb Z} }
\def\bbR{ {\mathbb R} }
\def\bbC{ {\mathbb C} }

\def\x{ {\vc x} }
\def\a{ {\vc a} }
\def\b{ {\vc b} }
\def\e{ {\vc e} }
\def\f{ {\vc f} }
\def\u{ {\vc u} }
\def\v{ {\vc v} }
\def\y{ {\vc y} }
\def\z{ {\vc z} }
\def\w{ {\vc w} }

\def\bt{ {\tilde b} }
\def\ct{ {\tilde c} }
\def\Ut{ {\tilde U} }
\def\Qt{ {\tilde Q} }
\def\Rt{ {\tilde R} }
\def\Xt{ {\tilde X} }
\def\acos{ {\rm acos}\, }

\def\red#1{ {\color{red} #1} }
\def\blue#1{ {\color{blue} #1} }
\def\green#1{ {\color{ForestGreen} #1} }
\def\magenta#1{ {\color{magenta} #1} }


\def\addtab#1={#1\;&=}

\def\meeq#1{\def\ccr{\\\addtab}
%\tabskip=\@centering
 \begin{align*}
 \addtab#1
 \end{align*}
  }  
  
  \def\leqaddtab#1\leq{#1\;&\leq}
  \def\mleeq#1{\def\ccr{\\\addtab}
%\tabskip=\@centering
 \begin{align*}
 \leqaddtab#1
 \end{align*}
  }  


\def\vc#1{\mbox{\boldmath$#1$\unboldmath}}

\def\vcsmall#1{\mbox{\boldmath$\scriptstyle #1$\unboldmath}}

\def\vczero{{\mathbf 0}}


%\def\beginlist{\begin{itemize}}
%
%\def\endlist{\end{itemize}}


\def\pr(#1){\left({#1}\right)}
\def\br[#1]{\left[{#1}\right]}
\def\fbr[#1]{\!\left[{#1}\right]}
\def\set#1{\left\{{#1}\right\}}
\def\ip<#1>{\left\langle{#1}\right\rangle}
\def\iip<#1>{\left\langle\!\langle{#1}\right\rangle\!\rangle}

\def\norm#1{\left\| #1 \right\|}

\def\abs#1{\left|{#1}\right|}
\def\fpr(#1){\!\pr({#1})}

\def\Re{{\rm Re}\,}
\def\Im{{\rm Im}\,}

\def\floor#1{\left\lfloor#1\right\rfloor}
\def\ceil#1{\left\lceil#1\right\rceil}


\def\mapengine#1,#2.{\mapfunction{#1}\ifx\void#2\else\mapengine #2.\fi }

\def\map[#1]{\mapengine #1,\void.}

\def\mapenginesep_#1#2,#3.{\mapfunction{#2}\ifx\void#3\else#1\mapengine #3.\fi }

\def\mapsep_#1[#2]{\mapenginesep_{#1}#2,\void.}


\def\vcbr{\br}


\def\bvect[#1,#2]{
{
\def\dots{\cdots}
\def\mapfunction##1{\ | \  ##1}
\begin{pmatrix}
		 \,#1\map[#2]\,
\end{pmatrix}
}
}

\def\vect[#1]{
{\def\dots{\ldots}
	\vcbr[{#1}]
}}

\def\vectt[#1]{
{\def\dots{\ldots}
	\vect[{#1}]^{\top}
}}

\def\Vectt[#1]{
{
\def\mapfunction##1{##1 \cr} 
\def\dots{\vdots}
	\begin{pmatrix}
		\map[#1]
	\end{pmatrix}
}}



\def\thetaB{\mbox{\boldmath$\theta$}}
\def\zetaB{\mbox{\boldmath$\zeta$}}


\def\newterm#1{{\it #1}\index{#1}}


\def\TT{{\mathbb T}}
\def\C{{\mathbb C}}
\def\R{{\mathbb R}}
\def\II{{\mathbb I}}
\def\F{{\mathcal F}}
\def\E{{\rm e}}
\def\I{{\rm i}}
\def\D{{\rm d}}
\def\dx{\D x}
\def\CC{{\cal C}}
\def\DD{{\cal D}}
\def\U{{\mathbb U}}
\def\A{{\cal A}}
\def\K{{\cal K}}
\def\DTU{{\cal D}_{{\rm T} \rightarrow {\rm U}}}
\def\LL{{\cal L}}
\def\B{{\cal B}}
\def\T{{\cal T}}
\def\W{{\cal W}}


\def\tF_#1{{\tt F}_{#1}}
\def\Fm{\tF_m}
\def\Fab{\tF_{\alpha,\beta}}
\def\FC{\T}
\def\FCpmz{\FC^{\pm {\rm z}}}
\def\FCz{\FC^{\rm z}}

\def\tFC_#1{{\tt T}_{#1}}
\def\FCn{\tFC_n}

\def\rmz{{\rm z}}

\def\chapref#1{Chapter~\ref{Chapter:#1}}
\def\secref#1{Section~\ref{Section:#1}}
\def\exref#1{Exercise~\ref{Exercise:#1}}
\def\lmref#1{Lemma~\ref{Lemma:#1}}
\def\propref#1{Proposition~\ref{Proposition:#1}}
\def\warnref#1{Warning~\ref{Warning:#1}}
\def\thref#1{Theorem~\ref{Theorem:#1}}
\def\defref#1{Definition~\ref{Definition:#1}}
\def\probref#1{Problem~\ref{Problem:#1}}
\def\corref#1{Corollary~\ref{Corollary:#1}}

\def\sgn{{\rm sgn}\,}
\def\Ai{{\rm Ai}\,}
\def\Bi{{\rm Bi}\,}
\def\wind{{\rm wind}\,}
\def\erf{{\rm erf}\,}
\def\erfc{{\rm erfc}\,}
\def\qqquad{\qquad\quad}
\def\qqqquad{\qquad\qquad}


\def\spand{\hbox{ and }}
\def\spodd{\hbox{ odd}}
\def\speven{\hbox{ even}}
\def\qand{\quad\hbox{and}\quad}
\def\qqand{\qquad\hbox{and}\qquad}
\def\qfor{\quad\hbox{for}\quad}
\def\qqfor{\qquad\hbox{for}\qquad}
\def\qas{\quad\hbox{as}\quad}
\def\qqas{\qquad\hbox{as}\qquad}
\def\qor{\quad\hbox{or}\quad}
\def\qqor{\qquad\hbox{or}\qquad}
\def\qqwhere{\qquad\hbox{where}\qquad}



%%% Words

\def\naive{na\"\i ve\xspace}
\def\Jmap{Joukowsky map\xspace}
\def\Mobius{M\"obius\xspace}
\def\Holder{H\"older\xspace}
\def\Mathematica{{\sc Mathematica}\xspace}
\def\apriori{apriori\xspace}
\def\WHf{Weiner--Hopf factorization\xspace}
\def\WHfs{Weiner--Hopf factorizations\xspace}

\def\Jup{J_\uparrow^{-1}}
\def\Jdown{J_\downarrow^{-1}}
\def\Jin{J_+^{-1}}
\def\Jout{J_-^{-1}}



\def\bD{\D\!\!\!^-}

\def\Abstract#1\par{\begin{abstract}#1\end{abstract}}
\def\Keywords#1\par{\begin{keywords}{#1}\end{keywords}}
\def\Section#1#2.{\section{#2}\label{Section:#1} }
\def\Appendix#1#2.{\appendix \section{#2}\label{Section:#1} }

\def\Subsectionl#1#2.{\subsection{#2}\label{subsec:#1}}
\def\Subsection#1.{\subsection{#1}}

\def\Subsubsection#1.{\subsubsection{#1}}


\def\Problem#1#2\par{\begin{problem}\label{Problem:#1} #2\end{problem}}
\def\Theorem#1#2\par{\begin{theorem}\label{Theorem:#1} #2\end{theorem}}
\def\Conjecture#1#2\par{\begin{conjecture}\label{Conjecture:#1} #2\end{conjecture}}
\def\Proposition#1#2\par{\begin{proposition}\label{Proposition:#1} #2\end{proposition}}
\def\Definition#1#2\par{\begin{definition}\label{Definition:#1} #2\end{definition}}
\def\Corollary#1#2\par{\begin{corollary}\label{Corollary:#1} #2\end{corollary}}
\def\Lemma#1#2\par{\begin{lemma}\label{Lemma:#1} #2\end{lemma}}
\def\Example#1#2\par{\begin{example}\label{Example:#1} #2\end{example}}
\def\Remark #1\par{\begin{remark*}#1\end{remark*}}

\def\figref#1{Figure~\ref{fig:#1}}

\def\Figurew[#1]#2#3\par{
\begin{figure}[tb]
\begin{center}{
\includegraphics[width=#2]{Figures/#1}}
\end{center}
\caption{#3}\label{fig:#1} 
\end{figure}
}

\def\Figure[#1]#2\par{
\begin{figure}[tb]
\begin{center}{
\includegraphics{Figures/#1}}
\end{center}
\caption{#2}\label{fig:#1} 
\end{figure}
}

\def\Figurefixed[#1]#2\par{
\Figurew[#1]{0.48 \hsize}{#2}\par
}

\def\Figuretwow#1#2#3#4\par{
\begin{figure}[tb]
\begin{center}{
\includegraphics[width=#3]{Figures/#1}\includegraphics[width=#3]{Figures/#2}}
\end{center}
\caption{#4}\label{fig:#1} 
\end{figure}
}

\def\Figuretwowframed#1#2#3#4\par{
\begin{figure}[tb]
\begin{center}{
\fbox{\includegraphics[width=#3]{Figures/#1}}\fbox{\includegraphics[width=#3]{Figures/#2}}}
\end{center}
\caption{#4}\label{fig:#1} 
\end{figure}
}

\def\Figuretwo[#1,#2]#3\par{
	\Figuretwow{#1}{#2}{0.48 \hsize}
		#3\par	
}

\def\Figuretwoframed[#1,#2]#3\par{
	\Figuretwowframed{#1}{#2}{0.48 \hsize}
		#3\par	
}

\def\Figurethreew#1#2#3#4#5\par{
\begin{figure}[tb]
\begin{center}{
\includegraphics[width=#4]{Figures/#1} \includegraphics[width=#4]{Figures/#2} \includegraphics[width=#4]{Figures/#3}}
\end{center}
\caption{#5}\label{fig:#1} %\prooflabel{#1}
\end{figure}
}

\def\Figurethree#1#2#3#4\par{
	\Figurethreew{#1}{#2}{#3}{0.3 \hsize}
		{#4}\par	
}

\def\Figurematrixfour#1#2#3#4#5\par{
\begin{figure}[tb]
\begin{center}{
\vbox{\hbox{\includegraphics[width= 0.48 \hsize]{Figures/#1} \includegraphics[width= 0.48 \hsize]{Figures/#2}}\hbox{\includegraphics[width= 0.48 \hsize]{Figures/#3}\includegraphics[width= 0.48 \hsize]{Figures/#4}}}}
\end{center}
\caption{#5}\label{fig:#1} %\prooflabel{#1}
\end{figure}
}


\def\questionequals{= \!\!\!\!\!\!{\scriptstyle ? \atop }\,\,\,}

\def\elll#1{\ell^{\lambda,#1}}
\def\elllp{\ell^{\lambda,p}}
\def\elllRp{\ell^{(\lambda,R),p}}


\def\elllRpz_#1{\ell_{#1{\rm z}}^{(\lambda,R),p}}


\def\sopmatrix#1{\begin{pmatrix}#1\end{pmatrix}}

\def\Proof{\begin{proof}}
\def\mqed{\end{proof}}

\gdef\reffilename{\jobname}
\def\References{\bibliography{\reffilename}}

\outer\def\ends{ 
\end{document}
}


\begin{document}



\textbf{Numerical Analysis MATH50003 (2024\ensuremath{\endash}25) Problem Sheet 8}

\textbf{Problem 1} Give explicit formulae for $\hat f_k$ and $\hat f_k^n$ for the following functions:
\[
\cos \ensuremath{\theta}, \cos 4\ensuremath{\theta}, \sin^4\ensuremath{\theta}, {3 \over 3 - {\rm e}^{\rm i \ensuremath{\theta}}}, {1 \over 1 - 2{\rm e}^{\rm i \ensuremath{\theta}}}
\]
\textbf{SOLUTION}

(1) Just expand in complex exponentials to find that
\[
\cos \ensuremath{\theta} = {\exp({\rm i} \ensuremath{\theta}) + \exp(-{\rm i} \ensuremath{\theta}) \over 2}
\]
that is $\hat f_1 = \hat f_{-1} = 1/2$, $\hat f_k = 0$ otherwise. Therefore for $p\ensuremath{\in} \ensuremath{\bbZ}$ we have
\begin{align*}
\hat f_k^1 &= \hat f_1 + \hat f_{-1} = 1 \\
\hat f_{2p}^2 &= 0, \hat f_{2p+1}^2 = \hat f_1 + \hat f_{-1} = 1 \\
\hat f_{1+np}^n &= \hat f_{-1+np}^n = 1/2, \hat f_k^n = 0
\end{align*}
for $n = 3,4,\ensuremath{\ldots}$.

(2) Similarly
\[
\cos 4 \ensuremath{\theta} = {\exp(4{\rm i} \ensuremath{\theta}) + \exp(-4{\rm i} \ensuremath{\theta}) \over 2}
\]
that is $\hat f_4 = \hat f_{-4} = 1/2$, $\hat f_k = 0$ otherwise. Therefore for $p\ensuremath{\in} \ensuremath{\bbZ}$ we have
\begin{align*}
\hat f_p^1 &= \hat f_4 + \hat f_{-4} = 1 \\
\hat f_{2p}^2 &= \hat f_4 + \hat f_{-4} = 1, \hat f_{2p+1}^2 = 0 \\
\hat f_{3p}^3 &= 0, \hat f_{3p\ensuremath{\pm}1}^3 = \hat f_{\ensuremath{\pm}4} =1/2 \\
\hat f_{4p}^4 &= \hat f_{-4} + \hat f_4 = 1, \hat f_{4p\ensuremath{\pm}1}^4 = 0,  \hat f_{4p+2}^4 = 0 \\
\hat f_{5p}^5 &= 0, \hat f_{5p+1}^5 = \hat f_{-4} = 1/2, \hat f_{5p-1}^5 = \hat f_{4} = 1/2,  \hat f_{5p\ensuremath{\pm}2}^5 =0  \\
\hat f_{6p}^6 &=0, \hat f_{6p\ensuremath{\pm}1}^6 = 0,  \hat f_{6p+2}^6 = \hat f_{-4} = 1/2,  \hat f_{6p-2}^6 = \hat f_{4} = 1/2, \hat f_{6p+3}^6 =0  \\
\hat f_{7p}^7 &= 0, \hat f_{7p\ensuremath{\pm}1}^7 = 0,  \hat f_{7p\ensuremath{\pm}2}^7 = 0, \hat f_{7p\ensuremath{\pm}3}^7 = \hat f_{\ensuremath{\mp}4} = 1/2 \\
\hat f_{8p}^8 &= \hat f_{8p\ensuremath{\pm}1}^8 = \hat f_{8p\ensuremath{\pm}2}^8 =  \hat f_{8p\ensuremath{\pm}3}^8 = 0, \hat f_{8p+4}^8 = \hat f_4 + \hat f_{-4} = 1 \\
\hat f_{k+pn}^n &= \hat f_k \hbox{ for $-4 \ensuremath{\leq} k \ensuremath{\leq} 4$, 0 otherwise}.
\end{align*}
(3) Here we have:
\meeq{
(\sin \ensuremath{\theta})^4= \left({\exp({\rm i} \ensuremath{\theta}) - \exp(-{\rm i} \ensuremath{\theta}) \over 2 {\rm i}}\right)^4
= \left({\exp(2{\rm i} \ensuremath{\theta}) -2 + \exp(-2{\rm i} \ensuremath{\theta}) \over -4}\right)^2 \ccr
= {\exp(4{\rm i} \ensuremath{\theta}) -4\exp(2{\rm i} \ensuremath{\theta}) +6 -4 \exp(-2{\rm i} \ensuremath{\theta})+\exp(-4{\rm i} \ensuremath{\theta}) \over 16}
}
that is $\hat f_{-4} = \hat f_4 = 1/16$, $\hat f_{-2} = \hat f_2 = -1/4$, $f_0 = 3/8$, $\hat f_k = 0$ otherwise. Therefore for $p\ensuremath{\in} \ensuremath{\bbZ}$ we have
\begin{align*}
\hat f_p^1 &=\hat f_{-4} + \hat f_{-2} + \hat f_0 + \hat f_2 +  \hat f_4 = 0 \\
\hat f_k^2 &= 0 \\
\hat f_{3p}^3 &= \hat f_0 = 3/8, \hat f_{3p+1}^3 = \hat f_{-2} + \hat f_4 =-3/16,  \hat f_{3p-1}^3 = \hat f_{2} + \hat f_{-4} =-3/16 \\
\hat f_{4p}^4 &= \hat f_0 + \hat f_{-4} + \hat f_4 = 1/2, \hat f_{4p\ensuremath{\pm}1}^4 = 0,  \hat f_{4p+2}^4 = \hat f_{2} + \hat f_{-2} =-1/2 \\
\hat f_{5p}^5 &= \hat f_0 = 3/8, \hat f_{5p+1}^5 = \hat f_{-4} = 1/16, \hat f_{5p-1}^5 = \hat f_{4} = 1/16,  \hat f_{5p+2}^5 = \hat f_2 = -1/4, \hat f_{5p-2}^5 = \hat f_{-2} = -1/4  \\
\hat f_{6p}^6 &= \hat f_0 = 3/8, \hat f_{6p\ensuremath{\pm}1}^6 = 0,  \hat f_{6p+2}^6 = \hat f_2 + \hat f_{-4} = -3/16,  \hat f_{6p-2}^6 = \hat f_{-2} + \hat f_{4} = -3/16, \hat f_{6p+3}^6 =0  \\
\hat f_{7p}^7 &= \hat f_0 = 3/8, \hat f_{7p\ensuremath{\pm}1}^7 = 0,  \hat f_{7p\ensuremath{\pm}2}^7 = \hat f_{\ensuremath{\pm}2} = -1/4, \hat f_{7p\ensuremath{\pm}3}^7 = \hat f_{\ensuremath{\mp}4} = 1/16 \\
\hat f_{8p}^8 &= \hat f_0 = 3/8, \hat f_{8p\ensuremath{\pm}1}^8 = 0,  \hat f_{8p\ensuremath{\pm}2}^8 = \hat f_{\ensuremath{\pm}2} = -1/4, \hat f_{8p\ensuremath{\pm}3}^8 = 0, \hat f_{8p+4}^8 = \hat f_4 + \hat f_{-4} = 1/8 \\
\hat f_{k+pn}^n &= \hat f_k \hbox{ for $-4 \ensuremath{\leq} k \ensuremath{\leq} 4$, 0 otherwise}.
\end{align*}
(4) Under the change of variables $z = {\rm e}^{{\rm i}\ensuremath{\theta}}$ we can use Geoemtric series to determine
\[
{3 \over 3 - z} = {1 \over 1- z/3} = \ensuremath{\sum}_{k=0}^\ensuremath{\infty} {z^k \over 3^k}
\]
That is $\hat f_k = 1/3^k$ for $k \ensuremath{\geq} 0$, and $\hat f_k = 0$ otherwise. We then have for $0 \ensuremath{\leq} k \ensuremath{\leq} n-1$
\[
\hat f_{k+pn}^n = \ensuremath{\sum}_{\ensuremath{\ell}=0}^\ensuremath{\infty} {1 \over 3^{k+\ensuremath{\ell}n}} = {1 \over 3^k} {1 \over 1 - 1/3^n} = {3^n \over 3^{n+k} - 3^k}
\]
(5) Now make the change of variables $z = {\rm e}^{-{\rm i} \ensuremath{\theta}}$ to get:
\[
{1 \over 1 - 2/z} = {1 \over -2/z} {1 \over 1 - z/2} = {1 \over -2/z} \ensuremath{\sum}_{k=0}^\ensuremath{\infty} {z^k \over 2^k}
= - \ensuremath{\sum}_{k=1}^\ensuremath{\infty} {{\rm e}^{-{\rm i} k \ensuremath{\theta}} \over 2^k}
\]
That is $\hat f_k = -1/2^{-k}$ for $k \ensuremath{\leq} -1$ and 0 otherwise. We then have for $-n \ensuremath{\leq} k \ensuremath{\leq} -1$
\[
\hat f_{k+pn}^n =- \ensuremath{\sum}_{\ensuremath{\ell}=0}^\ensuremath{\infty} {1 \over 2^{-k+\ensuremath{\ell}n}} = -{1 \over 2^{-k}} {1 \over 1 - 1/2^n} = -{2^{n+k} \over 2^n - 1}
\]
\textbf{END}

\textbf{Problem 2} Prove that if the first $\ensuremath{\lambda}-1$ derivatives $f(\ensuremath{\theta}), f'(\ensuremath{\theta}), \ensuremath{\ldots}, f^{(\ensuremath{\lambda}-1)}(\ensuremath{\theta})$ are 2\ensuremath{\pi}-periodic and $f^{(\ensuremath{\lambda})}$ is uniformly bounded  that
\[
|\hat f_k| = O(|k|^{-\ensuremath{\lambda}})\qquad \hbox{as $|k| \ensuremath{\rightarrow} \ensuremath{\infty}$}
\]
Use this to show for the Taylor case ($0 = \hat f_{-1} = \hat f_{-2} = \ensuremath{\cdots}$) that
\[
|f(\ensuremath{\theta}) - \ensuremath{\sum}_{k=0}^{n-1} \hat f_k {\rm e}^{{\rm i}k\ensuremath{\theta}}| = O(n^{1-\ensuremath{\lambda}})\qquad \hbox{as $n \ensuremath{\rightarrow} \ensuremath{\infty}$}
\]
\textbf{SOLUTION} A straightforward application of integration by parts yields the result
\[
\hat f_k = \frac{1}{2\ensuremath{\pi}} \int^{2\ensuremath{\pi}}_{0} f(\ensuremath{\theta}) {\rm e}^{-ik\ensuremath{\theta}} d\ensuremath{\theta} = \frac{(-i)^\ensuremath{\lambda}}{2\ensuremath{\pi} k^{\ensuremath{\lambda}}} \int^{2\ensuremath{\pi}}_{0} f^{(\ensuremath{\lambda})}(\ensuremath{\theta}) {\rm e}^{-ik\ensuremath{\theta}} d\ensuremath{\theta}
\]
given that $f^{(\ensuremath{\lambda})}$ is uniformly bounded, the second part follows directly from this result
\[
|\ensuremath{\sum}_{k=n}^{\ensuremath{\infty}} \hat f_k {\rm e}^{{\rm i}k\ensuremath{\theta}}| \ensuremath{\leq} \ensuremath{\sum}_{k=n}^{\ensuremath{\infty}} |\hat f_k | \ensuremath{\leq} C \ensuremath{\sum}_{k=n}^{\ensuremath{\infty}} k^{-\ensuremath{\lambda}}
\]
for some constant $C$. The result then follows by the dominant convergence test:
\[
\ensuremath{\sum}_{k=n}^{\ensuremath{\infty}} k^{-\ensuremath{\lambda}} \ensuremath{\leq} \int_{n-1}^\ensuremath{\infty} k^{-\ensuremath{\lambda}} {\rm d}k = O(n^{1-\ensuremath{\lambda}}).
\]
\textbf{END}

\textbf{Problem 3(a)} If $f$ is a trigonometric polynomial  ($\hat f_k = 0$ for $|k| > m$) show for $n \ensuremath{\geq} 2m+1$ that we can exactly recover $f$:
\[
f(\ensuremath{\theta}) = \ensuremath{\sum}_{k=-m}^m \hat f_k^n {\rm e}^{{\rm i} k \ensuremath{\theta}}
\]
\textbf{SOLUTION} This follows from
\[
\hat f_k^n = \ensuremath{\sum}_{p=-\ensuremath{\infty}}^\ensuremath{\infty} \hat f_{k+pn} = \hat f_k
\]
if $-m \ensuremath{\leq} k \ensuremath{\leq} m$ as if $p > 0$ we have $k + p n \ensuremath{\geq} k + 2m+1 \ensuremath{\geq} m+1$ hence $\hat f_{k+pn} = 0$ and if $k < 0$ we have $k - pn \ensuremath{\leq} k -2m-1 \ensuremath{\leq} -m-1$ hence $\hat f_{k+pn} = 0$.

\textbf{END}

\textbf{Problem 3(b)} For the general (non-Taylor) case and $n = 2m+1$, prove convergence for
\[
f_{-m:m}(\ensuremath{\theta}) := \ensuremath{\sum}_{k=-m}^m \hat f_k^n {\rm e}^{{\rm i} k \ensuremath{\theta}}
\]
to $f(\ensuremath{\theta})$ as $n \ensuremath{\rightarrow} \ensuremath{\infty}$. What is the rate of convergence if we know that the first $\ensuremath{\lambda}-1$ derivatives $f(\ensuremath{\theta}), f'(\ensuremath{\theta}), \ensuremath{\ldots}, f^{(\ensuremath{\lambda}-1)}(\ensuremath{\theta})$ are 2\ensuremath{\pi}-periodic and $f^{(\ensuremath{\lambda})}$ is uniformly bounded?

\textbf{SOLUTION}

Observe that by aliasing (see corollary in lecture notes) and triangle inequality we have the following
\[
|\hat f_k^n - \hat f_k|  \ensuremath{\leq} \ensuremath{\sum}_{p=1}^{\ensuremath{\infty}} (|\hat f_{k+pn}|+|\hat f_{k-pn}|)
\]
Using the result from Problem 2 yields
\[
|\hat f_k^n - \hat f_k| \ensuremath{\leq} \frac{C}{n^\ensuremath{\lambda}} \ensuremath{\sum}_{p=1}^{\ensuremath{\infty}} \frac{1}{\left(p + \frac{k}{n}\right)^\ensuremath{\lambda}} + \frac{1}{\left(p - \frac{k}{n}\right)^\ensuremath{\lambda}}
\]
now we pick $|q| < \frac{1}{2}$ (such that the estimate below will hold for both summands above) and construct an integral with convex and monotonocally decreasing integrand such that
\[
\left( p + q \right)^{-\ensuremath{\lambda}} < \int_{p-\frac{1}{2}}^{p+\frac{1}{2}} (x + q)^{-\ensuremath{\lambda}} dx
\]
more over summing over the left-hand side from $1$ to $\ensuremath{\infty}$ yields a bound by the integral:
\[
\int^{\ensuremath{\infty}}_{\frac{1}{2}} (x + q)^{-\ensuremath{\lambda}} dx = \frac{1}{\ensuremath{\lambda}}(\frac{1}{2} + q)^{- \ensuremath{\lambda} + 1}
\]
Finally let $q = \pm \frac{k}{n}$ to achieve the rate of convergence
\[
|\hat f_k^n - \hat f_k| \ensuremath{\leq} \frac{C_{\ensuremath{\lambda}}}{ n^{\ensuremath{\lambda}}} \left(  \left( \frac{1}{2} + k/n \right)^{ - \ensuremath{\lambda} + 1} + \left( \left( \frac{1}{2} - k/n \right) \right)^{- \ensuremath{\lambda} +1} \right)
\]
where $C_{\ensuremath{\lambda}}$ is a constant depending on $\ensuremath{\lambda}$. Note that it is indeed important to split the $n$ coefficients equally over the negative and positive coefficients as stated in the notes, due to the estatime we used above.

Finally, we have:
\begin{align*}
|f(\ensuremath{\theta}) - f_{-m:m}(\ensuremath{\theta})|
&= |\ensuremath{\sum}_{k=-m}^m {(\hat f_k - \hat f_k^n)z^k} + \ensuremath{\sum}_{k=m+1}^\ensuremath{\infty} {\hat f_k z^k}  + \ensuremath{\sum}_{k=-\ensuremath{\infty}}^{-m-1} {\hat f_k z^k} | \\
&\le \ensuremath{\sum}_{k=-m}^m | \hat f_k - \hat f_k^n | + \ensuremath{\sum}_{k=m+1}^\ensuremath{\infty} |\hat f_k| +  \ensuremath{\sum}_{k=-\ensuremath{\infty}}^{-m-1} |\hat f_k| \\
&\le \ensuremath{\sum}_{k=-m}^m {\frac{C_{\ensuremath{\lambda}}}{ n^{\ensuremath{\lambda}}} \left( \left( \frac{1}{2} + k/n \right)^{ - \ensuremath{\lambda} + 1} + \left( \left( \frac{1}{2} - k/n \right) \right)^{- \ensuremath{\lambda} +1} \right)} + \ensuremath{\sum}_{k=m+1}^\ensuremath{\infty} |\hat f_k| +  \ensuremath{\sum}_{k=-\ensuremath{\infty}}^{-m-1} |\hat f_k| \\
&= \frac{C_{\ensuremath{\lambda}}}{n^{\ensuremath{\lambda}}} 2^{\ensuremath{\lambda}} + \ensuremath{\sum}_{k=m+1}^\ensuremath{\infty} |\hat f_k| +  \ensuremath{\sum}_{k=-\ensuremath{\infty}}^{-m-1} |\hat f_k|  \\
&= O(n^{-\ensuremath{\lambda}}) + O(n^{1-\ensuremath{\lambda}} ) + O(n^{1-\ensuremath{\lambda}} ) \\
&= O(n^{1-\ensuremath{\lambda}})
\end{align*}
\textbf{END}

\textbf{Problem 3(c)} Show that $f_{-m:m}(\ensuremath{\theta})$ interpolates $f$ at $\ensuremath{\theta}_j = 2\ensuremath{\pi}j/n$ for $n = 2m+1$.

\textbf{SOLUTION} Note from the aliasing property we have
\meeq{
 \hat f_k^n {\rm e}^{{\rm i} k \ensuremath{\theta}_j} =  \hat f_k^n {\rm e}^{2 \ensuremath{\pi} {\rm i} kj/n} = \hat f_{k+n}^n {\rm e}^{2 \ensuremath{\pi} {\rm i} (k+n) j/n} \ccr
 = \hat f_{k+n}^n  {\rm e}^{{\rm i} (k+n) \ensuremath{\theta}_j}
}
Thus we have
\meeq{
f_{-m:m}(\ensuremath{\theta}_j) = \ensuremath{\sum}_{k=-m}^{-1} \hat f_k^n {\rm e}^{{\rm i} k \ensuremath{\theta}_j} + \ensuremath{\sum}_{k=0}^m \hat f_k^n {\rm e}^{{\rm i} k \ensuremath{\theta}_j} \ccr
 = \ensuremath{\sum}_{k=-m}^{-1} \hat f_{k+n}^n {\rm e}^{{\rm i} (k+n) \ensuremath{\theta}_j} + \ensuremath{\sum}_{k=0}^m \hat f_k^n {\rm e}^{{\rm i} k \ensuremath{\theta}_j} \ccr
 = \ensuremath{\sum}_{k=n-m}^{n-1} \hat f_{k}^n {\rm e}^{{\rm i} (k) \ensuremath{\theta}_j} + \ensuremath{\sum}_{k=0}^m \hat f_k^n {\rm e}^{{\rm i} k \ensuremath{\theta}_j} = f_n(\ensuremath{\theta}_j) = f(\ensuremath{\theta}_j)
}
\textbf{END}

\textbf{Problem 4(a)} Show for $0 \ensuremath{\leq} k,\ensuremath{\ell} \ensuremath{\leq} n-1$
\[
{1 \over n} \ensuremath{\sum}_{j=1}^n \cos k \ensuremath{\theta}_j \cos \ensuremath{\ell} \ensuremath{\theta}_j = \begin{cases} 1 & k = \ensuremath{\ell} = 0 \\
                                                  1/2 & k = \ensuremath{\ell} \\
                                                  0 & \hbox{otherwise}
                                                  \end{cases}
\]
for $\ensuremath{\theta}_j = \ensuremath{\pi}(j-1/2)/n$. Hint: Be careful as the $\ensuremath{\theta}_j$ differ from before, and only cover half the period, $[0,\ensuremath{\pi}]$. Using symmetry may help. You may also consider replacing $\cos$ with complex exponentials:
\[
\cos \ensuremath{\theta} = {{\rm e}^{{\rm i}\ensuremath{\theta}} + {\rm e}^{-{\rm i}\ensuremath{\theta}} \over 2}.
\]
\textbf{SOLUTION} The case $k = l = 0$ is immediate. Otherwise, we have,
\begin{align*}
\frac{1}{n}\ensuremath{\sum}_{j=1}^n \cos(k\ensuremath{\theta}_j)\cos(l\ensuremath{\theta}_j) &= {1 \over 4n}\ensuremath{\sum}_{j=1}^n \br[e^{i(k+l)\ensuremath{\theta}_j} + e^{-i(k+l)\ensuremath{\theta}_j} + e^{i(k-l)\ensuremath{\theta}_j} + e^{-i(k-l)\ensuremath{\theta}_j}].
\end{align*}
For $\ensuremath{\omega} = \exp(i \ensuremath{\pi}/n)$ and any $m$ not a multiple of $2n$ we have
\meeq{
\ensuremath{\sum}_{j=1}^n e^{im\ensuremath{\theta}_j} =  \ensuremath{\sum}_{j=0}^{n-1} e^{im \ensuremath{\pi}(j+1/2)/n} = e^{i m \ensuremath{\pi}/(2n)}  \ensuremath{\sum}_{j=0}^{n-1} e^{i m \ensuremath{\pi} j/n} = \ensuremath{\omega}^{m/2} \ensuremath{\sum}_{j=0}^{n-1} \ensuremath{\omega}^{m j} \ccr
 =  \ensuremath{\omega}^{m/2} {\ensuremath{\omega}^{nm} - 1 \over \ensuremath{\omega}^m - 1} =  \ensuremath{\omega}^{m/2} {(-1)^m - 1 \over \ensuremath{\omega}^m - 1}
}
and hence
\meeq{
\ensuremath{\sum}_{j=1}^n [ e^{im\ensuremath{\theta}_j} +  e^{-im\ensuremath{\theta}_j}] = \ensuremath{\omega}^{m/2} {(-1)^m - 1 \over \ensuremath{\omega}^m - 1} + \ensuremath{\omega}^{-m/2} {(-1)^m - 1 \over \ensuremath{\omega}^{-m} - 1} \ccr
= \ensuremath{\omega}^{m/2} {(-1)^m - 1 \over \ensuremath{\omega}^m - 1} + \ensuremath{\omega}^{m/2} {(-1)^m - 1 \over 1 - \ensuremath{\omega}^m}  = 0.
}
Since $0 < k+l \ensuremath{\leq} 2n-2$ we know $k+l$ is not a multiple of $2n$ hence
\[
\ensuremath{\sum}_{j=1}^n \br[e^{i(k+l)\ensuremath{\theta}_j} + e^{-i(k+l)\ensuremath{\theta}_j}] = 0.
\]
Now if $k = l$ we have
\[
\ensuremath{\sum}_{j=1}^n e^{i(k-l)\ensuremath{\theta}_j} = \ensuremath{\sum}_{j=1}^n e^{-i(k-l)\ensuremath{\theta}_j} = n.
\]
Otherwise $k-l \ensuremath{\neq} 0$ but also $1-n \ensuremath{\leq} k-l \ensuremath{\leq} n-1$ hence $k-l$ cannot be a multiple of $2n$. And thus
\[
\ensuremath{\sum}_{j=1}^n \br[e^{i(k-l)\ensuremath{\theta}_j} + e^{i(l-k)\ensuremath{\theta}_j}] = 0.
\]
\textbf{END}

\textbf{Problem 4(b)} Consider the Discrete Cosine Transform (DCT)
\[
C_n := \begin{bmatrix}
\sqrt{1/n} \\
 & \sqrt{2/n} \\
 && \ensuremath{\ddots} \\
 &&& \sqrt{2/n}
 \end{bmatrix}
\begin{bmatrix}
    1 & \ensuremath{\cdots} & 1\\
    \cos \ensuremath{\theta}_1 & \ensuremath{\cdots} & \cos \ensuremath{\theta}_n \\
    \ensuremath{\vdots} & \ensuremath{\ddots} & \ensuremath{\vdots} \\
    \cos (n-1)\ensuremath{\theta}_1 & \ensuremath{\cdots} & \cos (n-1)\ensuremath{\theta}_n
\end{bmatrix}
\]
for $\ensuremath{\theta}_j = \ensuremath{\pi}(j-1/2)/n$. Prove that $C_n$ is orthogonal: $C_n^\ensuremath{\top} C_n = C_n C_n^\ensuremath{\top} = I$.

\textbf{SOLUTION}

The components of $C_n$ are
\[
\ensuremath{\bm{\e}}_k^\ensuremath{\top} C_n  \ensuremath{\bm{\e}}_j =  {1 \over \sqrt{n}} \begin{cases} 1 & k = 1 \\ \sqrt{2} & k \ensuremath{\neq} 1 \end{cases} \cos((k-1)\ensuremath{\theta}_j),
\]
where $\ensuremath{\theta}_j = \ensuremath{\pi}(j-1/2)/n$. We find using the previous part:
\meeq{
\ensuremath{\bm{\e}}_k^\ensuremath{\top} C_n C_n^\ensuremath{\top} \ensuremath{\bm{\e}}_\ensuremath{\ell}  = \pr({\begin{cases} 1 & k = \ensuremath{\ell} = 1 \\ \sqrt{2} & k,\ensuremath{\ell} = 1, k \ensuremath{\neq} \ensuremath{\ell} \\
                                                    2 & k,\ensuremath{\ell} \ensuremath{\neq} 1 \end{cases}})
     {1 \over n} \ensuremath{\sum}_{j=1}^n \cos((k-1)\ensuremath{\theta}_j) \cos((\ensuremath{\ell}-1)\ensuremath{\theta}_j) \ccr
    = \pr({\begin{cases} 1 & k = \ensuremath{\ell} = 1 \\ \sqrt{2} & k,\ensuremath{\ell} = 1, k \ensuremath{\neq} \ensuremath{\ell} \\
                                                    2 & k,\ensuremath{\ell} \ensuremath{\neq} 1 \end{cases}}) \pr({\begin{cases} 1 & k = \ensuremath{\ell} = 1 \\
                                                  1/2 & k = \ensuremath{\ell} \\
                                                  0 & \hbox{otherwise}
                                                  \end{cases}})= \ensuremath{\delta}_{k \ensuremath{\ell}}.
}
\textbf{END}

\textbf{Problem 5} What polynomial interpolates $\cos z$ at $1, \exp(2\ensuremath{\pi}\I/3)$ and $\exp(-2\ensuremath{\pi}\I/3)$?

\textbf{SOLUTION} For $\ensuremath{\omega} =  \exp(2\ensuremath{\pi}\I/3)$, we use the DFT:
\meeq{
\Vectt[\hat f_0^3, \hat f_1^3, \hat f_2^3, \hat f_2^3] =
{1 \over 3} \begin{bmatrix}1 & 1 & 1  \\
                            1 &\exp(-2\ensuremath{\pi}\I/3) & \exp(2\ensuremath{\pi}\I/3) \\
                            1 & \exp(2\ensuremath{\pi}\I/3)  & \exp(-2\ensuremath{\pi}\I/3)
                            \end{bmatrix}
 \Vectt[\cos(1), \cos(\exp(2\ensuremath{\pi}\I/3)), \cos(\exp(-2\ensuremath{\pi}\I/3))] \ccr
 = {1 \over 3} \Vectt[\cos(1)+ \cos(\exp(2\ensuremath{\pi}\I/3))+ \cos(\exp(-2\ensuremath{\pi}\I/3)) ,
 \cos(1)+ \exp(-2\ensuremath{\pi}\I/3) \cos(\exp(2\ensuremath{\pi}\I/3))+ \exp(2\ensuremath{\pi}\I/3) \cos(\exp(-2\ensuremath{\pi}\I/3)) ,
 \cos(1)+ \exp(2\ensuremath{\pi}\I/3) \cos(\exp(2\ensuremath{\pi}\I/3))+ \exp(-2\ensuremath{\pi}\I/3) \cos(\exp(-2\ensuremath{\pi}\I/3))
 ]
}
That is, the polynomial is
\begin{align*}
&{\cos(1) + \cos(\exp(2\ensuremath{\pi}\I/3))+ \cos(\exp(-2\ensuremath{\pi}\I/3)) \over 3}  \cr
&+ {\cos(1)+ \exp(-2\ensuremath{\pi}\I/3) \cos(\exp(2\ensuremath{\pi}\I/3))+ \exp(2\ensuremath{\pi}\I/3) \cos(\exp(-2\ensuremath{\pi}\I/3))\over 3} z \cr
&+{\cos(1)+ \exp(2\ensuremath{\pi}\I/3) \cos(\exp(2\ensuremath{\pi}\I/3))+ \exp(-2\ensuremath{\pi}\I/3) \cos(\exp(-2\ensuremath{\pi}\I/3)) \over 3} z^2.
\end{align*}
\textbf{END}



\end{document}