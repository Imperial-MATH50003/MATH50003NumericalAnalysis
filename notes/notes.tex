\documentclass[12pt,a4paper]{book}

\usepackage[a4paper,text={16.5cm,25.2cm},centering]{geometry}
\usepackage{lmodern}
\usepackage{amssymb,amsmath}
\usepackage{bm}
\usepackage{graphicx}
\usepackage{microtype}
\usepackage{hyperref}
\usepackage{amsthm}
\usepackage{upquote}
\usepackage{listings}
\usepackage{appendix}
\usepackage[usename®s,dvipsnames]{xcolor}
\setlength{\parindent}{0pt}
\setlength{\parskip}{1.2ex}

\lstset{
    basicstyle=\ttfamily\footnotesize,
    upquote=true,
    breaklines=true,
    breakindent=0pt,
    keepspaces=true,
    showspaces=false,
    columns=fullflexible,
    showtabs=false,
    showstringspaces=false,
    escapeinside={(*@}{@*)},
    extendedchars=true,
}

\newcommand{\HLJLt}[1]{#1}
\newcommand{\HLJLw}[1]{#1}
\newcommand{\HLJLe}[1]{#1}
\newcommand{\HLJLeB}[1]{#1}
\newcommand{\HLJLo}[1]{#1}
\newcommand{\HLJLk}[1]{\textcolor[RGB]{148,91,176}{\textbf{#1}}}
\newcommand{\HLJLkc}[1]{\textcolor[RGB]{59,151,46}{\textit{#1}}}
\newcommand{\HLJLkd}[1]{\textcolor[RGB]{214,102,97}{\textit{#1}}}
\newcommand{\HLJLkn}[1]{\textcolor[RGB]{148,91,176}{\textbf{#1}}}
\newcommand{\HLJLkp}[1]{\textcolor[RGB]{148,91,176}{\textbf{#1}}}
\newcommand{\HLJLkr}[1]{\textcolor[RGB]{148,91,176}{\textbf{#1}}}
\newcommand{\HLJLkt}[1]{\textcolor[RGB]{148,91,176}{\textbf{#1}}}
\newcommand{\HLJLn}[1]{#1}
\newcommand{\HLJLna}[1]{#1}
\newcommand{\HLJLnb}[1]{#1}
\newcommand{\HLJLnbp}[1]{#1}
\newcommand{\HLJLnc}[1]{#1}
\newcommand{\HLJLncB}[1]{#1}
\newcommand{\HLJLnd}[1]{\textcolor[RGB]{214,102,97}{#1}}
\newcommand{\HLJLne}[1]{#1}
\newcommand{\HLJLneB}[1]{#1}
\newcommand{\HLJLnf}[1]{\textcolor[RGB]{66,102,213}{#1}}
\newcommand{\HLJLnfm}[1]{\textcolor[RGB]{66,102,213}{#1}}
\newcommand{\HLJLnp}[1]{#1}
\newcommand{\HLJLnl}[1]{#1}
\newcommand{\HLJLnn}[1]{#1}
\newcommand{\HLJLno}[1]{#1}
\newcommand{\HLJLnt}[1]{#1}
\newcommand{\HLJLnv}[1]{#1}
\newcommand{\HLJLnvc}[1]{#1}
\newcommand{\HLJLnvg}[1]{#1}
\newcommand{\HLJLnvi}[1]{#1}
\newcommand{\HLJLnvm}[1]{#1}
\newcommand{\HLJLl}[1]{#1}
\newcommand{\HLJLld}[1]{\textcolor[RGB]{148,91,176}{\textit{#1}}}
\newcommand{\HLJLs}[1]{\textcolor[RGB]{201,61,57}{#1}}
\newcommand{\HLJLsa}[1]{\textcolor[RGB]{201,61,57}{#1}}
\newcommand{\HLJLsb}[1]{\textcolor[RGB]{201,61,57}{#1}}
\newcommand{\HLJLsc}[1]{\textcolor[RGB]{201,61,57}{#1}}
\newcommand{\HLJLsd}[1]{\textcolor[RGB]{201,61,57}{#1}}
\newcommand{\HLJLsdB}[1]{\textcolor[RGB]{201,61,57}{#1}}
\newcommand{\HLJLsdC}[1]{\textcolor[RGB]{201,61,57}{#1}}
\newcommand{\HLJLse}[1]{\textcolor[RGB]{59,151,46}{#1}}
\newcommand{\HLJLsh}[1]{\textcolor[RGB]{201,61,57}{#1}}
\newcommand{\HLJLsi}[1]{#1}
\newcommand{\HLJLso}[1]{\textcolor[RGB]{201,61,57}{#1}}
\newcommand{\HLJLsr}[1]{\textcolor[RGB]{201,61,57}{#1}}
\newcommand{\HLJLss}[1]{\textcolor[RGB]{201,61,57}{#1}}
\newcommand{\HLJLssB}[1]{\textcolor[RGB]{201,61,57}{#1}}
\newcommand{\HLJLnB}[1]{\textcolor[RGB]{59,151,46}{#1}}
\newcommand{\HLJLnbB}[1]{\textcolor[RGB]{59,151,46}{#1}}
\newcommand{\HLJLnfB}[1]{\textcolor[RGB]{59,151,46}{#1}}
\newcommand{\HLJLnh}[1]{\textcolor[RGB]{59,151,46}{#1}}
\newcommand{\HLJLni}[1]{\textcolor[RGB]{59,151,46}{#1}}
\newcommand{\HLJLnil}[1]{\textcolor[RGB]{59,151,46}{#1}}
\newcommand{\HLJLnoB}[1]{\textcolor[RGB]{59,151,46}{#1}}
\newcommand{\HLJLoB}[1]{\textcolor[RGB]{102,102,102}{\textbf{#1}}}
\newcommand{\HLJLow}[1]{\textcolor[RGB]{102,102,102}{\textbf{#1}}}
\newcommand{\HLJLp}[1]{#1}
\newcommand{\HLJLc}[1]{\textcolor[RGB]{153,153,119}{\textit{#1}}}
\newcommand{\HLJLch}[1]{\textcolor[RGB]{153,153,119}{\textit{#1}}}
\newcommand{\HLJLcm}[1]{\textcolor[RGB]{153,153,119}{\textit{#1}}}
\newcommand{\HLJLcp}[1]{\textcolor[RGB]{153,153,119}{\textit{#1}}}
\newcommand{\HLJLcpB}[1]{\textcolor[RGB]{153,153,119}{\textit{#1}}}
\newcommand{\HLJLcs}[1]{\textcolor[RGB]{153,153,119}{\textit{#1}}}
\newcommand{\HLJLcsB}[1]{\textcolor[RGB]{153,153,119}{\textit{#1}}}
\newcommand{\HLJLg}[1]{#1}
\newcommand{\HLJLgd}[1]{#1}
\newcommand{\HLJLge}[1]{#1}
\newcommand{\HLJLgeB}[1]{#1}
\newcommand{\HLJLgh}[1]{#1}
\newcommand{\HLJLgi}[1]{#1}
\newcommand{\HLJLgo}[1]{#1}
\newcommand{\HLJLgp}[1]{#1}
\newcommand{\HLJLgs}[1]{#1}
\newcommand{\HLJLgsB}[1]{#1}
\newcommand{\HLJLgt}[1]{#1}


\let\QED=\blacksquare
\def\bbD{{\mathbb D}}
\def\bbZ{{\mathbb Z}}
\def\bbN{{\mathbb N}}
\def\bbF{{\mathbb F}}
\def\bbR{{\mathbb R}}
\def\bbT{{\mathbb T}}
\def\bbC{{\mathbb C}}
\def\emdash{\hbox{---}}
\def\endash{\hbox{--}}
\def\nsubset{\not\subset}
\def\ldq{``}
\def\x{{\vc x}}
\def\a{{\vc a}}
\def\b{{\vc b}}
\def\q{{\vc q}}
\def\c{{\vc c}}
\def\e{{\vc e}}
\def\f{{\vc f}}
\def\u{{\vc u}}
\def\w{{\vc w}}
\def\v{{\vc v}}
\def\y{{\vc y}}
\def\z{{\vc z}}
\def\k{{\vc k}}
\def\vchatf{{\vc {\hat f}}}
\def\zero{{\vc 0}}
\def\Lt{{\tilde L}}
\def\Pt{{\tilde P}}
\def\pt{{\tilde p}}
\def\Ut{{\tilde U}}
\def\baralpha{\bar\alpha}
\def\At{\tilde A}
\def\Rt{\tilde R}
\def\red#1{{\color{red} #1}}
\def\blue#1{{\color{blue} #1}}
\def\green#1{{\color{ForestGreen} #1}}
\def\euler{\E}
\def\ocaret{\wedge\mkern-19mu \bigcirc\,}

\def\fldown{{\rm fl}^{\rm down}}
\def\flup{{\rm fl}^{\rm up}}

\hypersetup
       {   pdfauthor = { {{Sheehan Olver}} },
           pdftitle={ {{MATH50003 Numerical Analysis}} },
           colorlinks=TRUE,
           linkcolor=black,
           citecolor=blue,
           urlcolor=blue
       }

\title{ MATH50003 Numerical Analysis }


\newtheorem{lemma}{Lemma}
\newtheorem{theorem}{Theorem}
\newtheorem{proposition}{Proposition}
\newtheorem{corollary}{Corollary}

\theoremstyle{definition}
\newtheorem{definition}{Definition}
\newtheorem{example}{Example}

\author{ Sheehan Olver }
\renewcommand{\thechapter}{\Roman{chapter}}


\def\addtab#1={#1\;&=}

\def\meeq#1{\def\ccr{\\\addtab}
%\tabskip=\@centering
 \begin{align*}
 \addtab#1
 \end{align*}
  }  
  
  \def\leqaddtab#1\leq{#1\;&\leq}
  \def\mleeq#1{\def\ccr{\\\addtab}
%\tabskip=\@centering
 \begin{align*}
 \leqaddtab#1
 \end{align*}
  }  


\def\vc#1{\mbox{\boldmath$#1$\unboldmath}}

\def\vcsmall#1{\mbox{\boldmath$\scriptstyle #1$\unboldmath}}

\def\vczero{{\mathbf 0}}


%\def\beginlist{\begin{itemize}}
%
%\def\endlist{\end{itemize}}


\def\pr(#1){\left({#1}\right)}
\def\br[#1]{\left[{#1}\right]}
\def\fbr[#1]{\!\left[{#1}\right]}
\def\set#1{\left\{{#1}\right\}}
\def\ip<#1>{\left\langle{#1}\right\rangle}
\def\iip<#1>{\left\langle\!\langle{#1}\right\rangle\!\rangle}

\def\norm#1{\left\| #1 \right\|}

\def\abs#1{\left|{#1}\right|}
\def\fpr(#1){\!\pr({#1})}

\def\Re{{\rm Re}\,}
\def\Im{{\rm Im}\,}

\def\floor#1{\left\lfloor#1\right\rfloor}
\def\ceil#1{\left\lceil#1\right\rceil}


\def\mapengine#1,#2.{\mapfunction{#1}\ifx\void#2\else\mapengine #2.\fi }

\def\map[#1]{\mapengine #1,\void.}

\def\mapenginesep_#1#2,#3.{\mapfunction{#2}\ifx\void#3\else#1\mapengine #3.\fi }

\def\mapsep_#1[#2]{\mapenginesep_{#1}#2,\void.}


\def\vcbr{\br}


\def\bvect[#1,#2]{
{
\def\dots{\cdots}
\def\mapfunction##1{\ | \  ##1}
\begin{pmatrix}
		 \,#1\map[#2]\,
\end{pmatrix}
}
}

\def\vect[#1]{
{\def\dots{\ldots}
	\vcbr[{#1}]
}}

\def\vectt[#1]{
{\def\dots{\ldots}
	\vect[{#1}]^{\top}
}}

\def\Vectt[#1]{
{
\def\mapfunction##1{##1 \cr} 
\def\dots{\vdots}
	\begin{pmatrix}
		\map[#1]
	\end{pmatrix}
}}



\def\thetaB{\mbox{\boldmath$\theta$}}
\def\zetaB{\mbox{\boldmath$\zeta$}}


\def\newterm#1{{\it #1}\index{#1}}


\def\TT{{\mathbb T}}
\def\C{{\mathbb C}}
\def\R{{\mathbb R}}
\def\II{{\mathbb I}}
\def\F{{\mathcal F}}
\def\E{{\rm e}}
\def\I{{\rm i}}
\def\D{{\rm d}}
\def\dx{\D x}
\def\CC{{\cal C}}
\def\DD{{\cal D}}
\def\U{{\mathbb U}}
\def\A{{\cal A}}
\def\K{{\cal K}}
\def\DTU{{\cal D}_{{\rm T} \rightarrow {\rm U}}}
\def\LL{{\cal L}}
\def\B{{\cal B}}
\def\T{{\cal T}}
\def\W{{\cal W}}


\def\tF_#1{{\tt F}_{#1}}
\def\Fm{\tF_m}
\def\Fab{\tF_{\alpha,\beta}}
\def\FC{\T}
\def\FCpmz{\FC^{\pm {\rm z}}}
\def\FCz{\FC^{\rm z}}

\def\tFC_#1{{\tt T}_{#1}}
\def\FCn{\tFC_n}

\def\rmz{{\rm z}}

\def\chapref#1{Chapter~\ref{Chapter:#1}}
\def\secref#1{Section~\ref{Section:#1}}
\def\exref#1{Exercise~\ref{Exercise:#1}}
\def\lmref#1{Lemma~\ref{Lemma:#1}}
\def\propref#1{Proposition~\ref{Proposition:#1}}
\def\warnref#1{Warning~\ref{Warning:#1}}
\def\thref#1{Theorem~\ref{Theorem:#1}}
\def\defref#1{Definition~\ref{Definition:#1}}
\def\probref#1{Problem~\ref{Problem:#1}}
\def\corref#1{Corollary~\ref{Corollary:#1}}

\def\sgn{{\rm sgn}\,}
\def\Ai{{\rm Ai}\,}
\def\Bi{{\rm Bi}\,}
\def\wind{{\rm wind}\,}
\def\erf{{\rm erf}\,}
\def\erfc{{\rm erfc}\,}
\def\qqquad{\qquad\quad}
\def\qqqquad{\qquad\qquad}


\def\spand{\hbox{ and }}
\def\spodd{\hbox{ odd}}
\def\speven{\hbox{ even}}
\def\qand{\quad\hbox{and}\quad}
\def\qqand{\qquad\hbox{and}\qquad}
\def\qfor{\quad\hbox{for}\quad}
\def\qqfor{\qquad\hbox{for}\qquad}
\def\qas{\quad\hbox{as}\quad}
\def\qqas{\qquad\hbox{as}\qquad}
\def\qor{\quad\hbox{or}\quad}
\def\qqor{\qquad\hbox{or}\qquad}
\def\qqwhere{\qquad\hbox{where}\qquad}



%%% Words

\def\naive{na\"\i ve\xspace}
\def\Jmap{Joukowsky map\xspace}
\def\Mobius{M\"obius\xspace}
\def\Holder{H\"older\xspace}
\def\Mathematica{{\sc Mathematica}\xspace}
\def\apriori{apriori\xspace}
\def\WHf{Weiner--Hopf factorization\xspace}
\def\WHfs{Weiner--Hopf factorizations\xspace}

\def\Jup{J_\uparrow^{-1}}
\def\Jdown{J_\downarrow^{-1}}
\def\Jin{J_+^{-1}}
\def\Jout{J_-^{-1}}



\def\bD{\D\!\!\!^-}

\def\Abstract#1\par{\begin{abstract}#1\end{abstract}}
\def\Keywords#1\par{\begin{keywords}{#1}\end{keywords}}
\def\Section#1#2.{\section{#2}\label{Section:#1} }
\def\Appendix#1#2.{\appendix \section{#2}\label{Section:#1} }

\def\Subsectionl#1#2.{\subsection{#2}\label{subsec:#1}}
\def\Subsection#1.{\subsection{#1}}

\def\Subsubsection#1.{\subsubsection{#1}}


\def\Problem#1#2\par{\begin{problem}\label{Problem:#1} #2\end{problem}}
\def\Theorem#1#2\par{\begin{theorem}\label{Theorem:#1} #2\end{theorem}}
\def\Conjecture#1#2\par{\begin{conjecture}\label{Conjecture:#1} #2\end{conjecture}}
\def\Proposition#1#2\par{\begin{proposition}\label{Proposition:#1} #2\end{proposition}}
\def\Definition#1#2\par{\begin{definition}\label{Definition:#1} #2\end{definition}}
\def\Corollary#1#2\par{\begin{corollary}\label{Corollary:#1} #2\end{corollary}}
\def\Lemma#1#2\par{\begin{lemma}\label{Lemma:#1} #2\end{lemma}}
\def\Example#1#2\par{\begin{example}\label{Example:#1} #2\end{example}}
\def\Remark #1\par{\begin{remark*}#1\end{remark*}}

\def\figref#1{Figure~\ref{fig:#1}}

\def\Figurew[#1]#2#3\par{
\begin{figure}[tb]
\begin{center}{
\includegraphics[width=#2]{Figures/#1}}
\end{center}
\caption{#3}\label{fig:#1} 
\end{figure}
}

\def\Figure[#1]#2\par{
\begin{figure}[tb]
\begin{center}{
\includegraphics{Figures/#1}}
\end{center}
\caption{#2}\label{fig:#1} 
\end{figure}
}

\def\Figurefixed[#1]#2\par{
\Figurew[#1]{0.48 \hsize}{#2}\par
}

\def\Figuretwow#1#2#3#4\par{
\begin{figure}[tb]
\begin{center}{
\includegraphics[width=#3]{Figures/#1}\includegraphics[width=#3]{Figures/#2}}
\end{center}
\caption{#4}\label{fig:#1} 
\end{figure}
}

\def\Figuretwowframed#1#2#3#4\par{
\begin{figure}[tb]
\begin{center}{
\fbox{\includegraphics[width=#3]{Figures/#1}}\fbox{\includegraphics[width=#3]{Figures/#2}}}
\end{center}
\caption{#4}\label{fig:#1} 
\end{figure}
}

\def\Figuretwo[#1,#2]#3\par{
	\Figuretwow{#1}{#2}{0.48 \hsize}
		#3\par	
}

\def\Figuretwoframed[#1,#2]#3\par{
	\Figuretwowframed{#1}{#2}{0.48 \hsize}
		#3\par	
}

\def\Figurethreew#1#2#3#4#5\par{
\begin{figure}[tb]
\begin{center}{
\includegraphics[width=#4]{Figures/#1} \includegraphics[width=#4]{Figures/#2} \includegraphics[width=#4]{Figures/#3}}
\end{center}
\caption{#5}\label{fig:#1} %\prooflabel{#1}
\end{figure}
}

\def\Figurethree#1#2#3#4\par{
	\Figurethreew{#1}{#2}{#3}{0.3 \hsize}
		{#4}\par	
}

\def\Figurematrixfour#1#2#3#4#5\par{
\begin{figure}[tb]
\begin{center}{
\vbox{\hbox{\includegraphics[width= 0.48 \hsize]{Figures/#1} \includegraphics[width= 0.48 \hsize]{Figures/#2}}\hbox{\includegraphics[width= 0.48 \hsize]{Figures/#3}\includegraphics[width= 0.48 \hsize]{Figures/#4}}}}
\end{center}
\caption{#5}\label{fig:#1} %\prooflabel{#1}
\end{figure}
}


\def\questionequals{= \!\!\!\!\!\!{\scriptstyle ? \atop }\,\,\,}

\def\elll#1{\ell^{\lambda,#1}}
\def\elllp{\ell^{\lambda,p}}
\def\elllRp{\ell^{(\lambda,R),p}}


\def\elllRpz_#1{\ell_{#1{\rm z}}^{(\lambda,R),p}}


\def\sopmatrix#1{\begin{pmatrix}#1\end{pmatrix}}

\def\Proof{\begin{proof}}
\def\mqed{\end{proof}}

\gdef\reffilename{\jobname}
\def\References{\bibliography{\reffilename}}

\outer\def\ends{ 
\end{document}
}


\begin{document}

\maketitle

\tableofcontents

\chapter{Calculus on a Computer}

In this first chapter we explore the basics of mathematical computing and numerical analysis.
In particular we investigate the following mathematical problems which can not in general be solved exactly:

\begin{enumerate}
\item Integration. General integrals have no closed form expressions. Can we use a computer to approximate the values of definite integrals?
\item Differentiation. Differentiating a formula as in calculus is usually algorithmic, however, it is often needed to compute derivatives without access to an underlying formula, eg,  a function defined only in code. Can we use a computer to approximate derivatives?  A very important application is in Machine Learning, where there is a need to compute gradients to determine the ``right" weights in a neural network. 
\item Root finding. There is no general formula for finding roots (zeros) of arbitrary functions, or even polynomials that are of degree 5 (quintics) or higher. Can we compute roots of general functions using a computer?
\end{enumerate}

In this chapter we discuss:

\begin{enumerate}
\item I.1 Rectangular rule: we review the rectangular rule for integration and deduce the {\it converge rate} of the approximation. In the lab/problem sheet  we investigate its implementation as well as extensions to the Trapezium rule. 
\item I.2 Divided differences: we investigate approximating derivatives by a divided difference and again deduce the convergence rates. In the lab/problem sheet we extend the approach to the central differences formula and computing second derivatives. We also observe a mystery: the approximations may have significant errors in practice, and there is a limit to the accuracy.
\item I.3 Dual numbers: we introduce the algebraic notion of a {\it dual number} which allows the implemention of {\it forward-mode automatic differentiation}, a high accuracy alternative to divided differences for computing derivatives.
\item I.4 Newton's method: Newton's method is a basic approach for computing roots/zeros of a function. We use dual numbers to implement this algorithm.
\end{enumerate}




\section{Rectangular rule}
One possible definition for an integral is the limit of a Riemann sum, for example:
\[
  \ensuremath{\int}_a^b f(x) {\rm d}x = \lim_{n \ensuremath{\rightarrow} \ensuremath{\infty}} h \ensuremath{\sum}_{j=1}^n f(x_j)
\]
where $x_j = a+jh$ are evenly spaced points dividing up the interval $[a,b]$, that is  with the \emph{step size} $h = (b-a)/n$. This suggests an algorithm known as the \emph{(right-sided) rectangular rule} for approximating an integral: choose $n$ large so that
\[
  \ensuremath{\int}_a^b f(x) {\rm d}x \ensuremath{\approx} h \ensuremath{\sum}_{j=1}^n f(x_j).
\]
In the lab we explore practical implementation of this approximation, and observe that the error in approximation is bounded by $C/n$ for some constant $C$. This can be expressed using \ensuremath{\ldq}Big-O" notation:
\[
\ensuremath{\int}_a^b f(x) {\rm d}x = h \ensuremath{\sum}_{j=1}^n f(x_j) + O(1/n).
\]
In these notes we consider the \ensuremath{\ldq}Analysis" part of \ensuremath{\ldq}Numerical Analysis": we want to \emph{prove} the convergence rate of the approximation, including finding an explicit expression for the constant $C$.

To tackle this question we consider the error incurred on a single panel $(x_{j-1},x_j)$, then sum up the errors on rectangles.

Now for a secret. There are only so many tools available in analysis (especially at this stage of your career), and one can make a safe bet that the right tool in any analysis proof is either (1) integration-by-parts, (2) geometric series or (3) Taylor series. In this case we use (1):

\begin{lemma}[(Right-sided) Rectangular Rule error on one panel] Assuming $f$ is differentiable we have
\[
\ensuremath{\int}_a^b f(x) {\rm d}x = (b-a) f(b) + \ensuremath{\delta}
\]
where $|\ensuremath{\delta}| \ensuremath{\leq} M (b-a)^2$ for $M = \sup_{a \ensuremath{\leq} x \ensuremath{\leq} b}|f'(x)|$.

\end{lemma}
\textbf{Proof} We write
\meeq{
\ensuremath{\int}_a^b f(x) {\rm d}x = \ensuremath{\int}_a^b (x-a)' f(x)  {\rm d}x = [(x-a) f(x)]_a^b - \ensuremath{\int}_a^b (x-a) f'(x) {\rm d} x \ccr
= (b-a) f(b) + \underbrace{\left(-\ensuremath{\int}_a^b (x-a) f'(x) {\rm d} x \right)}_\ensuremath{\delta}.
}
Recall that we can bound the absolute value of an integral by the supremum of the integrand times the width of the integration interval:
\[
\abs{\ensuremath{\int}_a^b g(x) {\rm d} x} \ensuremath{\leq} (b-a) \sup_{a \ensuremath{\leq} x \ensuremath{\leq} b}|g(x)|.
\]
The lemma thus follows since
\begin{align*}
\abs{\ensuremath{\int}_a^b (x-a) f'(x) {\rm d} x} &\ensuremath{\leq} (b-a) \sup_{a \ensuremath{\leq} x \ensuremath{\leq} b}|(x-a) f'(x)|  \\
&\ensuremath{\leq} (b-a) \sup_{a \ensuremath{\leq} x \ensuremath{\leq} b}|x-a| \sup_{a \ensuremath{\leq} x \ensuremath{\leq} b}|f'(x)|\\
&\ensuremath{\leq} M (b-a)^2.
\end{align*}
\ensuremath{\QED}

Now summing up the errors in each panel gives us the error of using the Rectangular rule:

\begin{theorem}[Rectangular Rule error] Assuming $f$ is differentiable we have
\[
\ensuremath{\int}_a^b f(x) {\rm d}x =  h \ensuremath{\sum}_{j=1}^n f(x_j) +  \ensuremath{\delta}
\]
where $|\ensuremath{\delta}| \ensuremath{\leq} M (b-a) h$ for $M = \sup_{a \ensuremath{\leq} x \ensuremath{\leq} b}|f'(x)|$, $h = (b-a)/n$ and $x_j = a + jh$.

\end{theorem}
\textbf{Proof} We split the integral into a sum of smaller integrals:
\[
\ensuremath{\int}_a^b f(x) {\rm d}x = \ensuremath{\sum}_{j=1}^n  \ensuremath{\int}_{x_{j-1}}^{x_j} f(x) {\rm d}x =
\ensuremath{\sum}_{j=1}^n  \br[(x_j - x_{j-1}) f(x_j) + \ensuremath{\delta}_j] =  h \ensuremath{\sum}_{j=1}^n f(x_j) +  \underbrace{\ensuremath{\sum}_{j=1}^n \ensuremath{\delta}_j}_\ensuremath{\delta}
\]
where $\ensuremath{\delta}_j$, the error on each panel as in the preceding lemma, satisfies
\[
|\ensuremath{\delta}_j| \ensuremath{\leq} (x_j-x_{j-1})^2 \sup_{x_{j-1} \ensuremath{\leq} x \ensuremath{\leq} x_j}|f'(x)| \ensuremath{\leq} M h^2.
\]
Thus using the triangular inequality we have
\[
|\ensuremath{\delta}| = \abs{ \ensuremath{\sum}_{j=1}^n \ensuremath{\delta}_j} \ensuremath{\leq} \ensuremath{\sum}_{j=1}^n |\ensuremath{\delta}_j| \ensuremath{\leq} M n h^2 = M(b-a)h.
\]
\ensuremath{\QED}

Note a consequence of this lemma is that the approximation converges as $n \ensuremath{\rightarrow} \ensuremath{\infty}$ (i.e. $h \ensuremath{\rightarrow} 0$). In the labs and problem sheets we will consider the left-sided rule:
\[
\ensuremath{\int}_a^b f(x) {\rm d}x \ensuremath{\approx}  h \ensuremath{\sum}_{j=0}^{n-1} f(x_j).
\]
We also consider the \emph{Trapezium rule}. Here we approximate an integral  by an affine function:
\[
\ensuremath{\int}_a^b f(x) {\rm d} x \ensuremath{\approx} \ensuremath{\int}_a^b {(b-x)f(a) + (x-a)f(b) \over b-a} \dx
= {b-a \over 2} \br[f(a) + f(b)].
\]
Subdividing an interval $a = x_0 < x_1 < \ensuremath{\ldots} < x_n = b$ and applying this approximation separately on each subinterval $[x_{j-1},x_j]$, where $h = (b-a)/n$ and $x_j = a + jh$, leads to the approximation
\[
\ensuremath{\int}_a^b f(x) {\rm d}x \ensuremath{\approx}  {h \over 2} f(a) + h \ensuremath{\sum}_{j=1}^{n-1} f(x_j) + {h \over 2} f(b)
\]
We shall see both experimentally and provably that this approximation converges faster than the rectangular rule.





\section{Divided Differences}
Given a function, how can we approximate its derivative at a point? We consider an intuitive approach to this problem using \emph{(Right-sided) Divided Differences}: 
\[
f'(x) \ensuremath{\approx} {f(x+h) - f(x) \over h}
\]
Note by the definition of the derivative we know that this approximation will converge to the true derivative as $h \ensuremath{\rightarrow} 0$. But in numerical approimxations we also need to consider the rate of convergence. 

Now in the previous section I mentioned there are three basic tools in analysis:  (1) integration-by-parts, (2) geometric series or (3) Taylor series. In this case we use (3):

\begin{proposition}[divided differences error] Suppose that $f$ is twice-differentiable on the interval $[x,x+h]$. The error in approximating the derivative using divided differences is
\[
f'(x) = {f(x+h) - f(x) \over h} + \ensuremath{\delta}
\]
where $|\ensuremath{\delta}| \ensuremath{\leq} Mh/2$ for  $M = \sup_{x \ensuremath{\leq} t \ensuremath{\leq} x+h} |f''(t)|$.

\end{proposition}
\textbf{Proof} Follows immediately from Taylor's theorem: recall that
\[
f(x+h) = f(x) + f'(x) h + {f''(t) \over 2} h^2
\]
for some $t \ensuremath{\in} [x,x+h]$. Rearranging we get
\[
f'(x) = {f(x+h) - f(x) \over 2} + \underbrace{\left(-{f''(t)\over 2 h^2} \right)}_\ensuremath{\delta}.
\]
We then bound:
\[
|\ensuremath{\delta}| \ensuremath{\leq} \abs{{f''(t) \over 2} h} \ensuremath{\leq} {M  h \over 2}.
\]
\ensuremath{\QED}

Unlike the rectangular rule, the computational cost of computing the divided difference is independent of $h$! We only need to evaluate a function $f$ twice and do a single division. Here we are assuming that the computational cost of evaluating $f$ is independent of the point of evaluation. Later we will investigate the details of how computers work with numbers via floating point,  and confirm that this is a sensible assumption.

So why not just set $h$ ridiculously small? In the lab we explore this question and observe that there are significant errors introduced in the numerical realisation of this algorithm. We will return to the question of understanding these errors after learning floating point numbers. 

There are alternative versions of divided differences. Left-side divided differences evaluates to the left of the point where we wish to know the derivative:
\[
f'(x) \ensuremath{\approx} {f(x) - f(x-h) \over h}
\]
and central differences:
\[
f'(x) \ensuremath{\approx} {f(x + h) - f(x - h) \over 2h}
\]
We can further arrive at an approximation to the second derivative by composing a left- and right-sided finite difference:
\[
f''(x) \ensuremath{\approx} {f'(x+h) - f'(x) \over h} \ensuremath{\approx} {{f(x+h) - f(x) \over h} - {f(x) - f(x-h) \over h} \over h}
= {f(x+h) - 2f(x)  + f(x-h) \over h^2}
\]
In the lab we investigate the convergence rate of these approximations (in particular, that  central differences is more accurate than standard divided differences) and observe that they too suffer from unexplained (for now) loss of accuracy as $h \ensuremath{\rightarrow} 0$. In the problem sheet we prove the theoretical convergence rate, which is never realised because of these errors.




% 
\section{Dual Numbers}
In this section we introduce a mathematically beautiful  alternative to divided differences for computing derivatives: \emph{dual numbers}. These are a commutative ring that \emph{exactly} compute derivatives, which when implemented on a computer gives very high-accuracy approximations to derivatives. They underpin forward-mode \href{https://en.wikipedia.org/wiki/Automatic_differentiation}{automatic differentation}. Automatic differentiation  is a basic tool in Machine Learning for computing gradients necessary for training neural networks.

\begin{definition}[Dual numbers] Dual numbers $\ensuremath{\bbD}$ are a commutative ring (over $\ensuremath{\bbR}$) generated by $1$ and $\ensuremath{\epsilon}$ such that $\ensuremath{\epsilon}^2 = 0$, that is,
\[
\ensuremath{\bbD} := \{a + b\ensuremath{\epsilon} \quad :\quad
 a,b \in \ensuremath{\bbR}, \quad \ensuremath{\epsilon}^2 =0 \}.
\]
\end{definition}

This is very much analoguous to complex numbers, which are a field generated by $1$ and $\I$ such that $\I^2 = -1$, that is,
\[
\ensuremath{\bbC} := \{a + b\I \quad :\quad
 a,b \in \ensuremath{\bbR}, \quad \I^2 =-1 \}.
\]
Compare multiplication of each number type which falls out of the rules of the generators:
\meeq{
(a + b \I) (c + d \I) = ac + (bc + ad) \I + bd \I^2 = ac -bd + (bc + ad) \I, \ccr
(a + b \ensuremath{\epsilon}) (c + d \ensuremath{\epsilon}) = ac + (bc + ad) \ensuremath{\epsilon} + bd \ensuremath{\epsilon}^2 = ac  + (bc + ad) \ensuremath{\epsilon}.
}
And just as we view $\ensuremath{\bbR} \ensuremath{\subset} \ensuremath{\bbC}$ by equating $a \ensuremath{\in} \ensuremath{\bbR}$ with $a + 0\I \ensuremath{\in} \ensuremath{\bbC}$, we can view $\ensuremath{\bbR} \ensuremath{\subset} \ensuremath{\bbD}$ by equating $a \ensuremath{\in} \ensuremath{\bbR}$ with $a + 0{\rm \ensuremath{\epsilon}} \ensuremath{\in} \ensuremath{\bbD}$.

Conceptually, dual numbers can be thought of as introducing an infinitesimally small $\ensuremath{\epsilon}$, where $\ensuremath{\epsilon}^2$ is so small it is treated as zero. This is the intuitive reason they allow for differentiation of functions. But we do not need to appeal to this calculus-like interpretation, instead, their construction and relationship to differentiation can be accomplished using purely algebraic reasoning.

\subsection{Differentiating polynomials}
Polynomials evaluated on dual numbers are well-defined as they depend only on the operations $+$ and $*$. From the formula for multiplication of dual numbers we deduce that evaluating a polynomial at a dual number $a + b \ensuremath{\epsilon}$ tells us the derivative of the polynomial at $a$:

\begin{theorem}[polynomials on dual numbers] Suppose $p$ is a polynomial. Then
\[
p(a + b \ensuremath{\epsilon}) = p(a) + b p'(a) \ensuremath{\epsilon}
\]
\end{theorem}
\textbf{Proof}

First consider $p(x) = x^n$ for $n \ensuremath{\geq} 0$.  The cases $n = 0$ and $n = 1$ are immediate. For $n > 1$ we have by induction:
\[
(a + b \ensuremath{\epsilon})^n = (a + b \ensuremath{\epsilon}) (a + b \ensuremath{\epsilon})^{n-1} = (a + b \ensuremath{\epsilon}) (a^{n-1} + (n-1) b a^{n-2} \ensuremath{\epsilon}) = a^n + b n a^{n-1} \ensuremath{\epsilon}.
\]
For a more general polynomial
\[
p(x) = \ensuremath{\sum}_{k=0}^n c_k x^k
\]
the result follows from linearity:
\[
p(a + b \ensuremath{\varepsilon}) = \ensuremath{\sum}_{k=0}^n c_k (a+b\ensuremath{\epsilon})^k = c_0 + \ensuremath{\sum}_{k=1}^n c_k (a^k +k b a^{k-1}\ensuremath{\epsilon})
= \ensuremath{\sum}_{k=0}^n c_k a^k + b \ensuremath{\sum}_{k=1}^n c_k k a^{k-1}\ensuremath{\epsilon} = p(a) + b p'(a) \ensuremath{\epsilon}.
\]
\ensuremath{\QED}

\begin{example}[differentiating polynomial] Consider computing $p'(2)$ where
\[
p(x) = (x-1)(x-2) + x^2.
\]
We can use dual numbers to differentiate, avoiding expanding in monomials or applying rules of differentiating:
\[
p(2+\ensuremath{\epsilon}) = (1+\ensuremath{\epsilon})\ensuremath{\epsilon} + (2+\ensuremath{\epsilon})^2 = \ensuremath{\epsilon} + 4 + 4\ensuremath{\epsilon} = 4 + \underbrace{5}_{p'(2)}\ensuremath{\epsilon}.
\]
\end{example}

\subsection{Differentiating other functions}
We can extend real-valued differentiable functions to dual numbers in a similar manner. First, consider a standard function with a Taylor series (e.g. ${\rm cos}$, ${\rm sin}$, ${\rm exp}$, etc.)
\[
f(x) = \ensuremath{\sum}_{k=0}^\ensuremath{\infty} f_k x^k
\]
so that $a$ is inside the radius of convergence. This leads naturally to a definition on dual numbers:
\meeq{
f(a + b \ensuremath{\epsilon}) = \ensuremath{\sum}_{k=0}^\ensuremath{\infty} f_k (a + b \ensuremath{\epsilon})^k = f_0 + \ensuremath{\sum}_{k=1}^\ensuremath{\infty} f_k (a^k + k a^{k-1} b \ensuremath{\epsilon}) = \ensuremath{\sum}_{k=0}^\ensuremath{\infty} f_k a^k +  \ensuremath{\sum}_{k=1}^\ensuremath{\infty} f_k k a^{k-1} b \ensuremath{\epsilon}  \ccr
  = f(a) + b f'(a) \ensuremath{\epsilon}.
}
More generally, given a differentiable function (which may not have a Taylor series) we can extend it to dual numbers:

\begin{definition}[dual extension] Suppose a real-valued function $f : \ensuremath{\Omega} \ensuremath{\rightarrow} \ensuremath{\bbR}$ is differentiable in $\ensuremath{\Omega} \ensuremath{\subset} \ensuremath{\bbR}$.  We can construct the \emph{dual extension} $\underline{f} : \ensuremath{\Omega} + \ensuremath{\epsilon}\ensuremath{\bbR} \ensuremath{\rightarrow} \ensuremath{\bbD}$ by defining
\[
\underline{f}(a + b \ensuremath{\epsilon}) := f(a) + b f'(a) \ensuremath{\epsilon}.
\]
By viewing $\ensuremath{\bbR} \ensuremath{\subset} \ensuremath{\bbD}$, it is natural to reuse the notation $f$ for the dual extension, hence when there's no chance of confusion we will identify $f(a + b \ensuremath{\epsilon}) \ensuremath{\equiv} \underline{f}(a+b \ensuremath{\epsilon})$. \end{definition}

Thus, for basic functions we have natural extensions:
\begin{align*}
\exp(a + b \ensuremath{\epsilon}) &:= \exp(a) + b \exp(a) \ensuremath{\epsilon} & (a,b \ensuremath{\in} \ensuremath{\bbR}) \\
\sin(a + b \ensuremath{\epsilon}) &:= \sin(a) + b \cos(a) \ensuremath{\epsilon} & (a,b \ensuremath{\in} \ensuremath{\bbR}) \\
\cos(a + b \ensuremath{\epsilon}) &:= \cos(a) - b \sin(a) \ensuremath{\epsilon} & (a,b \ensuremath{\in} \ensuremath{\bbR}) \\
\log(a + b \ensuremath{\epsilon}) &:= \log(a) + {b \over a} \ensuremath{\epsilon} & (a \ensuremath{\in} (0,\ensuremath{\infty}), b \ensuremath{\in} \ensuremath{\bbR}) \\
\sqrt{a+b \ensuremath{\epsilon}} &:= \sqrt{a} + {b \over 2 \sqrt{a}} \ensuremath{\epsilon} & (a \ensuremath{\in} (0,\ensuremath{\infty}), b \ensuremath{\in} \ensuremath{\bbR}) \\
|a + b \ensuremath{\epsilon}| &:= |a| + b\, {\rm sign} a\, \ensuremath{\epsilon} & (a \ensuremath{\in} \ensuremath{\bbR} \backslash \{0\} , b \ensuremath{\in} \ensuremath{\bbR})
\end{align*}
provided the function is differentiable at $a$. Note the last example does not have a convergent Taylor series (at $0$) but we can still extend it where it is differentiable.

Going further, we can add, multiply, and compose such dual-extensions. And the beauty is these automatically satisfy the right properties to be dual-extensions themselves, thus allowing for differentiation of  complicated functions built from basic differentiable building blocks.

The following lemma shows that addition and multiplication in some sense \ensuremath{\ldq}commute" with the dual-extension, hence we can recover the product rule from dual number multiplication:

\begin{lemma}[addition/multiplication] Suppose $f,g : \ensuremath{\Omega} \ensuremath{\rightarrow} \ensuremath{\bbR}$ are differentiable for $\ensuremath{\Omega} \ensuremath{\subset} \ensuremath{\bbR}$ and $c \ensuremath{\in} \ensuremath{\bbR}$. Then for $a \ensuremath{\in} \ensuremath{\Omega}$ and $b \ensuremath{\in} \ensuremath{\bbR}$ we have
\meeq{
\underline{f+g}(a+b\ensuremath{\epsilon}) = \underline{f}(a+b\ensuremath{\epsilon}) + \underline{g}(a+b\ensuremath{\epsilon}) \ccr 
\underline{c f}(a+b\ensuremath{\epsilon}) = c \underline{f}(a+b \ensuremath{\epsilon}) \ccr
\underline{f g}(a+b\ensuremath{\epsilon}) = \underline{f}(a+b \ensuremath{\epsilon}) \underline{g}(a+b \ensuremath{\epsilon})
}
\end{lemma}
\textbf{Proof} The first two are immediate due to linearity:
\meeq{
\underline{(f+g)}(a+b\ensuremath{\epsilon}) = (f+g)(a) + b(f + g)'(a) \ensuremath{\epsilon} \ccr
 = (f(a)+bf'(a)\ensuremath{\epsilon}) + (g(a)+bg'(a)\ensuremath{\epsilon}) = \underline{f}(a+b\ensuremath{\epsilon}) + \underline{g}(a+b\ensuremath{\epsilon}), \ccr
\underline{cf}(a+b\ensuremath{\epsilon}) = (cf)(a) + b(cf)'(a) \ensuremath{\epsilon} = c(f(a)+bf'(a)\ensuremath{\epsilon}) = c\underline{f}(a+b\ensuremath{\epsilon}).
}
The last property essentially captures the product rule of differentiation:
\meeq{
\underline{f g}(a+b\ensuremath{\epsilon}) = f(a)g(a) + b (f(a)g'(a)+f'(a) g'(a))\ensuremath{\epsilon}  \ccr
= (f(a)+bf'(a)\ensuremath{\epsilon})(g(a)+b g'(a)\ensuremath{\epsilon}) = \underline{f}(a+b \ensuremath{\epsilon}) \underline{g}(a+b \ensuremath{\epsilon}).
}
\ensuremath{\QED}

Furthermore composition recovers the chain rule:

\begin{lemma}[composition]  Suppose $f : \ensuremath{\Omega} \ensuremath{\rightarrow} \ensuremath{\Gamma}$ and $g : \ensuremath{\Gamma} \ensuremath{\rightarrow} \ensuremath{\bbR}$ are differentiable in $\ensuremath{\Omega},\ensuremath{\Gamma} \ensuremath{\subset} \ensuremath{\bbR}$. Then
\[
\underline{(f \ensuremath{\circ} g)}(a+b \ensuremath{\epsilon}) = \underline{f}(\underline{g}(a+b\ensuremath{\epsilon}))
\]
\end{lemma}
\textbf{Proof} Again it falls out of the properties of dual numbers:
\[
\underline{(f \ensuremath{\circ} g)}(a+b \ensuremath{\epsilon}) = f(g(a)) + bg'(a) f'(g(a)) \ensuremath{\epsilon} = \underline{f}(g(a)+bg'(a)\ensuremath{\epsilon}) = \underline{f}(\underline{g}(a+b\ensuremath{\epsilon}))
\]
\ensuremath{\QED}

A simple corollary is that any function defined in terms of addition, multiplication, composition, etc. of basic functions with dual-extensions will be differentiable via dual numbers. In this following example we see a practical realisation of this, where we differentiate a function by just evaluating it on dual numbers, implicitly, using the dual-extension for the basic build blocks:

\begin{example}[differentiating non-polynomial]

Consider differentiating $f(x) =  \exp(x^2 + \cos x)$ at the point $a = 1$, where  we automatically use the dual-extension of $\exp$ and $\cos$. We can differentiate $f$ by simply evaluating on the duals:
\[
f(1 + \ensuremath{\epsilon}) = \exp(1 + 2\ensuremath{\epsilon} + \cos 1 - \sin 1 \ensuremath{\epsilon}) =  \exp(1 + \cos 1) + \exp(1 + \cos 1) (2 - \sin 1) \ensuremath{\epsilon}.
\]
Therefore we deduce that
\[
f'(1) = \exp(1 + \cos 1) (2 - \sin 1).
\]
\end{example}




% 
\section{Newton's method}
In school you may recall learning Newton's method: a way of approximating zeros/roots to a function by using a local approximation by an affine function. That is, approximate a function $f(x)$ locally around an initial guess $x_0$ by its first order Taylor series:
\[
f(x) \ensuremath{\approx} f(x_0) + f'(x_0) (x-x_0)
\]
and then find the root of the right-hand side which is
\[
 f(x_0) + f'(x_0) (x-x_0) = 0 \ensuremath{\Leftrightarrow} x = x_0 - {f(x_0) \over f'(x_0)}.
\]
We can then repeat using this root as the new initial guess. In other words we have a sequence of \emph{hopefully} more accurate approximations:
\[
x_{k+1} = x_k - {f(x_k) \over f'(x_k)}.
\]
The convergence theory of Newton's method is rich and beautiful but outside the scope of this module. But provided $f$ is smooth, if $x_0$ is sufficiently close to a root this iteration will converge. 

Thus \emph{if} we can compute derivatives, we can (sometimes) compute roots. The lab will explore using dual numbers to accomplish this task. This is in some sense a baby version of how Machine Learning algorithms train neural networks.






% \chapter{Representing Numbers}

% In this chapter we aim to answer the question: when can we rely on computations done on a computer?  Why are some computations (differentiation via divided differences), extremely inaccurate whilst others (integration via rectangular rule) accurate up to about 16 digits?  In order to address these questions we need to dig deeper and understand at a basic level what a computer is actually doing when manipulating numbers. 

% Before we begin it is important to have a basic model of how a computer works. Our simplified model of a computer will consist of a \href{https://en.wikipedia.org/wiki/Central_processing_unit}{Central Processing Unit (CPU)}\ensuremath{\emdash}the  brains of the computer\ensuremath{\emdash}and \href{https://en.wikipedia.org/wiki/Computer_data_storage#Primary_storage}{Memory}\ensuremath{\emdash}where  data is stored. Inside the CPU there are \href{https://en.wikipedia.org/wiki/Processor_register}{registers}, where data is temporarily stored after being loaded from memory, manipulated by the CPU, then stored back to memory.  Memory is a sequence of bits: \texttt{1}s and \texttt{0}s, essentially ``on/off" switches, and memory is {\it finite}.  Finally, if one has a $p$-bit CPU (eg a 32-bit or 64-bit CPU), each register consists of exactly $p$-bits. Most likely $p = 64$ on your machine. 


% Thus representing numbers on a computer must overcome three fundamental limitations:
% \begin{enumerate}
% \item CPUs can only manipulate data $p$-bits at a time.
% \item Memory is finite (in particular at most $2^p$ bytes).
% \item There is no such thing as an ``error'': if anything goes wrong in the computation we must use some of the $p$-bits to indicate this.
% \end{enumerate}

% This is clearly problematic: there are an infinite number of integers and an uncountable number of reals! Each of which we need to store in precisely $p$-bits. Moreover, some operations are simply undefined, like division by 0.  This chapter discusses the solution used to this problem, alongside the mathematical analysis that is needed to understand the implications, in particular, that computations have {\it error}.

% In particular we discuss:

% \begin{enumerate}
% \item II.1 Integers: unsigned (non-negative) and signed integers are representable using exactly $p$-bits by using modular arithmetic in all operations.
% \item II.2 Reals:  real numbers are approximated by floating point numbers, which are a computers version of scientific notation.
% \item II.3 Floating Point Arithmetic:  arithmetic with floating point numbers is exact up-to-rounding, which introduces small-but-understandable errors in the computations. We explain how these errors can be analysed mathematically to get rigorous bounds. 
% \item II.4 Interval Arithmetic: rounding can be controlled in order to implement {\it interval arithmetic}, a way to compute rigorous bounds for computations. In the lab, we use this to compute up to 15 digits of ${\rm e} \equiv \exp 1$ rigorously with precise bounds on the error.
% \end{enumerate}


% \input{II.1.Integers}
% \input{II.2.Reals}
% \input{II.3.Arithmetic}
% \input{II.4.Intervals}


% \chapter{Numerical Linear Algebra}

% Many problems in mathematics are linear: for example, polynomial regression and
% differential equations. Numerical methods for such applications invariably result
% in (finite-dimensional) linear systems that must be solved numerically on a computer: 
% the dimensions of the problems are often in the 1000s, millions, or even billions.
% One would certainly not want to tackle that with Gaussian elimination by hand!
% In this chapter we discuss algorithms, and in particular matrix factorisations, that are
% computed using floating point operations. We also introduce some basic applications.



% In particular we discuss:

% \begin{enumerate}
%     \item III.1 Structured Matrices: we discuss special structured matrices such as triangular and tridiagonal.
%     \item III.2 Differential Equations: using divided differences we can reduce differential equations
%     to linear systems. This motivates the investigation of numerical algorithms for solving linear systems.
%     \item III.3 LU and Cholesky Factorisations: we look at computing a factorisation of a square matrix as a product of a lower and upper triangular matrix, including the special case where the matrix is symmetric positive
%     definite. Hidden in this is an algorithm to prove positive definiteness.
% \item III.4 Polynomial Regression: often in data science one needs to approximate data by a polynomial.
% We discuss how to reduce this problem to solving a rectangular least squares problem.
% \item III.5 Orthogonal Matrices: we discuss different types of orthogonal matrices, which will be used to simplify rectangular least squares problems.
% \item III.6 QR Factorisation: we introduce an algorithm to compute a factorisation of a rectangular matrix as a product of an orthogonal and upper triangular matrix, thereby solving least squares problems.
% \end{enumerate}

% 
\section{Structured Matrices}
We have seen how algebraic operations (\texttt{+}, \texttt{-}, \texttt{*}, \texttt{/}) are defined exactly in terms of rounding ($\ensuremath{\oplus}$, $\ensuremath{\ominus}$, $\ensuremath{\otimes}$, $\ensuremath{\oslash}$) for floating point numbers. Now we see how this allows us to do (approximate) linear algebra operations on matrices.

A matrix can be stored in different formats, in particular it is important for large scale simulations that we take advantage of \emph{sparsity}: if we know a matrix has entries that are guaranteed to be zero we can implement faster algorithms. We shall see that this comes up naturally in numerical methods for solving differential equations.

In particular, we will discuss some basic types of structure in matrices:

\begin{itemize}
\item[1. ] \emph{Dense}: This can be considered unstructured, where we need to store all entries in a vector or matrix. Matrix-vector multiplication reduces directly to standard algebraic operations. Solving linear systems with dense matrices will be discussed later.


\item[2. ] \emph{Triangular}: If a matrix is upper or lower triangular, multiplication requires roughly half the number of operations. Crucially, we can apply the inverse of a triangular matrix using forward- or back-substitution.


\item[3. ] \emph{Banded}: If a matrix is zero apart from entries a fixed distance from  the diagonal it is called banded and matrix-vector multiplication has a lower \emph{complexity}: the number of operations scales linearly with the dimension (instead of quadratically). We discuss three cases: diagonal, tridiagonal and bidiagonal matrices.

\end{itemize}
\textbf{Remark} For those who took the first half of the module, there was an important emphasis on working with \emph{linear operators} rather than \emph{matrices}. That is, there was an emphasis on basis-independent mathematical techniques, which is critical for extension of results to infinite-dimensional spaces (which might not have a complete basis). However, in terms of practical computation we need to work with some representation of an operator and the most natural is a matrix. And indeed we will see in the next section how infinite-dimensional differential equations can be solved by reduction to finite-dimensional matrices. (Restricting attention to matrices is also important as some of the students have not taken the first half of the module.)

\subsection{Dense matrices}
A basic operation is matrix-vector multiplication. For a field $\ensuremath{\bbF}$ (typically $\ensuremath{\bbR}$ or $\ensuremath{\bbC}$, or this can be relaxed to be a ring), consider a matrix and vector whose entries are in $\ensuremath{\bbF}$:
\[
A = \begin{bmatrix}
a_{11} & \ensuremath{\cdots} & a_{1n} \\
\ensuremath{\vdots} & \ensuremath{\ddots} & \ensuremath{\vdots} \\
a_{m1} & \ensuremath{\cdots} & a_{mn}
\end{bmatrix} = \begin{bmatrix} \ensuremath{\bm{\a}}_1 | \ensuremath{\cdots} | \ensuremath{\bm{\a}}_n \end{bmatrix} \ensuremath{\in} \ensuremath{\bbF}^{m \ensuremath{\times} n}, \qquad
\ensuremath{\bm{\x}} = \Vectt[x_1,\ensuremath{\vdots},x_n] \ensuremath{\in} \ensuremath{\bbF}^n.
\]
where $\ensuremath{\bm{\a}}_j = A \ensuremath{\bm{\e}}_j \ensuremath{\in} \ensuremath{\bbF}^m$ are the columns of $A$. Recall the usual definition of matrix multiplication:
\[
A\ensuremath{\bm{\x}} := \begin{bmatrix} \ensuremath{\sum}_{j=1}^n a_{1j} x_j \\ \ensuremath{\vdots} \\ \ensuremath{\sum}_{j=1}^n a_{mj} x_j \end{bmatrix}.
\]
When we are working with floating point numbers $A \ensuremath{\in} F^{m \ensuremath{\times} n}$ we obtain an approximation:
\[
A\ensuremath{\bm{\x}} \ensuremath{\approx} \begin{bmatrix} \ensuremath{\bigoplus}_{j=1}^n (a_{1j}  \ensuremath{\otimes} x_j) \\ \ensuremath{\vdots} \\  \ensuremath{\bigoplus}_{j=1}^n (a_{mj}  \ensuremath{\otimes} x_j) \end{bmatrix}.
\]
This actually encodes an algorithm for computing the entries.

This algorithm uses $O(m n)$ floating point operations (see the appendix if you are unaware of Big-O notation, here our complexities are implicitly taken to be when $m$ or $n$ tends to $\ensuremath{\infty}$): each of the $m$ entries consists of $n$ multiplications and $n-1$ additions, hence we have a total of $2n-1 = O(n)$ operations per row for a total of $m(2n-1) = O(mn)$ operations. For a square matrix this is $O(n^2)$ operations which we call \emph{quadratic complexity}. In the problem sheet we see how the floating point error can be bounded in terms of norms, thus reducing the problem to a purely mathematical concept.

Sometimes there are multiple ways of implementing numerical algorithms. We have an alternative formula where we multiply by columns:
\[
A \ensuremath{\bm{\x}} = x_1 \ensuremath{\bm{\a}}_1  + \ensuremath{\cdots} + x_n \ensuremath{\bm{\a}}_n.
\]
The floating point formula for this is exactly the same as the previous algorithm and the number of operations is the same. Just the order of operations has changed. Suprisingly, this latter version is significantly faster.

\textbf{Remark} Floating point operations are sometimes called FLOPs, which are a standard measurement  of speed of CPUs. However, FLOP sometimes uses an alternative definitions that combines an addition and multiplication as a single FLOP. In the lab we give an example showing that counting the precise number of operations is somewhat of a fools errand: algorithms such as the two approaches for matrix multiplication with the exact same number of operations can have wildly different speeds. We will therefore only be concerned with \emph{complexity}; the asymptotic growth (Big-O) of operations as $n \ensuremath{\rightarrow} \ensuremath{\infty}$, in which case the difference between FLOPs and operations is immaterial.

\subsection{Triangular matrices}
The simplest sparsity case is being triangular: where all entries above or below the diagonal are zero. We consider upper and lower triangular matrices:
\[
U = \begin{bmatrix}
u_{11} & \ensuremath{\cdots} & u_{1n} \\
 & \ensuremath{\ddots} & \ensuremath{\vdots} \\
 &  & u_{nn}
\end{bmatrix}, \qquad L = \begin{bmatrix}
\ensuremath{\ell}_{11} &  \\
\ensuremath{\vdots} & \ensuremath{\ddots} & \\
\ensuremath{\ell}_{n1} & \ensuremath{\cdots} & \ensuremath{\ell}_{nn}
\end{bmatrix}.
\]
Matrix multiplication can be modified to take advantage of the zero pattern of the matrix. Eg., if $L \ensuremath{\in} \ensuremath{\bbF}^{n \ensuremath{\times} n}$ is lower triangular we have:
\[
L\ensuremath{\bm{\x}} = \begin{bmatrix} \ensuremath{\ell}_{1,1} x_1 \\ \ensuremath{\sum}_{j=1}^2 \ensuremath{\ell}_{2j} x_j  \\ \ensuremath{\vdots} \\ \ensuremath{\sum}_{j=1}^n \ensuremath{\ell}_{nj} x_j \end{bmatrix}.
\]
When implemented in floating point this uses roughly half the number of multiplications: $1 + 2 + \ensuremath{\ldots} + n = n(n+1)/2$ multiplications. (It is also about twice as fast in practice.) The complexity is still quadratic: $O(n^2)$ operations.

Triangularity allows us to also invert systems using forward- or back-substitution. In particular if $\ensuremath{\bm{\x}}$ solves $L \ensuremath{\bm{\x}} = \ensuremath{\bm{\b}}$ then we have:
\[
x_k = {b_k - \ensuremath{\sum}_{j=1}^{k-1} \ensuremath{\ell}_{kj} x_j \over \ensuremath{\ell}_{kk}}
\]
Thus we can compute $x_1,x_2,\ensuremath{\ldots},x_n$ in sequence.

\subsection{Banded matrices}
A \emph{banded matrix} is zero off a prescribed number of diagonals. We call the number of (potentially) non-zero diagonals the \emph{bandwidths}:

\begin{definition}[bandwidths] A matrix $A$ has \emph{lower-bandwidth} $l$ if $a_{kj} = 0$ for all $k-j > l$ and \emph{upper-bandwidth} $u$ if $a_{kj} = 0$ for all $j-k > u$. We say that it has \emph{strictly lower-bandwidth} $l$ if it has lower-bandwidth $l$ and there exists a $j$ such that $a_{j+l,j} \ensuremath{\neq} 0$. We say that it has \emph{strictly upper-bandwidth} $u$ if it has upper-bandwidth $u$ and there exists a $k$ such that $a_{k,k+u} \ensuremath{\neq} 0$. \end{definition}

A square banded matrix has the sparsity pattern:
\[
A = \begin{bmatrix}
a_{11} & \ensuremath{\cdots} & a_{1,u+1} \\
\ensuremath{\vdots} & a_{22} & \ensuremath{\ddots} &  a_{2,u+2} \\
a_{1+l,1} & \ensuremath{\ddots} & \ensuremath{\ddots} & \ensuremath{\ddots} & \ensuremath{\ddots} \\
& a_{2+l,2} & \ensuremath{\ddots} & \ensuremath{\ddots} &  \ensuremath{\ddots} & a_{n-u,n} \\
&& \ensuremath{\ddots} & \ensuremath{\ddots} & \ensuremath{\ddots} & \ensuremath{\vdots} \\
&&& a_{n,n-l} & \ensuremath{\cdots} & a_{nn}
\end{bmatrix}
\]
A banded matrix has better complexity for matrix multiplication and solving linear systems:  we can multiply square banded matrices in linear complexity: $O(n)$ operations. We consider two cases in particular (in addition to diagonal): bidiagonal and tridiagonal.

\begin{definition}[Bidiagonal] If a square matrix has bandwidths $(l,u) = (1,0)$ it is \emph{lower-bidiagonal} and if it has bandwidths $(l,u) = (0,1)$ it is \emph{upper-bidiagonal}. \end{definition}

For example, if
\[
L = \begin{bmatrix}\ensuremath{\ell}_{11} \\
\ensuremath{\ell}_{21}& \ensuremath{\ell}_{22} \\ 
& \ensuremath{\ddots} & \ensuremath{\ddots} \\
 &&\ensuremath{\ell}_{n,n-1} &\ensuremath{\ell}_{nn}
\end{bmatrix}
\]
then lower-bidiagonal multiplication becomes
\[
L\ensuremath{\bm{\x}} = \begin{bmatrix} \ensuremath{\ell}_{1,1} x_1 \\ \ensuremath{\ell}_{21} x_1 + \ensuremath{\ell}_{22} x_2    \\ \ensuremath{\vdots} \\ 
\ensuremath{\ell}_{n,n-1} x_{n-1} + \ensuremath{\ell}_{nn} x_n \end{bmatrix}.
\]
This requires $O(1)$ operations per row (at most 2 multiplications and 1 addition) and hence the total is only $O(n)$ operations. A bidiagonal matrix is always triangular and we can also invert in $O(n)$ operations: if $L \ensuremath{\bm{\x}} = \ensuremath{\bm{\b}}$ then $x_1 = b_1/\ensuremath{\ell}_{11}$  and for $k = 2,\ensuremath{\ldots},n$ we can compute
\[
x_k = {b_k - \ensuremath{\ell}_{k-1,k} x_{k-1} \over \ensuremath{\ell}_{kk}}.
\]
\begin{definition}[Tridiagonal] If a square matrix has bandwidths $l = u = 1$ it is \emph{tridiagonal}. \end{definition}

For example,
\[
A = \begin{bmatrix} a_{11} & a_{12} \\
a_{21} & a_{22} & a_{23} \\
 & \ensuremath{\ddots} & \ensuremath{\ddots} & \ensuremath{\ddots} \\
&& a_{n-1,n-2} &                                 a_{n-1,n-1} & a_{n-1,n} \\
&&&a_{n,n-1} & a_{nn}
\end{bmatrix}
\]
is tridiagonal. Matrix multiplication is clearly $O(n)$ operations: each row has $O(1)$ non-zeros and there are $n$ rows. But so is solving linear systems, which we shall see later.




% \input{III.2.DifferentialEquations}
% 
\section{Cholesky factorisation}
In the special case where $A$ is  a real square \emph{symmetric positive definite} ($A \ensuremath{\in} \ensuremath{\bbR}^{n \ensuremath{\times} n}$ such that $A^\ensuremath{\top} = A$ and $\ensuremath{\bm{\x}}^\ensuremath{\top} A \ensuremath{\bm{\x}} > 0$ for all $\ensuremath{\bm{\x}} \ensuremath{\in} \ensuremath{\bbR}^n$, $\ensuremath{\bm{\x}} \ensuremath{\neq} 0$) matrix the LU factorisation has a special form called the \emph{Cholesky factorisation}:
\[
A = L L^\ensuremath{\top},
\]
i.e., $U = L^\ensuremath{\top}$. This provides an algorithmic way to \emph{prove} that a matrix is symmetric positive definite, and is roughly twice as fast as the LU factorisation to compute.

A \emph{Cholesky factorisation} is a form of Gaussian elimination (without pivoting) that exploits symmetry in the problem, resulting in a substantial speedup. It is only applicable for \emph{symmetric positive definite} (SPD) matrices, or rather, the algorithm for computing it succeeds if and only if the matrix is SPD. In other words, it gives an algorithmic way to prove whether or not a matrix is SPD.

\begin{definition}[positive definite] A square matrix $A \ensuremath{\in} \ensuremath{\bbR}^{n \ensuremath{\times} n}$ is \emph{positive definite} if for all $\ensuremath{\bm{\x}} \ensuremath{\in} \ensuremath{\bbR}^n, x \ensuremath{\neq} 0$ we have
\[
\ensuremath{\bm{\x}}^\ensuremath{\top} A \ensuremath{\bm{\x}} > 0
\]
\end{definition}

First we establish some basic properties of positive definite matrices:

\begin{proposition}[conjugating positive definite] If  $A \ensuremath{\in} \ensuremath{\bbR}^{n \ensuremath{\times} n}$ is positive definite and $V \ensuremath{\in} \ensuremath{\bbR}^{n \ensuremath{\times} n}$ is non-singular then
\[
V^\ensuremath{\top} A V
\]
is positive definite. \end{proposition}
\textbf{Proof}

For all  $\ensuremath{\bm{\x}} \ensuremath{\in} \ensuremath{\bbR}^n, \ensuremath{\bm{\x}} \ensuremath{\neq} 0$, define $\ensuremath{\bm{\y}} = V \ensuremath{\bm{\x}} \ensuremath{\neq} 0$ (since $V$ is non-singular). Thus we have
\[
\ensuremath{\bm{\x}}^\ensuremath{\top} V^\ensuremath{\top} A V \ensuremath{\bm{\x}} = \ensuremath{\bm{\y}}^\ensuremath{\top} A \ensuremath{\bm{\y}} > 0.
\]
\ensuremath{\QED}

\begin{proposition}[diag positivity] If $A \ensuremath{\in} \ensuremath{\bbR}^{n \ensuremath{\times} n}$ is positive definite then its diagonal entries are positive: $a_{kk} > 0$. \end{proposition}
\textbf{Proof}
\[
a_{kk} = \ensuremath{\bm{\e}}_k^\ensuremath{\top} A \ensuremath{\bm{\e}}_k > 0.
\]
\ensuremath{\QED}

\begin{lemma}[subslice positive definite] If $A \ensuremath{\in} \ensuremath{\bbR}^{n \ensuremath{\times} n}$ is positive definite and $\ensuremath{\bm{\k}} = [k_1,\ensuremath{\ldots},k_m]^\ensuremath{\top} \ensuremath{\in} \{1,\ensuremath{\ldots},n\}^m$ is a vector of $m$ integers where any integer appears only once,  then $A[\ensuremath{\bm{\k}},\ensuremath{\bm{\k}}] \ensuremath{\in} \ensuremath{\bbR}^{m \ensuremath{\times} m}$ is also positive definite. \end{lemma}
\textbf{Proof} For all $\ensuremath{\bm{\x}} \ensuremath{\in} \ensuremath{\bbR}^m, \ensuremath{\bm{\x}} \ensuremath{\neq} 0$, consider $\ensuremath{\bm{\y}} \ensuremath{\in} \ensuremath{\bbR}^n$ such that $y_{k_j} = x_j$ and zero otherwise. Then we have
\[
\ensuremath{\bm{\x}}^\ensuremath{\top} A[\ensuremath{\bm{\k}},\ensuremath{\bm{\k}}] \ensuremath{\bm{\x}} = \ensuremath{\sum}_{\ensuremath{\ell}=1}^m \ensuremath{\sum}_{j=1}^m x_\ensuremath{\ell} x_j a_{k_\ensuremath{\ell},k_j} = \ensuremath{\sum}_{\ensuremath{\ell}=1}^m \ensuremath{\sum}_{j=1}^m y_{k_\ensuremath{\ell}} y_{k_j} a_{k_\ensuremath{\ell},k_j}  = \ensuremath{\sum}_{\ensuremath{\ell}=1}^n \ensuremath{\sum}_{j=1}^n y_\ensuremath{\ell} y_j a_{\ensuremath{\ell},j} = \ensuremath{\bm{\y}}^\ensuremath{\top} A \ensuremath{\bm{\y}} > 0.
\]
\ensuremath{\QED}

Here is the key result:

\begin{theorem}[Cholesky and SPD] A matrix $A$ is symmetric positive definite if and only if it has a Cholesky factorisation
\[
A = L L^\ensuremath{\top}
\]
where $L$ is lower triangular with positive diagonal entries.

\end{theorem}
\textbf{Proof} If $A$ has a Cholesky factorisation it is symmetric ($A^\ensuremath{\top} = (L L^\ensuremath{\top})^\ensuremath{\top} = A$) and for $\ensuremath{\bm{\x}} \ensuremath{\neq} 0$ we have
\[
\ensuremath{\bm{\x}}^\ensuremath{\top} A \ensuremath{\bm{\x}} = (L^\ensuremath{\top}\ensuremath{\bm{\x}})^\ensuremath{\top} L^\ensuremath{\top} \ensuremath{\bm{\x}} = \|L^\ensuremath{\top}\ensuremath{\bm{\x}}\|^2 > 0
\]
where we use the fact that $L$ is non-singular.

For the other direction we will prove it by induction, with the $1 \ensuremath{\times} 1$ case being trivial. Assume all lower dimensional symmetric positive definite matrices have Cholesky decompositions. Write
\[
A = \begin{bmatrix} \ensuremath{\alpha} & \ensuremath{\bm{\v}}^\ensuremath{\top} \\
                    \ensuremath{\bm{\v}}   & K
                    \end{bmatrix} = \underbrace{\begin{bmatrix} \sqrt{\ensuremath{\alpha}} \\
                                    {\ensuremath{\bm{\v}} \over \sqrt{\ensuremath{\alpha}}} & I \end{bmatrix}}_{L_1}
                                    \begin{bmatrix} 1  \\ & K - {\ensuremath{\bm{\v}} \ensuremath{\bm{\v}}^\ensuremath{\top} \over \ensuremath{\alpha}} \end{bmatrix}
                                    \underbrace{\begin{bmatrix} \sqrt{\ensuremath{\alpha}} & {\ensuremath{\bm{\v}}^\ensuremath{\top} \over \sqrt{\ensuremath{\alpha}}} \\
                                     & I \end{bmatrix}}_{L_1^\ensuremath{\top}}.
\]
Note that $A_2 := K - {\ensuremath{\bm{\v}} \ensuremath{\bm{\v}}^\ensuremath{\top} \over \ensuremath{\alpha}}$ is a subslice of $L_1^{-1} A L_1^{-\ensuremath{\top}}$, hence by combining the previous propositions is itself SPD. Thus we can write
\[
A_2 = K - {\ensuremath{\bm{\v}} \ensuremath{\bm{\v}}^\ensuremath{\top} \over \ensuremath{\alpha}} = L_2 L_2^\ensuremath{\top}
\]
and hence $A = L L^\ensuremath{\top}$ for
\[
L= L_1 \begin{bmatrix}1 \\ & L_2 \end{bmatrix} = \begin{bmatrix} \sqrt{\ensuremath{\alpha}} \\ {\ensuremath{\bm{\v}} \over \sqrt{\ensuremath{\alpha}}} & L_2 \end{bmatrix}
\]
satisfies $A = L L^\ensuremath{\top}$. \ensuremath{\QED}

\begin{example}[Cholesky by hand] Consider the matrix
\[
A = \begin{bmatrix}
2 &1 &1 &1 \\
1 & 2 & 1 & 1 \\
1 & 1 & 2 & 1 \\
1 & 1 & 1 & 2
\end{bmatrix}
\]
Then $\ensuremath{\alpha}_1 = 2$, $\ensuremath{\bm{\v}}_1 = [1,1,1]$, and
\[
A_2 = \begin{bmatrix}
2 &1 &1 \\
1 & 2 & 1 \\
1 & 1 & 2
\end{bmatrix} - {1 \over 2} \begin{bmatrix} 1 \\ 1 \\ 1 \end{bmatrix} \begin{bmatrix} 1 & 1 & 1 \end{bmatrix}
={1 \over 2} \begin{bmatrix}
3 & 1 & 1 \\
1 & 3 & 1 \\
1 & 1 & 3
\end{bmatrix}.
\]
Continuing, we have $\ensuremath{\alpha}_2 = 3/2$, $\ensuremath{\bm{\v}}_2 = [1/2,1/2]$, and
\[
A_3 = {1 \over 2} \left( \begin{bmatrix}
3 & 1 \\ 1 & 3
\end{bmatrix} - {1 \over 3} \begin{bmatrix} 1 \\ 1  \end{bmatrix} \begin{bmatrix} 1 & 1  \end{bmatrix}
\right)
= {1 \over 3} \begin{bmatrix} 4 & 1 \\ 1 & 4 \end{bmatrix}
\]
Next, $\ensuremath{\alpha}_3 = 4/3$, $\ensuremath{\bm{\v}}_3 = [1]$, and
\[
A_4 = [4/3 - 3/4 * (1/3)^2] = [5/4]
\]
i.e. $\ensuremath{\alpha}_4 = 5/4$.

Thus we get
\[
L= \begin{bmatrix}
\sqrt{\ensuremath{\alpha}_1} \\
{\ensuremath{\bm{\v}}_1[1] \over \sqrt{\ensuremath{\alpha}_1}} & \sqrt{\ensuremath{\alpha}_2} \\
{\ensuremath{\bm{\v}}_1[2] \over \sqrt{\ensuremath{\alpha}_1}} & {\ensuremath{\bm{\v}}_2[1] \over \sqrt{\ensuremath{\alpha}_2}}  & \sqrt{\ensuremath{\alpha}_3} \\
{\ensuremath{\bm{\v}}_1[3] \over \sqrt{\ensuremath{\alpha}_1}} & {\ensuremath{\bm{\v}}_2[2] \over \sqrt{\ensuremath{\alpha}_2}}  & {\ensuremath{\bm{\v}}_3[1] \over \sqrt{\ensuremath{\alpha}_3}}  & \sqrt{\ensuremath{\alpha}_4}
\end{bmatrix}
 = \begin{bmatrix} \sqrt{2} \\ {1 \over \sqrt{2}} & \sqrt{3 \over 2} \\
{1 \over \sqrt{2}} & {1 \over \sqrt 6} & {2 \over \sqrt{3}} \\
{1 \over \sqrt{2}} & {1 \over \sqrt 6} & {1 \over \sqrt{12}} & {\sqrt{5} \over 2}
\end{bmatrix}
\]
\end{example}




% \input{III.4.Regression}
% \input{III.5.OrthogonalMatrices}
% \input{III.6.QR}


% \chapter{Approximation Theory}

% So far, we have seen intuitive numerical methods for computing derivatives, integrals, and solving
% differential equations, primarily based on representing functions by their values at a grid of points.
% But by using more sophisticated mathematical tools, we can achieve much more accurate and reliable
% numerical methods. In particular, we can effectively use Fourier series for computing very accurately with periodic functions,
% and orthogonal polynomials for non-periodic functions that are smooth within an interval.
% Here we introduce these fundamental tools and explore applications to quadrature (computing integrals) where they
% produce incredibly accurate approximations, ones that converge exponentially (or faster) for analytic functions.

% \begin{enumerate}
%     \item IV.1 Fourier Expansions: we discuss Fourier series and their usage in approximating periodic functions, using the Trapezium rule to compute the Fourier coefficients.
%     \item IV.2 Discrete Fourier Transform: The Trapezium rule approximation can be recast as a unitary matrix, known as the Discrete Fourier Transform (DFT). This is used to prove interpolation properties.
%     \item IV.3 Orthogonal Polynomials: For non-periodic functions we consider orthogonal polynomials, and discuss their basic properties.
%     \item IV.4 Classical Orthogonal Polynomials: For certain weights, orthogonal polynomials are classical and have addition structure that are useful for computations.
%     \item IV.5 Gaussian Quadrature: Finally, we revisit the  problem of computing integrals, and see that using orthogonal polynomials we can derive much more accurate methods.
%     \end{enumerate}

% We stop at integration, but Fourier and orthogonal polynomial expansions also lead to very effective scheme for solving differential equations
% and many other applications.

% \input{IV.1.Fourier}
% \input{IV.2.DFT}
% \input{IV.3.OrthogonalPolynomials}
% \input{IV.4.ClassicalOPs}
% \input{IV.5.GaussianQuadrature}


% \chapter{Future Directions}

% In these notes we explored some basic numerical algorithms for integration, differentiation, root-finding, solving differential equations, and
% polynomial regression.
% We further saw that rigorous computations and analysis could be deduced by understanding floating point arithmetic, and its usage in
% constructing interval arithmetic.
% We also saw the relationship between these problems and numerical linear algebra, exploring some fundamental algorithms for solving linear
% systems and least squares problems.  We concluded by introducing computational techniques for Fourier series and orthogonal polynomials. 
% Gaussian quadrature was a final example where more sophisticated mathematics can lead to more powerful numerical algorithms,
% achieving exponential convergence for smooth functions.

% This sets the ground work for many areas of computational, applied and even pure mathematics:

% \begin{enumerate}
% \item Machine Learning (ML): Dual numbers (forward-mode automatic differentiation), Newton's method, and polynomial regression
%  are baby versions of how ML trains neural networks.  As an example, consider a neural network trained to recognise whether a picture
%  is a cat or dog. On a basic level the neural network is a function with many parameters that maps from the set of images to ``cat" or ``dog". 
%  The process of training is essentially regression: find the right parameters so that our function equals (or approximately equals) the
%  training data at the given ``samples". Since this problem is nonlinear its not as simple as least squares, instead, it relies on more sophisticated
%  optimisation techniques, that require automatic differentiation to compute gradients for the parameters.
% \item \href{https://sciml.ai}{Scientific Machine Learning (SciML)}: A particular exciting field is the combination of ML and traditional numerical algorithms for differential equations.
% For example, if one is doing climate modelling it is possible to combine differential equations that capture physics with a neural network  trained to 
% fit real-world data. This is an area where Julia has excelled as it allows high-level, differentiable, code that is compiled and hence efficient.
%  \item Computer-assisted proofs: By combining interval arithmetic with methods for solving differential equations important theorems in dynamical
%  systems and elsewhere have been proven. The synthesis of numerical analysis and pure mathematics is really just at the beginning and will likely
%  become increasingly important as we run into the limits of what theorems humans can prove without computer assistance.
% \item Finite Element Methods (FEM): going beyond finite differences, finite elements are powerful techniques for solving partial differential
% equations using specially designed bases built from piecewise polynomials. Gaussian quadrature is a basic tool used to set up the
% sparse linear systems that result from this approach. \href{https://www.firedrakeproject.org}{Firedrake} is a software package for
% PDEs based on FEM that is developed at Imperial.
% \item Spectral methods: orthogonal polynomials in higher dimensions and spherical harmonics (which can be viewed as Fourier series on the sphere)
% underly spectral methods, which are powerful high-accuracy techniques for solving PDEs, used a lot in fluid simulations which
% need more accuracy than conventional FEM allows. \href{https://dedalus-project.org}{Dedalus} is a software package for
% solving fluids problems using spectral methods. At a more basic level \href{https://www.chebfun.org}{Chebfun} and my package 
% \href{https://github.com/JuliaApproximation/ApproxFun.jl}{ApproxFun.jl} give a user friendly way of computing with functions built
% on orthogonal polynomials.
% \end{enumerate}




% \appendix

% \chapter{Asymptotics and Computational Cost}
% 
We introduce Big-O, little-o and asymptotic notation and see how they can be used to describe computational cost. 

\section{Asymptotics as $n \ensuremath{\rightarrow} \ensuremath{\infty}$}
Big-O, little-o, and \ensuremath{\ldq}asymptotic to" are used to describe behaviour of functions at infinity. 

\begin{definition}[Big-O] 
\[
f(n) = O(\ensuremath{\phi}(n)) \qquad \hbox{(as $n \ensuremath{\rightarrow} \ensuremath{\infty}$)}
\]
means $\left|{f(n) \over \ensuremath{\phi}(n)}\right|$ is bounded for sufficiently large $n$. That is, there exist constants $C$ and $N_0$ such  that, for all $n \geq N_0$, $|{f(n) \over \ensuremath{\phi}(n)}| \leq C$. \end{definition}

\begin{definition}[little-O] 
\[
f(n) = o(\ensuremath{\phi}(n)) \qquad \hbox{(as $n \ensuremath{\rightarrow} \ensuremath{\infty}$)}
\]
means $\lim_{n \ensuremath{\rightarrow} \ensuremath{\infty}} {f(n) \over \ensuremath{\phi}(n)} = 0.$ \end{definition}

\begin{definition}[asymptotic to] 
\[
f(n) \ensuremath{\sim} \ensuremath{\phi}(n) \qquad \hbox{(as $n \ensuremath{\rightarrow} \ensuremath{\infty}$)}
\]
means $\lim_{n \ensuremath{\rightarrow} \ensuremath{\infty}} {f(n) \over \ensuremath{\phi}(n)} = 1.$ \end{definition}

\begin{example}[asymptotics with $n$]

\begin{itemize}
\item[1. ] \[
{\cos n \over n^2 -1} = O(n^{-2})
\]
as

\end{itemize}
\[
\left|{{\cos n \over n^2 -1} \over n^{-2}} \right| \leq \left| n^2 \over n^2 -1 \right|  \leq 2
\]
for $n \geq N_0 = 2$.

\begin{itemize}
\item[2. ] \[
\log n = o(n)
\]
as $\lim_{n \ensuremath{\rightarrow} \ensuremath{\infty}} {\log n \over n} = 0.$


\item[3. ] \[
n^2 + 1 \ensuremath{\sim} n^2
\]
as ${n^2 +1 \over n^2} \ensuremath{\rightarrow} 1.$

\end{itemize}
\end{example}

Note we sometimes write $f(O(\ensuremath{\phi}(n)))$ for a function of the form $f(g(n))$ such that $g(n) = O(\ensuremath{\phi}(n))$.

We have some simple algebraic rules:

\begin{proposition}[Big-O rules]
\begin{align*}
O(\ensuremath{\phi}(n))O(\ensuremath{\psi}(n)) = O(\ensuremath{\phi}(n)\ensuremath{\psi}(n))  \qquad \hbox{(as $n \ensuremath{\rightarrow} \ensuremath{\infty}$)} \\
O(\ensuremath{\phi}(n)) + O(\ensuremath{\psi}(n)) = O(|\ensuremath{\phi}(n)| + |\ensuremath{\psi}(n)|)  \qquad \hbox{(as $n \ensuremath{\rightarrow} \ensuremath{\infty}$)}.
\end{align*}
\end{proposition}
\textbf{Proof} See any standard book on asymptotics, eg \href{https://www.taylorfrancis.com/books/mono/10.1201/9781439864548/asymptotics-special-functions-frank-olver}{F.W.J. Olver, Asymptotics and Special Functions}. \ensuremath{\QED}

\section{Asymptotics as $x \ensuremath{\rightarrow} x_0$}
We also have Big-O, little-o and "asymptotic to" at a point:

\begin{definition}[Big-O] 
\[
f(x) = O(\ensuremath{\phi}(x)) \qquad \hbox{(as $x \ensuremath{\rightarrow} x_0$)}
\]
means $|{f(x) \over \ensuremath{\phi}(x)}|$ is bounded in a neighbourhood of $x_0$. That is, there exist constants $C$ and $r$ such  that, for all $0 \leq |x - x_0| \leq r$, $|{f(x) \over \ensuremath{\phi}(x)}| \leq C$. \end{definition}

\begin{definition}[little-O] 
\[
f(x) = o(\ensuremath{\phi}(x)) \qquad \hbox{(as $x \ensuremath{\rightarrow} x_0$)}
\]
means $\lim_{x \ensuremath{\rightarrow} x_0} {f(x) \over \ensuremath{\phi}(x)} = 0.$ \end{definition}

\begin{definition}[asymptotic to] 
\[
f(x) \ensuremath{\sim} \ensuremath{\phi}(x) \qquad \hbox{(as $x \ensuremath{\rightarrow} x_0$)}
\]
means $\lim_{x \ensuremath{\rightarrow} x_0} {f(x) \over \ensuremath{\phi}(x)} = 1.$ \end{definition}

\begin{example}[asymptotics with $x$]
\[
\exp x = 1 + x + O(x^2) \qquad \hbox{as $x \ensuremath{\rightarrow} 0$}
\]
since $\exp x = 1 + x + {\exp t \over 2} x^2$ for some $t \in [0,x]$ and
\[
\left|{{\exp t \over 2} x^2 \over x^2}\right| \leq {3 \over 2}
\]
provided $x \leq 1$. \end{example}

\section{Computational cost}
We will use Big-O notation to describe the computational cost of algorithms. Consider the following simple sum
\[
\sum_{k=1}^n x_k^2
\]
which we might implement as:


\begin{lstlisting}
(*@\HLJLk{function}@*) (*@\HLJLnf{sumsq}@*)(*@\HLJLp{(}@*)(*@\HLJLn{x}@*)(*@\HLJLp{)}@*)
    (*@\HLJLn{n}@*) (*@\HLJLoB{=}@*) (*@\HLJLnf{length}@*)(*@\HLJLp{(}@*)(*@\HLJLn{x}@*)(*@\HLJLp{)}@*)
    (*@\HLJLn{ret}@*) (*@\HLJLoB{=}@*) (*@\HLJLnfB{0.0}@*)
    (*@\HLJLk{for}@*) (*@\HLJLn{k}@*) (*@\HLJLoB{=}@*) (*@\HLJLni{1}@*)(*@\HLJLoB{:}@*)(*@\HLJLn{n}@*)
        (*@\HLJLn{ret}@*) (*@\HLJLoB{=}@*) (*@\HLJLn{ret}@*) (*@\HLJLoB{+}@*) (*@\HLJLn{x}@*)(*@\HLJLp{[}@*)(*@\HLJLn{k}@*)(*@\HLJLp{]}@*)(*@\HLJLoB{{\textasciicircum}}@*)(*@\HLJLni{2}@*)
    (*@\HLJLk{end}@*)
    (*@\HLJLn{ret}@*)
(*@\HLJLk{end}@*)
\end{lstlisting}

\begin{lstlisting}
sumsq (generic function with 1 method)
\end{lstlisting}


Each step of this algorithm consists of one memory look-up (\texttt{z = x[k]}), one multiplication (\texttt{w = z*z}) and one addition (\texttt{ret = ret + w}). We will ignore the memory look-up in the following discussion. The number of CPU operations per step is therefore 2 (the addition and multiplication). Thus the total number of CPU operations is $2n$. But the constant $2$ here is misleading: we didn't count the memory look-up, thus it is more sensible to just talk about the asymptotic complexity, that is, the \emph{computational cost} is $O(n)$.

Now consider a double sum like:
\[
\sum_{k=1}^n \sum_{j=1}^k x_j^2
\]
which we might implement as:


\begin{lstlisting}
(*@\HLJLk{function}@*) (*@\HLJLnf{sumsq2}@*)(*@\HLJLp{(}@*)(*@\HLJLn{x}@*)(*@\HLJLp{)}@*)
    (*@\HLJLn{n}@*) (*@\HLJLoB{=}@*) (*@\HLJLnf{length}@*)(*@\HLJLp{(}@*)(*@\HLJLn{x}@*)(*@\HLJLp{)}@*)
    (*@\HLJLn{ret}@*) (*@\HLJLoB{=}@*) (*@\HLJLnfB{0.0}@*)
    (*@\HLJLk{for}@*) (*@\HLJLn{k}@*) (*@\HLJLoB{=}@*) (*@\HLJLni{1}@*)(*@\HLJLoB{:}@*)(*@\HLJLn{n}@*)
        (*@\HLJLk{for}@*) (*@\HLJLn{j}@*) (*@\HLJLoB{=}@*) (*@\HLJLni{1}@*)(*@\HLJLoB{:}@*)(*@\HLJLn{k}@*)
            (*@\HLJLn{ret}@*) (*@\HLJLoB{=}@*) (*@\HLJLn{ret}@*) (*@\HLJLoB{+}@*) (*@\HLJLn{x}@*)(*@\HLJLp{[}@*)(*@\HLJLn{j}@*)(*@\HLJLp{]}@*)(*@\HLJLoB{{\textasciicircum}}@*)(*@\HLJLni{2}@*)
        (*@\HLJLk{end}@*)
    (*@\HLJLk{end}@*)
    (*@\HLJLn{ret}@*)
(*@\HLJLk{end}@*)
\end{lstlisting}

\begin{lstlisting}
sumsq2 (generic function with 1 method)
\end{lstlisting}


Now the inner loop is $O(1)$ operations (we don't try to count the precise number), which we do $k$ times for $O(k)$ operations as $k \ensuremath{\rightarrow} \ensuremath{\infty}$. The outer loop therefore takes
\[
\ensuremath{\sum}_{k = 1}^n O(k) = O\left(\ensuremath{\sum}_{k = 1}^n k\right) = O\left( {n (n+1) \over 2} \right) = O(n^2)
\]
operations.





% \chapter{Permutation Matrices}
% 
Permutation matrices are matrices that represent the action of permuting the entries of a vector, that is, matrix representations of the symmetric group $S_n$, acting on $\ensuremath{\bbR}^n$. Recall every $\ensuremath{\sigma} \ensuremath{\in} S_n$ is a bijection between $\{1,2,\ensuremath{\ldots},n\}$ and itself. We can write a permutation $\ensuremath{\sigma}$ in \emph{Cauchy notation}:
\[
\begin{pmatrix}
 1 & 2 & 3 & \ensuremath{\cdots} & n \cr
 \ensuremath{\sigma}_1 & \ensuremath{\sigma}_2 & \ensuremath{\sigma}_3 & \ensuremath{\cdots} & \ensuremath{\sigma}_n
 \end{pmatrix}
\]
where $\{\ensuremath{\sigma}_1,\ensuremath{\ldots},\ensuremath{\sigma}_n\} = \{1,2,\ensuremath{\ldots},n\}$ (that is, each integer appears precisely once). We denote the \emph{inverse permutation} by $\ensuremath{\sigma}^{-1}$, which can be constructed by swapping the rows of the Cauchy notation and reordering.

We can encode a permutation in vector $\mathbf \ensuremath{\sigma} = [\ensuremath{\sigma}_1,\ensuremath{\ldots},\ensuremath{\sigma}_n]$.  This induces an action on a vector (using indexing notation)
\[
\ensuremath{\bm{\v}}[\mathbf \ensuremath{\sigma}] = \begin{bmatrix}v_{\ensuremath{\sigma}_1}\\ \vdots \\ v_{\ensuremath{\sigma}_n} \end{bmatrix}
\]
\begin{example}[permutation of a vector]  Consider the permutation $\ensuremath{\sigma}$ given by
\[
\begin{pmatrix}
 1 & 2 & 3 & 4 & 5 \cr
 1 & 4 & 2 & 5 & 3
 \end{pmatrix}
\]
We can apply it to a vector:


\begin{lstlisting}
(*@\HLJLk{using}@*) (*@\HLJLn{LinearAlgebra}@*)
(*@\HLJLn{\ensuremath{\sigma}}@*) (*@\HLJLoB{=}@*) (*@\HLJLp{[}@*)(*@\HLJLni{1}@*)(*@\HLJLp{,}@*) (*@\HLJLni{4}@*)(*@\HLJLp{,}@*) (*@\HLJLni{2}@*)(*@\HLJLp{,}@*) (*@\HLJLni{5}@*)(*@\HLJLp{,}@*) (*@\HLJLni{3}@*)(*@\HLJLp{]}@*)
(*@\HLJLn{v}@*) (*@\HLJLoB{=}@*) (*@\HLJLp{[}@*)(*@\HLJLni{6}@*)(*@\HLJLp{,}@*) (*@\HLJLni{7}@*)(*@\HLJLp{,}@*) (*@\HLJLni{8}@*)(*@\HLJLp{,}@*) (*@\HLJLni{9}@*)(*@\HLJLp{,}@*) (*@\HLJLni{10}@*)(*@\HLJLp{]}@*)
(*@\HLJLn{v}@*)(*@\HLJLp{[}@*)(*@\HLJLn{\ensuremath{\sigma}}@*)(*@\HLJLp{]}@*) (*@\HLJLcs{{\#}}@*) (*@\HLJLcs{we}@*) (*@\HLJLcs{permutate}@*) (*@\HLJLcs{entries}@*) (*@\HLJLcs{of}@*) (*@\HLJLcs{v}@*)
\end{lstlisting}

\begin{lstlisting}
5-element Vector(*@{{\{}}@*)Int64(*@{{\}}}@*):
  6
  9
  7
 10
  8
\end{lstlisting}


Its inverse permutation $\ensuremath{\sigma}^{-1}$ has Cauchy notation coming from swapping the rows of the Cauchy notation of $\ensuremath{\sigma}$ and sorting:
\[
\begin{pmatrix}
 1 & 4 & 2 & 5 & 3 \cr
 1 & 2 & 3 & 4 & 5
 \end{pmatrix} \rightarrow \begin{pmatrix}
 1 & 2 & 4 & 3 & 5 \cr
 1 & 3 & 2 & 5 & 4
 \end{pmatrix} 
\]
\end{example}

Note that the operator
\[
P_\ensuremath{\sigma}(\ensuremath{\bm{\v}}) = \ensuremath{\bm{\v}}[{\mathbf \ensuremath{\sigma}}]
\]
is linear in $\ensuremath{\bm{\v}}$, therefore, we can identify it with a matrix whose action is:
\[
P_\ensuremath{\sigma} \begin{bmatrix} v_1\\ \vdots \\ v_n \end{bmatrix} = \begin{bmatrix}v_{\ensuremath{\sigma}_1} \\ \vdots \\ v_{\ensuremath{\sigma}_n}  \end{bmatrix}.
\]
The entries of this matrix are
\[
P_\ensuremath{\sigma}[k,j] = \ensuremath{\bm{\e}}_k^\ensuremath{\top} P_\ensuremath{\sigma} \ensuremath{\bm{\e}}_j = \ensuremath{\bm{\e}}_k^\ensuremath{\top} \ensuremath{\bm{\e}}_{\ensuremath{\sigma}^{-1}_j} = \ensuremath{\delta}_{k,\ensuremath{\sigma}^{-1}_j} = \ensuremath{\delta}_{\ensuremath{\sigma}_k,j}
\]
where $\ensuremath{\delta}_{k,j}$ is the \emph{Kronecker delta}:
\[
\ensuremath{\delta}_{k,j} := \begin{cases} 1 & k = j \\
                        0 & \hbox{otherwise}
                        \end{cases}.
\]
This construction motivates the following definition:

\begin{definition}[permutation matrix] $P \in \ensuremath{\bbR}^{n \ensuremath{\times} n}$ is a permutation matrix if it is equal to the identity matrix with its rows permuted. \end{definition}

\begin{proposition}[permutation matrix inverse]  Let $P_\ensuremath{\sigma}$ be a permutation matrix corresponding to the permutation $\ensuremath{\sigma}$. Then
\[
P_\ensuremath{\sigma}^\ensuremath{\top} = P_{\ensuremath{\sigma}^{-1}} = P_\ensuremath{\sigma}^{-1}
\]
That is, $P_\ensuremath{\sigma}$ is \emph{orthogonal}:
\[
P_\ensuremath{\sigma}^\ensuremath{\top} P_\ensuremath{\sigma} = P_\ensuremath{\sigma} P_\ensuremath{\sigma}^\ensuremath{\top} = I.
\]
\end{proposition}
\textbf{Proof}

We prove orthogonality via:
\[
\ensuremath{\bm{\e}}_k^\ensuremath{\top} P_\ensuremath{\sigma}^\ensuremath{\top} P_\ensuremath{\sigma} \ensuremath{\bm{\e}}_j = (P_\ensuremath{\sigma} \ensuremath{\bm{\e}}_k)^\ensuremath{\top} P_\ensuremath{\sigma} \ensuremath{\bm{\e}}_j = \ensuremath{\bm{\e}}_{\ensuremath{\sigma}^{-1}_k}^\ensuremath{\top} \ensuremath{\bm{\e}}_{\ensuremath{\sigma}^{-1}_j} = \ensuremath{\delta}_{k,j}
\]
This shows $P_\ensuremath{\sigma}^\ensuremath{\top} P_\ensuremath{\sigma} = I$ and hence $P_\ensuremath{\sigma}^{-1} = P_\ensuremath{\sigma}^\ensuremath{\top}$. 

\ensuremath{\QED}







\end{document}